% URT
% -*-LaTeX-*-
% Converted automatically from troff to LaTeX by tr2tex on Tue Aug  7 18:10:14 1990
% tr2tex was written by Kamal Al-Yahya at Stanford University
% (Kamal%Hanauma@SU-SCORE.ARPA)


\documentstyle[troffman]{article}
\begin{document}
%
% input file: applymap.1
%
% Copyright (c) 1986, University of Utah
\phead{APPLYMAP}{1}{Nov\ 12,\ 1986}
 1
\shead{NAME}
applymap -- Apply the color map in an RLE file to the pixel data
\shead{SYNOPSIS}
{\bf applymap}
[
{\bf --l}
] [
{\bf --o}
{\it outfile}
] [ 
{\it infile}
] 
\shead{DESCRIPTION}
This program takes the color map in an
{\it RLE}{\rm (5)}
file and modifies the pixel values by applying the color map to them.
If there is more than one color channel in the input file,
the color map in the input file should have the same number of
channels.  If the input file has a single color channel, the
output file will have the same number of color channels as the color
map.

Each pixel in the input file is mapped as follows:  For a
multi-channel input file, a pixel in channel
{\it i}
is mapped as
{\it map[i][pixel]\ $>$$>$\ 8}{\rm ,}
producing a pixel in output channel
{\it i}{\rm .}
The right shift takes the 16 bit color map value to an 8 bit pixel
value.  For a single channel input file, to produce a pixel in output
channel
{\it i}
is produced from the corresponding input pixel value
as
{\it map[i][pixel]\ $>$$>$\ 8}{\rm .}
\shead{OPTIONS}
\begin{TPlist}{{\bf --l}}
\item[{{\bf --l}}]
This option will cause a linear (identity) color map to be loaded into the
output file.  Otherwise, the output file will have no color map.
\item[{{\it infile}}]
The input will be read from this file, otherwise, input will
be taken from stdin.
\item[{{\bf --o}{\it \ outfile}
}]
If specified, output will be written to this file, otherwise it will
go to stdout.
\end{TPlist}\shead{SEE ALSO}
{\it rleldmap}{\rm (1),}
{\it urt}{\rm (1),}
{\it RLE}{\rm (5).}
\shead{AUTHOR}
Spencer W. Thomas, University of Utah
\shead{BUGS}
If the image data and color map channels in the input file do not conform to
the restriction stated above (NN or 1N) the program will most likely
core dump.
\newpage
% -*-LaTeX-*-
% Converted automatically from troff to LaTeX by tr2tex on Tue Aug  7 18:10:14 1990
% tr2tex was written by Kamal Al-Yahya at Stanford University
% (Kamal%Hanauma@SU-SCORE.ARPA)


%[troffman]{article}
%
%
% input file: avg4.1
%
% Copyright (c) 1986, University of Utah
% Template man page.  Taken from wtm's page for getcx3d
\phead{AVG4}{1}{Nov\ 12,\ 1986}
 1
\shead{NAME}
avg4 -- Downfilter an image by simple averaging.
\shead{SYNOPSIS}
{\bf avg4}
[
{\bf --o}
{\it outfile}
] [ 
{\it infile}
] 
\shead{DESCRIPTION}
{\it Avg4}
downfilters an RLE image into a resulting image of 1/4th the size,
by simply averaging four pixel values in the input image to produce a single
pixel in the output.  If the original image does not contain an alpha
channel, avg4 creates one by counting the number of non-zero pixels in each
group of four input pixels and using the count to produce a coverage value. 
While the alpha channel produced this way is crude (only four levels of
coverage) it is enough to make a noticeable improvement in the edges of 
composited images.  
\shead{OPTIONS}
\begin{TPlist}{{\it infile}}
\item[{{\it infile}}]
The input will be read from this file, otherwise, input will
be taken from stdin.
\item[{{\bf --o}{\it \ outfile}
}]
If specified, output will be written to this file, otherwise it will
go to stdout.
\end{TPlist}\shead{SEE ALSO}
{\it fant}{\rm (1),}
{\it rlecomp}{\rm (1),}
{\it smush}{\rm (1),}
{\it urt}{\rm (1),}
{\it RLE}{\rm (5).}
\shead{AUTHOR}
Rod Bogart, John W. Peterson
\shead{BUGS}
Very simple minded -- more elaborate filters could be implemented.

\newpage
% -*-LaTeX-*-
% Converted automatically from troff to LaTeX by tr2tex on Tue Aug  7 18:10:15 1990
% tr2tex was written by Kamal Al-Yahya at Stanford University
% (Kamal%Hanauma@SU-SCORE.ARPA)


%[troffman]{article}
%
%
% input file: crop.1
%
% Copyright (c) 1986, University of Utah
% Template man page.  Taken from wtm's page for getcx3d
\phead{CROP}{1}{Nov\ 12,\ 1986}
 1
\shead{NAME}
crop -- Change the size of an RLE image
\shead{SYNOPSIS}
{\bf crop} 
[
{\bf --b}
] [
{\it xmin\ ymin\ xmax\ ymax}
]
[
{\bf --o}
{\it outfile}
] [ 
{\it infile} 
] 
\shead{DESCRIPTION}
{\it }
Crop changes the size of an RLE image.  The command line numbers
{\it xmin\ ymin\ xmax\ ymax}
specify the bounds of the resulting image.  If the resulting image is larger 
than the original, 
{\it crop}
supplies blank pixels, otherwise pixels are thrown away.
\shead{OPTIONS}
\begin{TPlist}{{\bf --b}}
\item[{{\bf --b}}]
The input image is cropped to the enclosing box.  Extra rows and columns of 
black pixels are removed.  The %
\it infile %
\rm must be a file; no piped input is
allowed for this option.
\item[{{\bf --o}{\it \ outfile}
}]
If specified, output will be written to this file, otherwise it will
go to stdout.
\item[{{\it infile}}]
The input will be read from this file, otherwise, input will
be taken from stdin.
\end{TPlist}\shead{SEE ALSO}
{\it repos(1),}
{\it urt}{\rm (1),}
{\it RLE}{\rm (5).}
\shead{AUTHOR}
Rod Bogart
\shead{BUGS}
Could be combined with 
{\it repos}{\rm .}
Does not check to see if the input and output regions are disjoint.


\newpage
% -*-LaTeX-*-
% Converted automatically from troff to LaTeX by tr2tex on Tue Aug  7 18:10:15 1990
% tr2tex was written by Kamal Al-Yahya at Stanford University
% (Kamal%Hanauma@SU-SCORE.ARPA)


%[troffman]{article}
%
%
% input file: cubitorle.1
%
% Copyright (c) 1988, University of Utah
\phead{CUBITORLE}{1}{6\ February\ 1988}
 1
\shead{NAME}
cubitorle -- Convert cubicomp image to an RLE format file.
\shead{SYNOPSIS}
{\bf cubitorle}
[
{\bf --o} 
{\it outfile}
] 
{\it inprefix}
\shead{DESCRIPTION}
{\it Cubitorle}
converts a set of files in the Cubicomp image format to a raster file in the 
Utah Raster Toolkit RLE format.
{\it Cubitorle}
expects as input a set of files of the form "%
\it inprefix%
\rm .r8", 
"%
\it inprefix%
\rm .g8", and "%
\it inprefix%
\rm .b8".  These files are combined
to form a single 
{\it RLE}{\rm (5)}
file.
The output is written to
{\it stdout}
unless an output file name is given using the
{\bf --o}
option.
\shead{OPTIONS}
\begin{TPlist}{{\bf --o}
}
\item[{{\bf --o}
}]
Allows specification of an output file name.
\end{TPlist}\shead{SEE ALSO}
{\it rleflip}{\rm (1),}
{\it urt}{\rm (1),}
{\it RLE}{\rm (5).}
\shead{AUTHOR}
Rod Bogart
\newpage
% -*-LaTeX-*-
% Converted automatically from troff to LaTeX by tr2tex on Tue Aug  7 18:10:16 1990
% tr2tex was written by Kamal Al-Yahya at Stanford University
% (Kamal%Hanauma@SU-SCORE.ARPA)


%[troffman]{article}
%
%
% input file: dvirle.1
%
\phead{DVIRLE}{1}{May\ 12,\ 1987}
 1
\shead{NAME}
dvirle -- convert dvi version 2 files, produced by TeX82, to RLE images
\shead{SYNOPSIS}
{\bf dvirle}
[
{\bf --m}
{\it number}
] [
{\bf --h}
] [
{\bf --s}
] [
{\bf --d}
{\it number}
] [
{\bf --x}
{\it xfilter}
] [
{\bf --y}
{\it yfilter}
]
{\it infile.dvi}
\shead{DESCRIPTION}
{\it Dvirle}
converts .dvi files produced by
{\it TeX}{\rm (1)}
to
{\it RLE}{\rm (5)}
format.  The basic process involves two passes.  In the first
pass, the .dvi file is converted into a list of characters.
The second pass takes this list and converts it to
{\it RLE}{\rm .}
The image is filtered to produce gray-scale letters.  300dpi fonts are
used, producing an unfiltered page size of approximately 25003500
pixels.  The default is to average this by 5 pixels in the X direction
and 5 in the Y, producing a 510708 image.  The filtering parameters can be
altered with the 
{\bf --x}
and
{\bf --y}
flags.

The
{\bf --m}
{\it number}
option is used to change the device magnification (which is in addition to
any magnification defined in the TeX source file).
{\it Number}
should be replaced by an integer which is 1000 times the magnification
you want.
for example,
{\bf --m} 1315
would produce output magnified to 131.5\% of true size.  The default is
no magnification (1000).  Note, however, that a site will only
support particular magnifications.  If you get error messages indicating that
fonts are missing when using this option, you probably have picked an
unsupported magnification.

The
{\bf --h}
flag, when
supplied, causes the image to be converted "on its side" (rotated by
90 degrees).

Normally the first pass prints the page numbers from the .dvi file.  The
{\bf --s}
flag suppresses these.

The default
{\it maxdrift}
parameter is 2 pixels (1/100th of an inch); the
{\bf --d}
option may be used to alter this.  The
{\it maxdrift}
parameter determines just how much font spacing is allowed to
influence character positioning.  The default value 2 allows a small
amount of variation within words without allowing any letters to
become too far out of position.

The output file contains a number of separate 
{\it RLE}
images concatenated, one for each page in the input.  These can be
separated with
{\it rlesplit}{\rm (1).}
The output images have a single image channel and
an identical "alpha" channel.  For compositing with a colored
background, it will be necessary to use 
{\it rleswap}{\rm (1)}
to expand it to 3 color channels.

The shell script 
{\it topcrop}
will crop off the top 384 lines of the output image (assuming the
default %
\it LaTeX \rm%
page size and %
\it dvirle \rm%
filtering parameters),
making it suitable for viewing on a (384512) frame buffer.
\nofill
    topcrop $<$file.rle $>$cropfile.rle
\fill

A better solution is to use something like the following %
\it LaTeX \rm%
macros to set the page size so that, with the default filter
parameters, the output images will be 510384.
\nofill
%.ta 1in 4in
\bs newcommand\{\bs maxpage\}\{	\%\% Make page as large as possible
	\bs setlength\{\bs topmargin\}\{0in\}
	\bs setlength\{\bs oddsidemargin\}\{0pt\}
	\bs setlength\{\bs evensidemargin\}\{0pt\}
	\bs setlength\{\bs marginparwidth\}\{0pt\}
	\bs setlength\{\bs marginparsep\}\{0pt\}
	\bs setlength\{\bs headheight\}\{0pt\}
	\bs setlength\{\bs headsep\}\{0pt\}
	\bs setlength\{\bs textwidth\}\{6.5in\}\}
\bs newcommand\{\bs plainpage\}\{	\%\% Page with space for headers
	\bs pagestyle\{plain\}
	\bs setlength\{\bs textheight\}\{4.0667in\}
	\bs setlength\{\bs footheight\}\{12pt\}
	\bs setlength\{\bs footskip\}\{24pt\}
	\bs maxpage\}
			
\bs newcommand\{\bs headingspage\}\{	\%\% Page with headers
	\bs pagestyle\{headings\}
	\bs setlength\{\bs textheight\}\{4.0667in\}
	\bs setlength\{\bs footheight\}\{12pt\}
	\bs setlength\{\bs footskip\}\{24pt\}
	\bs maxpage\}
\bs newcommand\{\bs emptypage\}\{	\%\% Page with no headers
	\bs pagestyle\{empty\}
	\bs setlength\{\bs textheight\}\{4.4in\}
	\bs setlength\{\bs footheight\}\{0pt\}
	\bs setlength\{\bs footskip\}\{0pt\}
	\bs maxpage\}
\fill
\shead{FILES}
{\it dvirle1}{\rm \ \ \ \ first\ pass}
\nwl
{\it dvirle2}{\rm \ \ \ \ second\ pass}
\shead{SEE ALSO}
\raggedright
{\it rleflip}{\rm (1),}
{\it rlesplit}{\rm (1),}
{\it rleswap}{\rm (1),}
{\it urt}{\rm (1),}
{\it RLE}{\rm (5).}
%.ad b
\shead{AUTHOR}
The original (Versatec) version was written by Janet Incerpi of Brown
University.  Richard Furuta and Carl Binding of the University of
Washington modified the programs for DVI version 2 files.  Chris Torek
of the University of Maryland rewrote both passes in order to make
them run at reasonable speeds.  Spencer W. Thomas of the University of
Utah converted it to produce RLE images as output.
\shead{BUGS}
The %
\bf --h \rm%
option doesn't work properly.  Use 
{\it rleflip}{\rm (1)}
instead.

Truncates pages wider than 2550 pixels (8.5 inches).

Doesn't handle missing fonts gracefully.

Should be a single program, instead of a shell script and two
programs.  Doesn't use the usual RLE argument and file name
conventions.  Should output the TeX page numbers as picture comments.
\newpage
% -*-LaTeX-*-
% Converted automatically from troff to LaTeX by tr2tex on Tue Aug  7 18:10:17 1990
% tr2tex was written by Kamal Al-Yahya at Stanford University
% (Kamal%Hanauma@SU-SCORE.ARPA)


%[troffman]{article}
%
%
% input file: fant.1
%
% Copyright (c) 1986, University of Utah
\phead{FANT}{1}{June\ 15,\ 1990}
 1
\shead{NAME}
fant -- perform simple spatial transforms on an image
\shead{SYNOPSIS}
{\bf fant}
% sample options...
[
{\bf --a}
{\it angle}
] 
[
{\bf --b}
]
[
{\bf --o}
{\it outfile}
]
[
{\bf --p}
{\it xoff\ yoff}
]
[
{\bf --s} 
{\it xscale\ yscale}
]
[
{\bf --v}
]
[ 
{\it infile}
] 
\shead{DESCRIPTION}
{\it Fant}
rotates or scales an image by an arbitrary amount.  It does this by
using pixel integration (if the image size is reduced) or pixel interpolation
if the image size is increased.  Because it works with subpixel precision,
aliasing artifacts are not introduced (hah! see BUGS).  
{\it Fant}
uses a two-pass 
sampling technique to perform the transformation.  If
{\it infile}
is "--" or absent, input is read from the standard input.
\shead{OPTIONS}
\begin{TPlist}{{\bf --a}{\it \ angle}
}
\item[{{\bf --a}{\it \ angle}
}]
Amount to rotate image by, a real number from 0 to 45 degrees (positive
numbers rotate clockwise).  Use rleflip(1) first to rotate an image by larger
amounts.
\item[{{\bf --b}}]
Blur the resulting image. 
Always interpolate between pixels of the input image.  This results in
fewer artifacts but a slightly blurry resulting image.  Recommended for
pathological cases such as single pixel wide lines in the input image.
\item[{{\bf --o}{\it \ outfile}
}]
Specifies where to place the resulting image.  The default is to write
to stdout.  If
{\it outfile}
is "--", the output will be written to the standard output stream.
\item[{{\bf --p}{\it \ xoff\ yoff}
}]
Specifies where the origin of the image is -- the image is rotated or scaled
about this point.  If no origin is specified, the center of the image is used.
\item[{{\bf --s}{\it \ xscale\ yscale}
}]
The amount (in real numbers) to scale an image by.  This is often
useful for correcting the aspect of an image for display on a frame
buffer with non square pixels.  For this use, the origin should be
specified as 0, 0 (see %
\bf --p \rm%
above).  If an image is only scaled
in Y and no rotation is performed,
{\it fant}
only uses one sampling pass over the image, cutting the computation time
in half.
\item[{{\bf --v}}]
Verbose output.  Primarily for debugging.
\end{TPlist}\par\noindent
\shead{SEE ALSO}
{\it avg4}{\rm (1),}
{\it rleflip}{\rm (1),}
{\it urt}{\rm (1),}
{\it RLE}{\rm (5),}
\nwl
Fant, Karl M. "A Nonaliasing, Real-Time, Spatial Transform Technique",
%
\it IEEE CG\&A\rm%
, January, 1986, p. 71.
\shead{AUTHORS}
John W. Peterson,
James S. Painter
\shead{BUGS}
{\it Fant}
uses a rather poor anti-aliasing filter (a box filter).  This is usually
good enough but will exhibit noticeable aliasing artifacts on nasty
input images.
\newpage
% -*-LaTeX-*-
% Converted automatically from troff to LaTeX by tr2tex on Tue Aug  7 18:10:17 1990
% tr2tex was written by Kamal Al-Yahya at Stanford University
% (Kamal%Hanauma@SU-SCORE.ARPA)


%[troffman]{article}
%
%
% input file: get4d.1
%
% Copyright (c) 1989, University of Utah
\phead{GET4D}{1}{June\ 20,\ 1989}
 1
\shead{NAME}
get4d -- get RLE images to a Silicon Graphics Iris/4D display
\shead{SYNOPSIS}
{\bf get4d}
[
{\bf --D}
] [
{\bf --f}
] [
{\bf --\{GS}\}
] [
{\bf --g}
{\it disp\_gamma}
] [
{\bf --\{iI}\}
{\it image\_gamma}
] [
{\bf --n}
] [
{\bf --p}
{\it xpos\ ypos}
] [
{\bf --s}
{\it xsize\ ysize}
] [
{\bf --w}
] [ 
{\it infile}
]

\shead{DESCRIPTION}
This program displays an
{\it RLE}{\rm (5)}
file on a
{\it Silicon} Graphics Iris/4D
display or
{\it IBM} RS6000
with the GL library.

The default behavior is to display the image in RGB color.  An option is
provided to force black and white display.  There is currently no support in
{\it get4D}
for non-24-bit color (lookup table modes), but the
{\it getmex} (1)
program should work on 8-bit 4D's which cannot do RGB display.

The GT graphics fast pixel access routines are used by default on 4D/GT and
GTX machines, and Personal Irises.  The
{\bf --G}
option is provided to force this mode, if the string returned by the
{\it gversion}{\rm (3g)}
function changes, or is different on future 4D's.

The penalty of GT mode is not being able to resize or pan the window, but
redisplay is so fast that there is no need to do so.  You can also go into
"slow mode" on GT machines by giving the
{\bf --S}
flag.  Slow mode allows resizing the window and panning with the mouse.
\shead{OPTIONS}
\begin{TPlist}{{\bf %
\bf --p} %
\it xpos\ ypos%
\rm }
\item[{{\bf %
\bf --p} %
\it xpos\ ypos%
\rm }]
Position of the lower left corner of the window.
\item[{{\bf %
\bf --s} %
\it xsize\ ysize%
\rm }]
Initial size of the window (slow mode only.)
\item[{{\bf --f}}]
Normally,
{\it get4d}
will fork itself after putting the image on the screen, so that the
parent process may return the shell, leaving an "invisible" child to
keep the image refreshed.  If 
{\bf --f}
is specified, get4d will remain attached to the shell, whence it may be killed
with an interrupt signal.  In either case the window manager "quit" menu
button can be used to kill 
{\it get4d.}
\item[{{\bf --g}{\it \ display\_gamma}
}]
Specify the gamma of the display monitor.  If this flag is not specified,
{\it get4d}
looks in the user's home directory for a .gamma file.  This file is 
produced by the
{\it gamma}{\rm (1g)}
SGI command (This is not done on the IBM R6000).  The value in the .gamma
file is used to determine the gamma of the display by calculating (2.4 /
{\it gamma\_value}
) and using that as the 
{\it disp\_gamma.}
\item[{{\bf --i}{\it \ image\_gamma}
}]
Specify the gamma (contrast) of the image.  A low contrast image,
suited for direct display without compensation on a high contrast
monitor (as most monitors are) will have a gamma of less than one.
The default image gamma is 1.0.  Image gamma may also be specified by
a picture comment in the
{\it RLE} (5)
file of the form
{\bf image\_gamma=}{\it gamma.}
The command line argument will override the value in the file if specified.
\item[{{\bf --I}{\it \ image\_gamma}
}]
An alternate method of specifying the image gamma, the number
following
{\bf --I}
is the gamma of the display for which the image was originally
computed (and is therefore 1.0 divided by the actual gamma of the
image).  Image display gamma may also be specified by
a picture comment in the
{\it RLE} (5)
file of the form
{\bf display\_gamma=}{\it gamma.}
The command line argument will override the value in the file if specified.
\item[{{\bf --n}}]
Do not draw a window border.
\item[{{\bf --w}}]
This flag forces
{\it get4d}
to produce a gray scale dithered image instead of a color image.
Color input will be transformed to black and white via the
{\it NTSC} Y
transform.
\item[{{\bf --D}}]
"Debug mode".  The operations in the input
{\it RLE}{\rm (5)}
file will be printed as they are read.
\item[{{\it file}}]
Name of the
{\it RLE}{\rm (5)}
file to display.  If not specified, the image will be read from the
standard input.
\end{TPlist}\par\noindent
In "slow mode" You can "pan" a small window around in an image by
clicking the
{\it left} mouse button
in the image.  The position in the image
under the cursor will jump to the center of the window.  The
{\it F9} key
or
{\it Alt} keys
reset the view to position the center of the image in the center of the
window.  Furthermore,
{\it control-F9} (or control-Alt)
saves the current view, and
{\it shift-F9} (or shift-Alt)
restores it.
\shead{NOTE}
If you have a shaded image that looks "too dark", it is probably because the
gamma is not set on the display.  (The default gamma is 1, which assumes that
gamma compensation will be done once and for all by programs producing images.)
{\it gamma} 2
is better when the image producing program does not do the gamma correction.
You may want to put a gamma command in your .login file.
\shead{SEE ALSO}
{\it getmex}{\rm (1),}
{\it urt}{\rm (1),}
{\it gversion}{\rm (3g),}
{\it gamma}{\rm (1g),}
{\it RLE}{\rm (5).}
\shead{AUTHOR}
Russ Fish, University of Utah.
Based on getX, by Spencer W. Thomas.
\newpage
% -*-LaTeX-*-
% Converted automatically from troff to LaTeX by tr2tex on Tue Aug  7 18:10:18 1990
% tr2tex was written by Kamal Al-Yahya at Stanford University
% (Kamal%Hanauma@SU-SCORE.ARPA)


%[troffman]{article}
%
%
% input file: get_orion.1
%
% Copyright (c) 1986, University of Utah
\phead{GET\_ORION}{1}{July\ 20,\ 1987}
 1
\shead{NAME}
get\_orion -- get RLE images to an Orion graphics display
\shead{SYNOPSIS}
{\bf get\_orion}
[
{\bf --D}
] [
{\bf --b}
] [
{\bf --f}
] [
{\bf --g}{\it \ gam}
] [
{\bf --l}
] [
{\bf --r}
] [ 
{\it infile}
]
\shead{DESCRIPTION}
This program displays an
{\it RLE}{\rm (5)}
file on a High Level Hardware Orion graphics display running the StarPoint 
graphics system.  It uses a
dithering technique to take a full-colour or grey scale image into
the limited number of colours available.

The default behavior is to display the image in colour using
a 216 colour map (6 intensities per primary). However, an 
{\it RLE(5)}
file with 1 colour and 3 colour map channels is treated as a
special case with the colour map in the header loaded as the graphics 
colour map
and the data used to index this map.  In this mode of operation no
dithering is done as the file is assumed to be the output of some
program which has selected the "best" possible colours for the image
and has already corrected some of the errors produced by the quantization.  
An option is provided to force a grey scale display of colour images.

{\it Get\_orion} 
uses the standard window manager creation procedure to create a window at a
particular location on the screen.  The size of the window is the size of the 
image. 

\shead{OPTIONS}
\begin{TPlist}{{\bf --D}}
\item[{{\bf --D}}]
"Debug mode".  The operations in the input
{\it RLE(5)}
file will be printed as they are read.
\item[{{\bf --b}}]
Forces
{\it getOrion}
to produce a grey scale dithered image instead of a colour image using 128 
shades of grey.  Colour input will be transformed to grey level using the 
NTSC Y transform.
\item[{{\bf --f}}]
Normally 
{\it get\_orion} 
will only use entries 0-239 of the graphics device colour map, as
the others are used by the window manager for background, icons,
etc.  This option will force it to use all 256 entries and is useful
only when the image has been specified with a 24-bit colour map
\item[{{\bf --g}{\it \ gam}
}]
Specifies, as a floating point number, the gamma correction factor to be used 
when correcting the colour map. 
\item[{{\bf --l}}]
Use a linear colour map. Identical to having a gamma of 1.
\item[{{\bf --r}}]
Use "reverse" mode for display.  The scanlines are by default displayed from 
the bottom-up, this option displays them from the top-down.  Useful
for applications which have produced the scanlines starting from the
top one.
\bf
\item[{{\it infile}}]
Name of file to display.  If none specified, the image will be read from 
standard input.
\end{TPlist}\shead{SEE ALSO}
{\it urt}{\rm (1),}
{\it RLE}{\rm (5).}
\shead{AUTHOR}
Gianpaolo Tommasi, Computer Laboratory, University of Cambridge.
The code is based on other "get" routines.
\shead{DEFICIENCIES}
The window cannot be moved whilst the image is being displayed.

Because of the way the graphics memory is organized displaying images in 
GM\_BW mode is slow.

\newpage
% -*-LaTeX-*-
% Converted automatically from troff to LaTeX by tr2tex on Tue Aug  7 18:10:18 1990
% tr2tex was written by Kamal Al-Yahya at Stanford University
% (Kamal%Hanauma@SU-SCORE.ARPA)


%[troffman]{article}
%
%
% input file: getap.1
%
% Copyright (c) 1986, University of Utah
\phead{GETAP}{1}{Feb\ 3,\ 1987}
 1
\shead{NAME}
getap -- get RLE images to an Apollo display
\shead{SYNOPSIS}
{\bf getap}
[
{\bf --b}
] [
{\bf --g} 
{\it gamma}
] [
{\bf --l}
] [
{\bf --n}
] [
{\bf --r}
] [ 
{\bf --t}{\it \ text}
] [
{\bf --w}
] [
{\bf --x}{\it \ left}
] [
{\bf --y}{\it \ top}
] [
{\it file}
]

\shead{DESCRIPTION}
This program displays an
{\it RLE}{\rm (5)}
file on an Apollo workstations running the Display Manager.
It uses a dithering technique to take a
full-color or gray scale image into the limited number of colors
typically available, unless "borrow mode" is specified.  Under
borrow mode, the 24 bit mode of the Apollo hardware is used (if it's
available).  On bitmap displays, 
{\it getap}
converts the image to black and white and uses bitmap dithering.
\shead{OPTIONS}
\begin{TPlist}{{\bf --b}}
\item[{{\bf --b}}]
This tells
{\it getap}
to use "borrow mode" instead of an apollo window, if the hardware supports
it.  The only hardware that supports this are the DN550, DN560 and the
DN660.  The bottom portion of the image is chopped off on workstations without
square screens.
\item[{{\bf --g} gamma}]
This loads a color map with the specified gamma.
\item[{{\bf --l}}]
Loads a linear color map.
\item[{{\bf --n}}]
"No border" mode.  The Apollo window will have no border or annotation drawn.
\item[{{\bf --r}}]
On black and white displays, this flips the orientation of black and white.
\item[{{\bf --t}{\it \ text}
}]
Displays the 
{\it text}
string at the bottom of the image.
\item[{{\bf --w}}]
Convert the image to grays before displaying.  On 4-bit displays, this
produces a significantly nicer looking image.
\item[{{\bf --x}{\it \ left}
}]
Specify position of the left edge of the window.
\item[{{\bf --y}{\it \ top}
}]
Specify position of the top edge of the window.
\end{TPlist}\shead{SEE ALSO}
{\it urt}{\rm (1),}
{\it RLE}{\rm (5).}
\shead{AUTHOR}
John W. Peterson
\shead{BUGS}
{\it Getap}
is pretty sloppy about dealing with the color map, particularly in window
mode.

Since Apollo workstations now support the X window system, 
{\it getap}
is mostly subsumed by 
{\it getx11}{\rm .}

\newpage
% -*-LaTeX-*-
% Converted automatically from troff to LaTeX by tr2tex on Tue Aug  7 18:10:19 1990
% tr2tex was written by Kamal Al-Yahya at Stanford University
% (Kamal%Hanauma@SU-SCORE.ARPA)


%[troffman]{article}
%
%
% input file: getbob.1
%
% Copyright (c) 1988, University of Utah
\phead{GETBOB}{1}{Jun\ 24,\ 1988}
 1
\shead{NAME}
getbob -- Display RLE files on HP Bobcat screens.
\shead{SYNOPSIS}
{\bf getbob}
[
{\bf --l}
] [
{\bf --g}
{\it gamma}
] [
{\bf --p}
{\it x} y
] [
{\bf --d}
{\it display}
] [
{\bf --x}
{\it driver}
] [
{\it infile}
]
\shead{DESCRIPTION}
{\it Getbob}
reads a file in
{\it RLE}{\rm (5)}
format and displays it on an HP bobcat screen.
It uses a dithering technique to take a
full-color or gray scale image into the limited number of colors
typically available.
\shead{OPTIONS}
\begin{TPlist}{{\bf --l}}
\item[{{\bf --l}}]
Use a linear map.
\item[{{\bf --g}{\it \ gamma}
}]
Use a gamma correction value of 
{\it gamma}{\rm .}
\item[{{\bf --d}{\it \ device}
}]
Use the device specified.  The default is /dev/graphics.
\item[{{\bf --x}{\it \ driver}
}]
Use the driver specified.  The default is hp98710.
\item[{{\bf --p}{\it \ x\ y}
}]
Position image lower left hand corner on the display.  The
x and y position is given in pixels with the origin taken as
the lower left hand corner of the display.  This flag is only
useful with the
{\bf --d}{\rm ,}
and
{\bf --x}
flags.
\item[{{\it infile}}]
This option is used to name the input file.  If not present, input is taken
from
{\it stdin.}
\end{TPlist}\shead{SEE ALSO}
{\it urt}{\rm (1),}
{\it RLE}{\rm (5).}
\shead{AUTHOR}
Mark Bloomenthal, University of Utah
\newpage
% -*-LaTeX-*-
% Converted automatically from troff to LaTeX by tr2tex on Tue Aug  7 18:10:19 1990
% tr2tex was written by Kamal Al-Yahya at Stanford University
% (Kamal%Hanauma@SU-SCORE.ARPA)


%[troffman]{article}
%
%
% input file: getcx3d.1
%
\phead{GETCX3D}{1}{June\ 24,\ 1986}
 1
\shead{NAME}
getcx3d -- display an RLE(5) image on the Chromatics
\shead{SYNOPSIS}
{\bf getcx3d}
[
{\bf --O}
] [
{\bf --B}
] [
{\bf --d}
] [
{\bf --t}
] [
{\bf --p\ x\ y}
] [
{\bf --l}
] [ 
{\it infile}
%... ]
\shead{DESCRIPTION}
This program displays an
{\it RLE}{\rm (5)}
image on a Chromatics CX 1536 (raster dimensions 1536115224)
running CX3D.

{\it Getcx3d}
will display black and white and full color
images, ignoring the alpha channel if
present.  All three background styles of the
{\it RLE}{\rm (5)}
format are supported: (0) write every pixel,
(1) do not write background pixels (overlay) and (2) clear to background;
see the
{\bf --O}
and
{\bf --B}
options.
You may position an image at some place other than (0, 0)
on the screen; see the
{\bf --p}
option.
The
{\bf --d}
and
{\bf --t}
options magnify the image; see below.  The bounding box of the image
is the only part of the image that is ever displayed (i.e. clear to
background will only clear the area within the bounding box, not the
entire screen.)  The color maps within the CX are not changed.  Colors
are passed through a gamma correction map (gamma\_value = 2.5 in
round(255 * ((x / 255) \^{} (1 / gamma\_value))), judged best for the
monitor connected to the CX) on the host before they are sent to the
CX.  Use
{\bf --l}
to pass colors through a linear map.  Finally, any color maps
specified by the RLE file are ignored.  This is a bug.
\shead{OPTIONS}
\begin{TPlist}{{\bf --O}}
\item[{{\bf --O}}]
Force overlay background style.  Ignoring the background style
indicated in the RLE file this option will overlay the RLE image,
causing the previous image on the CX to show through pixels of
background color of the present image.
\item[{{\bf --B}}]
Force clear to background style.  Ignoring the background style
indicated in the RLE file this option will clear the bounding box area
of the RLE file before displaying the image.
\item[{{\bf --d}}]
Double the image size.  Display four pixels for every one pixel of the
RLE file.
\item[{{\bf --t}}]
Triple the image size.  Display nine pixels for every one pixel of the
RLE file.
\item[{{\bf --p}{\it \ x\ y}
}]
Reposition the image.  Place the left corner of the image (0, 0) at
some place other than the left corner of the CX.  Note that the left
corner of the image is (0, 0), which may be different from the left
corner of the bounding box of the image.  The bounding box is the only
area of the image that is ever displayed.
\item[{{\bf --l}}]
Use a linear map.  By default all colors are passed through a gamma
correction map on
the host before they are sent to the CX.  This option causes no
mapping to take place.
\item[{{\it infile}}]
Name of file to display.  If not specified or if
{\it --}{\rm ,}
an RLE encoded image is read from the standard input.

Any number of images may be displayed with one invocation of
{\it getcx3d}{\rm .}
\end{TPlist}\shead{FILES}
/dev/dr0
\shead{SEE ALSO}
{\it urt}{\rm (1),}
{\it RLE}{\rm (5).}
\shead{AUTHOR}
W. Thomas McCollough, Jr., University of Utah
\shead{BUGS}
Color maps are not loaded.

If interrupted with a catchable signal,
{\it getcx3d}
will close the CX gently, allowing future access without rebooting.
If
{\it getcx3d}
is stopped, however, and then (before it is continued) killed with any
signal, then the CX may be left in a bad state.
\newpage
% -*-LaTeX-*-
% Converted automatically from troff to LaTeX by tr2tex on Tue Aug  7 18:10:20 1990
% tr2tex was written by Kamal Al-Yahya at Stanford University
% (Kamal%Hanauma@SU-SCORE.ARPA)


%[troffman]{article}
%
%
% input file: getfb.1
%
% Copyright (c) 1986, University of Utah
\phead{GETFB}{1}{Feb\ 12,\ 1987}
 1
\shead{NAME}
getfb -- display an RLE file on a BRL libfb frame buffer.
\shead{SYNOPSIS}
{\bf getfb}
[
{\bf --d}
] [ 
{\it infile}
] 
\shead{DESCRIPTION}
This program displays an
{\it RLE}{\rm (5)}
file on any frame buffer supported by the BRL 
{\it libfb}{\rm .}
The option
{\bf --d}
prints the image header information and turns on debugging of the
{\it RLE}
file.  All of the
{\it RLE}
opcodes will be printed as they are read from the input file.
If an input
{\it rlefile}
is not specified, input will be taken from standard input.
\shead{SEE ALSO}
{\it urt}{\rm (1),}
{\it RLE}{\rm (5).}
\shead{AUTHOR}
Paul Stay, Ballistic Research Laboratory.
\shead{LIMITATIONS}
Will only display images up to 1024 pixels wide.  Will not display
black and white (single channel) images correctly.  Ignores color map
in
{\it RLE}
file.
\newpage
% -*-LaTeX-*-
% Converted automatically from troff to LaTeX by tr2tex on Tue Aug  7 18:10:21 1990
% tr2tex was written by Kamal Al-Yahya at Stanford University
% (Kamal%Hanauma@SU-SCORE.ARPA)


%[troffman]{article}
%
%
% input file: getgmr.1
%
\phead{GETGMR}{1}{9/14/82}
 1
\shead{NAME}
getgmr -- Restore an RLE image to a Grinnell GMR-27 frame buffer.
\shead{SYNOPSIS}
{\bf getgmr}
[
{\bf --q}
] [
{\bf --d}
] [
{\bf --\{BO}\}
] [
{\bf --\{pP}\}
{\it x} y
] [
{\bf --c}
{\it channel}
[
{\it into}
] ] [ 
{\it infile}
]
\shead{DESCRIPTION}
Displays an
{\it RLE}{\rm (5)}
file on a Grinnell GMR-27 frame buffer.
\begin{TPlist}{{\bf --q}}
\item[{{\bf --q}}]
Query the given file.  Determine if it is an
{\it RLE}{\rm (5)}
file.  Does not affect the frame buffer.
\item[{{\bf --D}}]
Debug the given file.  Print information about each command in the input
file.  Displays as it prints.
\item[{{\bf --B}}]
If the file was saved with %
\bf --B \rm%
or %
\bf --O\rm%
, restore the
background color before restoring the image data.
\item[{{\bf --O}}]
If the file was saved with %
\bf --B \rm%
or %
\bf --O\rm%
, restore the image
data in overlay mode. Only areas of the original image which were not
the background color are restored.  The rest of the image already in
the frame buffer is undisturbed.
\item[{{\bf --p}{\it }{\bf x}{\it y}
}]
Reposition the image.  The original lower left corner is positioned at [x, y]
before restoring the image.  A warning:  A saved image should not be
repositioned so that any saved data wraps around the X borders.  If the file
was not saved with %
\bf --B \rm%
or %
\bf --O\rm%
, this includes background areas.
\item[{{\bf --P}{\it }{\bf x}{\it y}
}]
Reposition the image incrementally, that is, %
\it x \rm%
and %
\it y \rm%
are
taken as offsets from the original position of the image.
\item[{%
\bf --c \rm%
%
\it channel \rm%
[ %
\it into \rm%
]}]
Put only the given color
{\it channel}
into the frame buffer.  If
{\it into}
is specified, loads it into that channel.
If the input file is black and
white (one channel), then
{\bf --c}
{\it channel}
is equivalent to
{\bf --c}
{\it 0} channel.
\item[{{\bf infile}}]
Name of file to display.  If not specified, input is read from stdin.
\end{TPlist}\shead{SEE ALSO}
{\it RLE}{\rm (5).}
\shead{AUTHOR}
Spencer W. Thomas, 
Todd Fuqua
\shead{BUGS}
Seems to interact poorly with Grinnell hardware bugs at times.
\newpage
% -*-LaTeX-*-
% Converted automatically from troff to LaTeX by tr2tex on Tue Aug  7 18:10:21 1990
% tr2tex was written by Kamal Al-Yahya at Stanford University
% (Kamal%Hanauma@SU-SCORE.ARPA)


%[troffman]{article}
%
%
% input file: getiris.1
%
% Copyright (c) 1986, University of Utah
\phead{GETIRIS}{1}{Jan\ 20,\ 1987}
 1
\shead{NAME}
getiris -- display an RLE image on a Silicon Graphics Iris Workstation.
\shead{SYNOPSIS}
{\bf getiris}
[
{\it infile}
]
\shead{DESCRIPTION}
This program displays an
{\it RLE}{\rm (5)}
file on a
{\it Silicon} Graphics Iris
that is not running the 
window manager.  It uses the full 24 bits of color available on an iris.
After the picture is displayed, press any mouse button to erase the screen.
{\it Getiris}
does not work on the 4D series, only on the 2400 (or, I assume, 3000) series
machines.
\shead{OPTIONS}
\begin{TPlist}{{\it infile}}
\item[{{\it infile}}]
Name of the
{\it RLE}{\rm (5)}
file to display.  If not specified, the image will be read from the
standard input.
\end{TPlist}\shead{SEE ALSO}
{\it get4d}{\rm (1),}
{\it getmex}{\rm (1),}
{\it urt}{\rm (1),}
{\it RLE}{\rm (5).}
\shead{AUTHOR}
Glenn McMinn and Rod Bogart, University of Utah.
\newpage
% -*-LaTeX-*-
% Converted automatically from troff to LaTeX by tr2tex on Tue Aug  7 18:10:22 1990
% tr2tex was written by Kamal Al-Yahya at Stanford University
% (Kamal%Hanauma@SU-SCORE.ARPA)


%[troffman]{article}
%
%
% input file: getmac.1
%
% Copyright (c) 1986, University of Utah
\phead{GETMAC}{1}{Jun\ 22,\ 1988}
 1
\shead{NAME}
getmac -- Display RLE images on a MacIntosh display.
\shead{SYNOPSIS}
{\bf getmac}
[
{\bf --d}
] [ 
{\it infile}
]
\shead{DESCRIPTION}
This program displays an
{\it RLE}{\rm (5)}
file on a MacIntosh display.
It uses a dithering technique to take a
full-color or gray scale image into the limited number of colors
available.

Clicking on the close box or typing a
{\bf q}
exits the program and returns to the MPW shell.  All other event
processing is suspended until the program exits.
\shead{OPTIONS}
\begin{TPlist}{{\bf --d}}
\item[{{\bf --d}}]
Disables dithering.
\item[{{\bf infile}}]
Name of the
{\it RLE}{\rm (5)}
file to display.  If not specified, the image will be read from the
standard input.
\end{TPlist}\shead{SEE ALSO}
{\it urt}{\rm (1),}
{\it RLE}{\rm (5).}
\shead{AUTHOR}
Spencer W. Thomas, University of Utah
\par
John Peterson, Apple Computer Inc.
\shead{BUGS}
Behaves unpredictably when it runs out of memory.
If you have 2mb or less, don't run under multifinder.
\shead{DEFICIENCIES}
Ignores the color map.
\newpage
% -*-LaTeX-*-
% Converted automatically from troff to LaTeX by tr2tex on Tue Aug  7 18:10:22 1990
% tr2tex was written by Kamal Al-Yahya at Stanford University
% (Kamal%Hanauma@SU-SCORE.ARPA)


%[troffman]{article}
%
%
% input file: getmex.1
%
% Copyright (c) 1986, University of Utah
\phead{GETMEX}{1}{Jan\ 20,\ 1987}
 1
\shead{NAME}
getmex -- get RLE images to an Iris display under the window manager
\shead{SYNOPSIS}
{\bf getmex}
[
{\bf --f}
] [
{\bf --w}
] [
{\bf --D}
] [
{\bf --m} 
{\it mapstart}
] [ 
{\it infile}
]
\shead{DESCRIPTION}
This program displays an
{\it RLE}{\rm (5)}
file on a
{\it Silicon} Graphics Iris
display running the
{\it mex}
or
{\it 4Sight}
window manager.  It uses a dithering technique to take a
full-color or gray scale image into the limited number of colors
typically available under
{\it mex}{\rm .}
Its default behavior is to try to
display the image in color, an option is provided to force black and
white display.  Several
{\it getmex}
processes running simultaneously will share color map entries.

{\it getmex}
uses the standard 
window creation procedure to create a window with a location and size
specified by the user, with the restriction that the window will be no larger
than the input image.  If the window is smaller than the image, the center of
the image will be visible in the window.

You can "pan" a small window around in an image by attaching the mouse to
the window using
{\it mex}
or 
{\it 4Sight}
and clicking the
{\it left} mouse button
in the image.  The position in the image
under the cursor will jump to the center of the window.  The
{\it SETUP}
key resets the view to position the center of the image in the center of the
window.  Furthermore,
{\it control-SETUP}
saves the current view, and
{\it shift-SETUP}
restores it.
\shead{OPTIONS}
\begin{TPlist}{{\bf --f}}
\item[{{\bf --f}}]
Normally,
{\it getmex}
will fork itself after putting the image on the screen, so that the
parent process may return the shell, leaving an "invisible" child to
keep the image refreshed.  If 
{\bf --f}
is specified, getmex will remain attached to the shell, whence it may
be killed with an interrupt signal or via
the window manager.
\item[{{\bf --w}}]
This flag forces
{\it getmex}
to produce a gray scale dithered image instead of a color image.
Color input will be transformed to black and white via the
{\it NTSC} Y
transform.  Since a 128-step greyscale is used, this will produce a
much smoother looking image than color dithering.
\item[{{\bf --D}}]
"Debug mode".  The operations in the input
{\it RLE}{\rm (5)}
file will be printed as they are read.
\item[{{\bf --m}{\it \ mapstart}
}]
Specifies the starting location of the block of color map to be used by
{\it getmex} .
(There are 1024 colors available on the Iris 2400/3000s under
{\it mex}{\rm .)}
The default for color images is a block of 512 rgb colors starting at
location 512 in the color map.  Black-and-white images default to a block
of 128 grey shades starting at location 128.  Both the rgb and grey ramps
are gamma-corrected in the same way as the
{\it makemap}
program in the /usr/gifts/mextools/tools directory.  You probably want
to set up your initial color map using
{\it makemap} .
\item[{{\it infile}}]
Name of the
{\it RLE}{\rm (5)}
file to display.  If not specified, the image will be read from the
standard input.
\end{TPlist}\shead{SEE ALSO}
{\it get4d}{\rm (1),}
{\it getiris}{\rm (1),}
{\it urt}{\rm (1),}
{\it RLE}{\rm (5).}
\shead{AUTHOR}
Russ Fish, University of Utah.
\newpage
% -*-LaTeX-*-
% Converted automatically from troff to LaTeX by tr2tex on Tue Aug  7 18:10:23 1990
% tr2tex was written by Kamal Al-Yahya at Stanford University
% (Kamal%Hanauma@SU-SCORE.ARPA)


%[troffman]{article}
%
%
% input file: getqcr.1
%
% Copyright (c) 1988, University of Utah
\phead{GETQCR}{1}{Jan\ 25,\ 1988}
 1
\shead{NAME}
getqcr -- Photograph an RLE image with the Matrix QCR-Z camera
\shead{SYNOPSIS}
{\bf getqcr}
[
{\bf --v}
] [
{\bf --c}
] [
{\bf --d}
] [
{\bf --f}
] [
{\bf --p}
{\it xpos} ypos
] [
{\bf --e}
{\it exposures}
]
{\it infile}
\shead{DESCRIPTION}
{\it Getqcr}
photographs an image on the Matrix QCR-Z camera.  The program
reads the image once for each channel, and displays this on QCR-Z,
moving the filter wheel as appropriate.  The colormap is currently
ignored, so one must be applied first if needed (see applymap(1)).
Since the QCR supports large images (2K or 4K pixels in size), 
most images will need to be stretched to fill the QCR-Z's image area.
Both 
{\it fant}{\rm (1)}
and
{\it rlezoom}{\rm (1)}
perform this function.

The current support library assumes the QCR-Z is connected to an HP
Series 300 machine, the library may need modifications for other HPIB
interfaces.
\shead{OPTIONS}
\begin{TPlist}{{\bf --v}}
\item[{{\bf --v}}]
This enables verbose output.  Since exposing large images takes several
minutes, this is generally useful to monitor progress.
\item[{{\bf --e}{\it \ exposures}
}]
Expose the film 
{\it exposures}
number of times.  This is much faster than running getqcr multiple times.
\item[{{\bf --d}}]
Double expose (same as "%
\bf --e 2\rm%
").
\item[{{\bf --f}}]
Select high resolution (4K) mode.  Default is low resolution (2K).
\item[{{\bf --c}}]
Center the image.  This ignores the position values in RLE header,
and centers the image in the middle of the QCR-Z's camera field.
The proper resolution (2K or 4K) is automatically selected depending
on the image size (%
\bf -f \rm%
is ignored if %
\bf -c \rm%
is specified).
\item[{{\bf --p}{\it \ xpos\ ypos}
}]
Position the image at a specific point.  Note getqcr uses the RLE
coordinate system (origin at the bottom left) instead of the QCR-Z
coordinate system.
\end{TPlist}\shead{SEE ALSO}
\raggedright
{\it applymap}{\rm (1),}
{\it rlezoom}{\rm (1),}
{\it fant}{\rm (1),}
{\it rleflip}{\rm (1),}
{\it urt}{\rm (1),}
{\it RLE}{\rm (5).}
%.ad b
\shead{AUTHOR}
John W. Peterson
\shead{BUGS}
The color map should be applied automatically.

Currently uses "row" mode, it may run faster in "raw" mode.

Single channel images should be photographed in black and white (they 
currently come out red).

It was written for the 4x5 film back.  Shutter and film advance controls
for the 35mm and Oxberry backs are not implemented.
\newpage
% -*-LaTeX-*-
% Converted automatically from troff to LaTeX by tr2tex on Tue Aug  7 18:10:23 1990
% tr2tex was written by Kamal Al-Yahya at Stanford University
% (Kamal%Hanauma@SU-SCORE.ARPA)


%[troffman]{article}
%
%
% input file: getren.1
%
% Copyright (c) 1986, University of Utah
\phead{GETREN}{1}{Nov\ 1,\ 1987}
 1
\shead{NAME}
getren -- get RLE images to an HP98721 ("Renaissance") display
\shead{SYNOPSIS}
{\bf getren}
[
{\bf --p} 
{\it xpos} ypos
] [
{\bf --O}
] [
{\bf --P}
{\it xoff} yoff
] [
{\bf --d}
{\it display}
] [
{\bf --x}
{\it driver}
] [ 
{\it infile}
]
\shead{DESCRIPTION}
This program displays an
{\it RLE}{\rm (5)}
file on an HP 98721 "Renaissance" display configured with at
least 24 bits per pixel.  If a color map exists in the file,
it is loaded into the display, otherwise a linear map is used.
\shead{OPTIONS}
\begin{TPlist}{{\bf --p}{\it \ xpos\ ypos}
}
\item[{{\bf --p}{\it \ xpos\ ypos}
}]
position the image at 
{\it xpos,} ypos.
\item[{{\bf --P}{\it \ xoff\ yoff}
}]
Offset the image position by
{\it xoff} yoff
\item[{{\bf --O}}]
Don't clear the screen (overlay mode)
\item[{{\bf --d}{\it \ display}
}]
Gives the name of the display device to which the image is to be
displayed.  The default is "/dev/hp98721".
\item[{{\bf --x}{\it \ driver}
}]
Gives the name of the device driver to be used to communicate with
the display device.  The default is "hp98721".
\item[{{\it infile}}]
The input will be read from this file.  If
{\it infile}
is "--" or is not specified, the input will be read from the standard
input stream.
\end{TPlist}\shead{SEE ALSO}
{\it read98721}{\rm (1),}
{\it urt}{\rm (1),}
{\it RLE}{\rm (5).}
\shead{AUTHOR}
John W. Peterson, University of Utah, with input from Filippo Tampieri of
Cornell and Eric Haines of 3D/Eye.
\shead{BUGS}
The program assumes a full 24 bit Renaissance display.  The HP graphics
library supports automatically dithering for displays with fewer bitplanes,
but getren ignores this.
\nwl
The device and driver names are compiled in as "/dev/hp98721" and "hp98721",
respectively.  This may need changing on systems configured differently
(in particular, systems with the Renaissance as their sole display
may use a different name for the device).
\newpage
% -*-LaTeX-*-
% Converted automatically from troff to LaTeX by tr2tex on Tue Aug  7 18:10:24 1990
% tr2tex was written by Kamal Al-Yahya at Stanford University
% (Kamal%Hanauma@SU-SCORE.ARPA)


%[troffman]{article}
%
%
% input file: getsun.1
%
\phead{GETSUN}{1}{October\ 6,\ 1987}
 1
\shead{NAME}
getsun -- get RLE images to a sun window 
\shead{SYNOPSIS}
{\bf getsun}
[
{\bf --\{wW}\}
] [
{\bf --D}
] [
{\bf --l} 
{\it levels}
] [
{\bf --\{iI}\}
{\it image\_gamma}
] [
{\bf --g}
{\it display\_gamma}
] [ 
{\it file}
]
\shead{DESCRIPTION}
This program displays an
{\it RLE}{\rm (5)}
file in a sun window
display.  It uses a dithering technique to take a
full-color or gray scale image into the limited number of colors
available under sun windows.
Its default behavior is to try to
display the image in color with as many brightness levels as possible
(except on a one bit deep display), options are provided to limit the
number of levels or to force black and white display.  Several
{\it getsun}
processes running simultaneously with the same color resolution will
share color map entries.

Other options allow control over the gamma, or contrast, of the image.
The dithering process assumes that the incoming image has a gamma of
1.0 (i.e., a 200 in the input represents an intensity twice that of
a 100.)  If this is not the case, the input values must be adjusted
before dithering via the
{\bf --i}
or 
{\bf --I}
option.  The input file may also specify the gamma of the image via a
picture comment (see below).  The output display is assumed to have a gamma of
2.5 (standard for color TV monitors).  This may be modified via the
{\bf --g}
option if a display with a different gamma is used.

{\it Getsun}
creates a sun window
the size of the image being displayed.  The header of the new window
displays the name of the image being displayed and its size.
\shead{OPTIONS}
\begin{TPlist}{{\bf --w}}
\item[{{\bf --w}}]
This flag forces
{\it getsun}
to produce a gray scale dithered image instead of a color image.
Color input will be transformed to black and white via the
{\it NTSC} Y
transform.  On a low color resolution display (a display with only 4
bits, for example), this will produce a much smoother looking image
than color dithering.  It may be used in conjunction with
{\bf --l}
to produce an image with a specified number of gray levels.
\item[{{\bf --W}}]
This flag forces
{\it getsun}
to display the image as a black and white bitmap image.  This is the
only mode available on monochrome (non gray scale) displays (and is
the default there).  Black pixels will be displayed with pixel value 0
and white with pixel value 1. 
\item[{{\bf --D}}]
"Debug mode".  The operations in the input
{\it RLE}{\rm (5)}
file will be printed as they are read.
\item[{{\bf --l}{\it \ levels}
}]
Specify the number of gray or color levels to be used in the dithering
process. The default is 5 except on monochrome (non gray scale) displays.
Levels must be in the range 2,6.
\item[{{\bf --i}{\it \ image\_gamma}
}]
Specify the gamma (contrast) of the image.  A low contrast image,
suited for direct display without compensation on a high contrast
monitor (as most monitors are) will have a gamma of less than one.
The default image gamma is 1.0.  Image gamma may also be specified by
a picture comment in the
{\it RLE} (5)
file of the form
{\bf image\_gamma=}{\it gamma.}
The command line argument will override the value in the file if specified.
\item[{{\bf --I}{\it \ image\_gamma}
}]
An alternate method of specifying the image gamma, the number
following
{\bf --I}
is the gamma of the display for which the image was originally
computed (and is therefore 1.0 divided by the actual gamma of the
image).  Image display gamma may also be specified by
a picture comment in the
{\it RLE} (5)
file of the form
{\bf display\_gamma=}{\it gamma.}
The command line argument will override the value in the file if specified.
\item[{{\bf --g}{\it \ display\_gamma}
}]
Specify the gamma of the 
{\it sun}
display monitor.  The default value is 2.5, suitable for most color TV
monitors (this is the gamma value assumed by the NTSC video standard).
\item[{{\it infile}}]
Name of the
{\it RLE}{\rm (5)}
file to display.  If not specified, the image will be read from the
standard input.
\end{TPlist}\shead{SEE ALSO}
{\it getx11}{\rm (1),}
{\it urt}{\rm (1),}
{\it RLE}{\rm (5).}
\shead{AUTHOR}
Philip J. Klimbal, RIACS
\shead{BUGS}
Single channel input files with color map should be displayed as such
by loading the colormap directly,
instead of mapping the input to 24 bits and then dithering back to 8.
\newpage
% -*-LaTeX-*-
% Converted automatically from troff to LaTeX by tr2tex on Tue Aug  7 18:10:25 1990
% tr2tex was written by Kamal Al-Yahya at Stanford University
% (Kamal%Hanauma@SU-SCORE.ARPA)


%[troffman]{article}
%
%
% input file: gettaac.1
%
% Copyright (c) 1990, Southwest Research Institute
% All rights reserved.
%
% Redistribution and use in source and binary forms are permitted
% provided that the above copyright notice and this paragraph are
% duplicated in all such forms and that any documentation,
% advertising materials, and other materials related to such
% distribution and use acknowledge that the software was developed
% by Southwest Research Institute.  The name of Southwest Research
% Institute may not be used to endorse or promote products derived
% from this software without specific prior written permission.
% THIS SOFTWARE IS PROVIDED ``AS IS'' AND WITHOUT ANY EXPRESS OR
% IMPLIED WARRANTIES, INCLUDING, WITHOUT LIMITATION, THE IMPLIED
% WARRANTIES OF MERCHANTIBILITY AND FITNESS FOR A PARTICULAR PURPOSE.
%
\phead{GETTAAC}{1}{July\ 5,\ 1990}
 1
\shead{NAME}
gettaac -- display an RLE image on a Sun TAAC-1.
\shead{SYNOPSIS}
{\bf gettaac}

\shead{DESCRIPTION}
This program displays an
{\it RLE}{\rm (5)}
file on a
{\it Sun} TAAC-1
that is running in single monitor mode under Sunview.
It uses the full 24 bits of color available on the TAAC.
The RLE file is either read from the standard input or
from the file name entered in the control panel.  If the
file name is entered from the control panel,
{\it csh}{\rm (1)}
style tilde expansion and
file name completion are supported.  The control
panel allows the gamma to be set and the file to be
loaded as either a gray scale or rgb image.
\shead{SEE ALSO}
{\it urt}{\rm (1),}
{\it RLE}{\rm (5).}
\shead{AUTHOR}
Keith S. Pickens, Southwest Research Institute.


\newpage
% -*-LaTeX-*-
% Converted automatically from troff to LaTeX by tr2tex on Tue Aug  7 18:10:25 1990
% tr2tex was written by Kamal Al-Yahya at Stanford University
% (Kamal%Hanauma@SU-SCORE.ARPA)


%[troffman]{article}
%
%
% input file: getx10.1
%
% Copyright (c) 1986, University of Utah
\phead{GETX10}{1}{Jan\ 20,\ 1987}
 1
\shead{NAME}
getx10 -- get RLE images to an X display
\shead{SYNOPSIS}
{\bf getx10}
[
{\bf --\{bB}\}
] [
{\bf --\{cwW}\}
] [
{\bf --D}
] [
{\bf --f}
] [
{\bf --m}
] [
{\bf --p}
] [
{\bf --z}
] [
{\bf --=}
{\it window\_geometry}
] [
{\bf --d}
{\it display}
] [
{\bf --\{iI}\}
{\it image\_gamma}
] [
{\bf --g}
{\it display\_gamma}
] [
{\bf --l} 
{\it levels}
] [ 
{\it infile}
]

\shead{DESCRIPTION}
This program displays an
{\it RLE}{\rm (5)}
file on an 
{\it X} Version 10
display.  It uses a dithering technique to take a
full-color or gray scale image into the limited number of colors
typically available under
{\it X}{\rm .}
Its default behavior is to try to
display the image in color with as many brightness levels as possible
(except on a one bit deep display), options are provided to limit the
number of levels or to force black and white display.  Several
{\it getx10}
processes running simultaneously with the same color resolution will
share color map entries.

Other options allow control over the gamma, or contrast, of the image.
The dithering process assumes that the incoming image has a gamma of
1.0 (i.e., a 200 in the input represents an intensity twice that of
a 100.)  If this is not the case, the input values must be adjusted
before dithering via the
{\bf --i}
or 
{\bf --I}
option.  The input file may also specify the gamma of the image via a
picture comment (see below).  The output display is assumed to have a gamma of
2.5 (standard for color TV monitors).  This may be modified via the
{\bf --g}
option if a display with a different gamma is used.

{\it Getx10}
uses the standard 
{\it X}
window creation procedure to create a window with a location and size
specified by the user, with the restriction that the window must be at
least as large as the input image.  If the window is turned into an
icon, a smaller version of the image will be displayed in the icon.  A
shifted mouse click on either the window or icon will cause the image
to be removed.
\shead{OPTIONS}
\begin{TPlist}{{\bf --b}}
\item[{{\bf --b}}]
After displaying the image in a window, 
{\it getx10}
will attempt to set your "root" window background tiling pattern to
the image.  There are some strict limitations on image size for this
to work (at least in X10).  A color or gray-scale image must be
smaller than 256x256, and a black and white 
{\rm (}{\bf --W}{\rm )}
image smaller than
about 720x720.  If the image is larger than this, a strip from the top
of the image will be displayed in the background.  Note that if you
kill the 
{\it getx10}
window, the color map entries will not be protected; any other program
that asks for a color map entry will likely get one that is being used
by the background image.
\item[{{\bf --B}}]
Loads the image into the background as above, but does not display it
in a window.
{\it Getx10}
exits after loading the background, leaving the color map unprotected,
as above.
\item[{{\bf --c}}]
This flag suppresses all dithering, and causes
{\it getx10}
to load the color map in the image file directly into the display.
Channel 0 of the image will be treated as a set of indices into the
color map.  If there are not enough color map entries available in the
display, as many as fit will be loaded and all other "colors" will be
mapped to black.  The picture comment 
{\bf color\_map\_length=}{\it maplen}
can be used to specify the exact number of relevant color map entries.
\item[{{\bf --D}}]
"Debug mode".  The operations in the input
{\it RLE}{\rm (5)}
file will be printed as they are read.
\item[{{\bf --f}}]
Normally,
{\it getx10}
will fork itself after putting the image on the screen, so that the
parent process may return the shell, leaving an "invisible" child to
keep the image refreshed.  If 
{\bf --f}
is specified, 
{\it getx10}
will not exit to the shell until the image is removed.
\item[{{\bf --m}}]
Just loads the color map.  This may be suitable for fixing up the
color map used by the root background.
\item[{{\bf --p}}]
{\it Getx10}
tries to copy the image to an off-screen pixmap for quick refresh.  On
some displays, this will fail if no off-screen memory is available.
The image will disappear shortly after it is completed when this
happens.  You should specify 
{\it --p}
to prevent the attempt to use a pixmap.
\item[{{\bf --w}}]
This flag forces
{\it getx10}
to produce a gray scale dithered image instead of a color image.
Color input will be transformed to black and white via the
{\it NTSC} Y
transform.  On a low color resolution display (a display with only 4
bits, for example), this will produce a much smoother looking image
than color dithering.  It may be used in conjunction with
{\bf --l}
to produce an image with a specified number of gray levels.
\item[{{\bf --W}}]
This flag forces
{\it getx10}
to display the image as a black and white bitmap image.  This is the
only mode available on monochrome (non gray scale) displays (and is
the default there).  Black pixels will be displayed with pixel value 0
and white with pixel value 1 (note that these may not be black and
white on certain displays, or if they have been modified with 
{\it xset}{\rm .)}
\item[{{\bf --z}}]
This flag creates a zoom window for the image.  The new window is created by 
the standard 
{\it X}
window creation process.  The mouse can be used in the image window to select
the area to zoom.  Pressing any button will reset the center of the zoom
window to be the selected pixel.  A clickdrag in the image window will resize
the zoom window to enclose the selected region.  Pressing the left button in
the zoom window will decrease the zoom factor, but will keep the same number 
of pixels zoomed.  The right button increases the zoom factor.  If the middle
button is pressed in the zoom window, position information will be printed
for the selected zoom pixel.  Note that the info will be printed only if
{\bf --f}
is given with the
{\bf --z}
option.  One may also resize the zoom window to change the number of pixels
that are zoomed.
\item[{{\bf --d}{\it \ display}
}]
Give the name of the 
{\it X}
display to display the image on.  Defaults to the value of the
environment variable
{\it DISPLAY}{\rm .}
\item[{{\bf --=}{\it \ window\_geometry}
}]
Specify the geometry of the window in which the image will be
displayed.  This is useful mostly for giving the location of the
window, as the size of the window will be at least as large as the
size of the image.  The
{\it window\_geometry}
specification need not begin with an "=" sign.
\item[{{\bf --i}{\it \ image\_gamma}
}]
Specify the gamma (contrast) of the image.  A low contrast image,
suited for direct display without compensation on a high contrast
monitor (as most monitors are) will have a gamma of less than one.
The default image gamma is 1.0.  Image gamma may also be specified by
a picture comment in the
{\it RLE} (5)
file of the form
{\bf image\_gamma=}{\it gamma.}
The command line argument will override the value in the file if specified.
\item[{{\bf --I}{\it \ image\_gamma}
}]
An alternate method of specifying the image gamma, the number
following
{\bf --I}
is the gamma of the display for which the image was originally
computed (and is therefore 1.0 divided by the actual gamma of the
image).  Image display gamma may also be specified by
a picture comment in the
{\it RLE} (5)
file of the form
{\bf display\_gamma=}{\it gamma.}
The command line argument will override the value in the file if specified.
\item[{{\bf --g}{\it \ display\_gamma}
}]
Specify the gamma of the 
{\it X}
display monitor.  The default value is 2.5, suitable for most color TV
monitors (this is the gamma value assumed by the NTSC video standard).
\item[{{\bf --l}{\it \ levels}
}]
Specify the number of gray or color levels to be used in the dithering
process.  If not this many levels are available,
{\it getx10}
will try successively fewer levels until it is able to allocate enough
color map entries.
\item[{{\it infile}}]
Name of the
{\it RLE}{\rm (5)}
file to display.  If not specified, the image will be read from the
standard input.
\end{TPlist}\shead{SEE ALSO}
{\it urt}{\rm (1),}
{\it RLE}{\rm (5).}
\shead{AUTHOR}
Spencer W. Thomas, University of Utah
\shead{BUGS}
It gets an X error when displaying an image only one line high.
\shead{DEFICIENCIES}
It totally ignores the 
{\it .Xdefaults} 
file.

\newpage
% -*-LaTeX-*-
% Converted automatically from troff to LaTeX by tr2tex on Tue Aug  7 18:10:31 1990
% tr2tex was written by Kamal Al-Yahya at Stanford University
% (Kamal%Hanauma@SU-SCORE.ARPA)


%[troffman]{article}
%
%
% input file: giftorle.1
%
% Copyright (c) 1989, David Koblas
\phead{GIFTORLE}{1}{}

\shead{NAME}
giftorle -- Convert GIF images to RLE format
\shead{SYNOPSIS}
{\bf giftorle}
[
{\bf --c}
]
[
{\bf --o}
{\it outfile.rle}
]
[
{\it infile.gif} ...
]
\shead{DESCRIPTION}
{\it Giftorle}
converts a file from Graphics Interchange Format (GIF) format into RLE format.
Images converted with 
{\it giftorle}
will need to be flipped with
{\it rleflip}{\rm \ --v}
for correct presentation.  Multiple input images may be converted,
these will be written sequentially to the output RLE file.
\shead{OPTIONS}
\begin{TPlist}{{\bf --c}}
\item[{{\bf --c}}]
Preserve the colormap that the GIF image contains, otherwise the
colormap is applied to input image.
\item[{{\bf --o}{\it \ outfile.rle}
}]
If specified, the output will be written to this file.  If 
{\it outfile.rle}
is "--", or if it is not specified, the output will be written to the
standard output stream.
\item[{{\it infile.gif} ...}]
The input will be read from these files.  If
{\it infile.gif}
is "--" or is not specified, the input will be read from the standard
input stream.
\end{TPlist}\shead{MISC}
GIF and Graphics Interchange Format are both trademarks of CompuServe
Incorporated.
\shead{SEE ALSO}
{\it rletogif}{\rm (1),}
{\it rleflip}{\rm (1),}
{\it urt}{\rm (1),}
{\it RLE}{\rm (5).}
\shead{AUTHOR}
David Koblas (koblas@mips.com or koblas@cs.uoregon.edu)

\shead{BUGS}
Should probably flip the image itself (or at least have an option).
\newpage
% -*-LaTeX-*-
% Converted automatically from troff to LaTeX by tr2tex on Tue Aug  7 18:10:31 1990
% tr2tex was written by Kamal Al-Yahya at Stanford University
% (Kamal%Hanauma@SU-SCORE.ARPA)


%[troffman]{article}
%
%
% input file: graytorle.1
%
% Copyright (c) 1988, University of Utah
\phead{GRAYTORLE}{1}{Jun\ 24,\ 1988}
 1
\shead{NAME}
graytorle -- Merges gray scale images into an RLE format file.
\shead{SYNOPSIS}
{\bf graytorle} 
[
{\bf --a}
] [
{\bf --h}
{\it hdrsize}
] [
{\bf --o}
{\it outfile}
] 
{\it xsize} ysize
{\it infiles}
\shead{DESCRIPTION}
{\it Graytorle}
reads a list of 8-bit gray scale images in unencoded binary format and converts
them to an
{\it RLE}{\rm (5)}
image with the number of channels corresponding to the number of input files.
A command line option allows specifying one of the files as an alpha channel.
\shead{OPTIONS}
\begin{TPlist}{{\bf --a}}
\item[{{\bf --a}}]
Designates the first file in the input list as being information for the
alpha channel of the image.
\item[{{\bf --h}{\it \ hdrsize}
}]
Often gray scale image files include some sort of header information.  This
option allows specification of a count of bytes to discard at the beginning
of each input file.
\item[{{\it xsize}}]
Specifies the horizontal resolution of the input files.
\item[{{\it ysize}}]
Specifies the vertical resolution of the input files.
\item[{{\bf --o}{\it \ outfile}
}]
This option is used to name the output file.  Otherwise, output is written
to
{\it stdout.}
\item[{{\it infiles}}]
List of input files.
\end{TPlist}\shead{SEE ALSO}
{\it rletogray}{\rm (1),}
{\it urt}{\rm (1),}
{\it RLE}{\rm (5).}
\shead{AUTHOR}
Michael J. Banks, University of Utah.


\newpage
% -*-LaTeX-*-
% Converted automatically from troff to LaTeX by tr2tex on Tue Aug  7 18:10:32 1990
% tr2tex was written by Kamal Al-Yahya at Stanford University
% (Kamal%Hanauma@SU-SCORE.ARPA)


%[troffman]{article}
%
%
% input file: into.1
%
\phead{INTO}{1}{Utah\ 12/17/84}

\shead{NAME}
into -- copy into a file without destroying it
\shead{SYNOPSIS}
{\bf into}
[
{\bf --f}
] 
{\it outfile}
\shead{DESCRIPTION}
{\it Into}
copies its standard input into the specified
{\it outfile,}
but doesn't actually modify the file
until it gets EOF.
This is useful in a pipeline for putting stuff back in the "same place."
The
{\it outfile}
is not overwritten if that would make it zero length, unless
the
{\bf --f}
option is given.  That option also
forces overwriting of the
{\it outfile}
even if it is not directly writable (as long as the directory is writable).
\shead{SEE ALSO}
pipe(2)
\shead{BUGS}
For efficiency reasons, the directory containing the
{\it outfile}
must be writable by the invoker.
\nwl
The original
{\it outfile's}
owner is not preserved.
\newpage
% -*-LaTeX-*-
% Converted automatically from troff to LaTeX by tr2tex on Tue Aug  7 18:10:32 1990
% tr2tex was written by Kamal Al-Yahya at Stanford University
% (Kamal%Hanauma@SU-SCORE.ARPA)


%[troffman]{article}
%
%
% input file: mcut.1
%
% Copyright (c) 1986, University of Utah
\phead{MCUT}{1}{Nov\ 8,\ 1987}
 1
\shead{NAME}
mcut -- Quantize colors in an image using the median cut algorithm
\shead{SYNOPSIS}
{\bf mcut}
[
{\bf --n}
{\it colors}
]
[
{\bf --d}
] [
{\bf --o}
{\it outfile}
]
{\it infile}
\shead{DESCRIPTION}
{\it Mcut}
reads an RLE file and tries to choose the "best" subset of colors to
represent the colors present in the original image.  A common use for this
is to display a 24 bit image on a frame buffer with only eight bits per
pixel using a 24 bit color map.  
{\it Mcut}
first quantizes intensity values
from eight bits to five bits, and then chooses the colors from this space.

{\it Mcut}
runs in two passes; the first pass scans the image to find the color
distributions, and the second pass maps the original colors into color
map indices.  The output file has a color map containing the colors
%
\it mcut \rm%
has chosen.  %
\it Mcut \rm%
also sets the picture comment
"color\_map\_length" equal to the number of colors it has chosen.  The
{\it getx11}
program (among others) will use this color map instead of dithering.
\shead{OPTIONS}
\begin{TPlist}{{\bf --n}{\it \ ncolors}
}
\item[{{\bf --n}{\it \ ncolors}
}]
Limit the number of colors chosen to ncolors.  The default is 200.
\item[{{\bf --d}}]
Uses Floyd/Steinberg dither to hide contouring.  Greatly improves images
that have a small number of colors.
\item[{{\it infile}}]
The input will be read from this file.  If it is a multi-image file,
each image will be quantized to its own colormap.
Piped input is not allowed.  
\item[{{\bf --o}{\it \ outfile}
}]
If specified, output will be written to this file, otherwise it will
go to stdout.
\end{TPlist}\shead{SEE ALSO}
{\it getx11}{\rm (1),}
{\it rlequant}{\rm (1),}
{\it urt}{\rm (1),}
{\it RLE}{\rm (5),}
\nwl
"Color Image Quantization for Frame Buffer Display", by Paul Heckbert,
Procedings of SIGGRAPH '82, July 1982, p. 297.
\shead{AUTHOR}
Robert Mecklenburg, John W. Peterson, University of Utah.
\shead{BUGS}
The initial quantization is hardwired to five bits.  This should be an
option.

\newpage
% -*-LaTeX-*-
% Converted automatically from troff to LaTeX by tr2tex on Tue Aug  7 18:10:33 1990
% tr2tex was written by Kamal Al-Yahya at Stanford University
% (Kamal%Hanauma@SU-SCORE.ARPA)


%[troffman]{article}
%
%
% input file: mergechan.1
%
% Copyright (c) 1986, University of Utah
\phead{MERGECHAN}{1}{Nov\ 8,\ 1987}
 1
\shead{NAME}
mergechan -- merge channels from several RLE files into a single output stream
\shead{SYNOPSIS}
{\bf mergechan}
[
{\bf --a}
] [
{\bf --o}
{\it outfile}
] 
{\it infiles} ...
\shead{DESCRIPTION}
{\it Mergechan}
takes input from several RLE files and combines them into a single output
stream.  Each channel in the output stream comes from the respective filename
specified on the input (i.e., channel zero is taken from the first file,
channel one from the next, etc).  The same file can be specified more than
once.  If the 
{\bf --a}
flag is given, the channels are numbered from -1 (the alpha channel) instead
of zero.  All of the input channels must have exactly the same dimensions
(use
{\it crop}{\rm (1)}
to adjust files to fit each other).

Mergechan is typically used to introduce an alpha mask from another source
into an image, or combine color channels digitized independently.

If %
\bf --o \rm%
is specified, the output will be written to
{\it outfile}{\rm .}
\shead{SEE ALSO}
{\it crop}{\rm (1),}
{\it rleswap}{\rm (1),}
{\it urt}{\rm (1),}
{\it RLE}{\rm (5).}
\shead{AUTHOR}
John W. Peterson, University of Utah.
\shead{BUGS}
Mergechan is totally ignorant of the color maps of the input files.

The restriction that all input files must be the same size could probably
be removed.
\newpage
% -*-LaTeX-*-
% Converted automatically from troff to LaTeX by tr2tex on Tue Aug  7 18:10:33 1990
% tr2tex was written by Kamal Al-Yahya at Stanford University
% (Kamal%Hanauma@SU-SCORE.ARPA)


%[troffman]{article}
%
%
% input file: painttorle.1
%
% Copyright (c) 1986, University of Utah
\phead{PAINTTORLE}{1}{December\ 20,\ 1986}
 1
\shead{NAME}
painttorle -- Convert MacPaint images to RLE format.
\shead{SYNOPSIS}
{\bf painttorle}
[
{\bf --c}
[
{\it red}
] [
{\it green}
] [
{\it blue}
] [
{\it alpha}
] ] [
{\bf --r}
] [
{\bf --o}
{\it outfile.rle}
] [
{\it infile.paint}
]
\shead{DESCRIPTION}
{\it Painttorle}
converts a file from MacPaint format into RLE format.  Because MacPaint and
RLE disagree on which end is up, the output should be sent through
{\it rleflip}
to preserve orientation. 
\shead{OPTIONS}
\begin{TPlist}{{\bf --c}{\it [red]\ [green]\ [blue]\ [alpha]}
}
\item[{{\bf --c}{\it [red]\ [green]\ [blue]\ [alpha]}
}]
Allows the color values to be specified (the default is 255).
\item[{{\bf --r}}]
Invert the color of the MacPaint pixels (reverse video).
\item[{{\it infile.paint}}]
The input paint data will be read from this file, otherwise, input will
be taken from stdin.
\item[{{\bf --o}{\it \ outfile.rle}
}]
If specified, output will be written to this file, otherwise it will
go to stdout.
\end{TPlist}\shead{SEE ALSO}
{\it rletopaint}{\rm (1),}
{\it urt}{\rm (1),}
{\it RLE}{\rm (5).}
\shead{AUTHOR}
John W. Peterson
\newpage
% -*-LaTeX-*-
% Converted automatically from troff to LaTeX by tr2tex on Tue Aug  7 18:10:34 1990
% tr2tex was written by Kamal Al-Yahya at Stanford University
% (Kamal%Hanauma@SU-SCORE.ARPA)


%[troffman]{article}
%
%
% input file: pgmtorle.1
%
% Copyright (c) 1990, Minnesota Supercomputer Center, Inc.
\phead{PGMTORLE}{1}{July\ 20,\ 1990}
 1
\shead{NAME}
pgmtorle -- convert a pbmplus/pgm image file into an RLE image file.
\shead{SYNOPSIS}
{\bf pgmtorle}
[
{\bf --h}
] [
{\bf --v}
] [
{\bf --a}
] [
{\bf --o}{\it \ outfile}
] [
{\it filename}
]
\shead{DESCRIPTION}
This program converts PBMPLUS grayscale (pgm) image files into Utah
{\it RLE}{\rm (5)}
image files.  PBMPLUS/pgm image files contain the image dimensions and 8-bit
pixels with no matte or alpha data.  When converting to an RLE file, the alpha
channel may optionally be computed.  The RLE file will contain a "grayscale"
image (8 bits) with no colormap.  The origins of PBMPLUS and Utah RLE files
are in the upper left and lower left corners respectively.  Therefore, it is
common practice to send the output of this program through the "rleflip"
utility -- see examples below.  These RLE files may then be viewed using any
RLE image viewer.
\par
\shead{OPTIONS}
\begin{TPlist}{{\bf --v}}
\item[{{\bf --v}}]
This option will cause pgmtorle to operate in verbose mode.  The header
information is written to "stderr".  Actually, there is not much header
information stored in a PBMPLUS file so this information is minimal.
\item[{{\bf --h}}]
This option allows the header of the PBMPLUS image to be dumped to "stderr"
without converting the file.  It is equivalent to using the --v option except
that no file conversion takes place.
\item[{{\bf --a}}]
This option will cause pgmtorle to use the grayscale data to compute an alpha
channel in the resulting RLE file.  For any non-zero grayscale data, the alpha
channel will contain a value of 255.  The resulting RLE image file will
contain one color channel and one alpha channel.
\item[{{\bf --o}{\it \ outfile}
}]
If specified, the output will be written to this file.  If 
{\it outfile}
is "--", or if it is not specified, the output will be written to the
standard output stream.
\item[{{\it infile.pgm}}]
The name of the PBMPLUS image data file to be converted.  This file must end
in ".pgm".  However, it is not necessary to supply the ".pgm" extension as it
will be added to the supplied name if not already there.
\end{TPlist}\shead{EXAMPLES}
\begin{TPlist}{pgmtorle --v test.pgm --o test.rle}
\item[{pgmtorle --v test.pgm --o test.rle}]
While running in verbose mode, convert test.pgm to RLE format and store
resulting data in test.rle.
\item[{pgmtorle test $|$ rleflip --v $>$test.rle}]
Convert test.pgm to RLE format and flip its contents so that it may be
correctly displayed.
\item[{pgmtorle --h test}]
Dump the header information of the PBMPLUS file called test.pgm.
\end{TPlist}\shead{SEE ALSO}
{\it ppmtorle}{\rm (1),}
{\it rletoppm}{\rm (1),}
{\it urt}{\rm (1),}
{\it RLE}{\rm (5).}
\shead{AUTHOR}
\nwl
Wesley C. Barris
\nwl
Army High Performance Computing Research Center (AHPCRC)
\nwl
Minnesota Supercomputer Center, Inc.
\newpage
% -*-LaTeX-*-
% Converted automatically from troff to LaTeX by tr2tex on Tue Aug  7 18:10:34 1990
% tr2tex was written by Kamal Al-Yahya at Stanford University
% (Kamal%Hanauma@SU-SCORE.ARPA)


%[troffman]{article}
%
%
% input file: ppmtorle.1
%
% Copyright (c) 1990, Minnesota Supercomputer Center, Inc.
\phead{PPMTORLE}{1}{July\ 30,\ 1990}
 1
\shead{NAME}
ppmtorle -- convert a PBMPLUS/ppm image file into an RLE image file.
\shead{SYNOPSIS}
{\bf ppmtorle}
[
{\bf --h}
] [
{\bf --v}
] [
{\bf --a}
] [
{\bf --o}{\it \ outfile}
] [
{\it infile.ppm}
]
\shead{DESCRIPTION}
This program converts PBMPLUS full-color (ppm) image files into Utah
{\it RLE}{\rm (5)}
image files.  PBMPLUS/ppm image files contain the image dimensions and image
data in the form of RGB triplets.  When converting to an RLE file, the alpha
channel may be optionally computed.  The origins of PBMPLUS and Utah RLE files
are  in the upper left and lower left corners respectively.  Therefore, it is
common practice to send the output of this program through the "rleflip"
utility -- see examples below.
\par
The RLE file will contain a "true color" image
(24 bits).  These RLE files may then be viewed using any RLE image viewer.  When
they are displayed on an 8 bit display, the image must be dithered.  In order
to produce a better looking image (on 8 bit displays), it is recommended that
the image be quantizing (to 8 bit mapped color) prior to its display.  This may
be done by piping the output of this program into the Utah
{\it mcut}{\rm (1)}
or
{\it rlequant}{\rm (1)}
utilities.
An example of this is shown later.
\par
\shead{OPTIONS}
\begin{TPlist}{{\bf --v}}
\item[{{\bf --v}}]
This option will cause ppmtorle to operate in verbose mode.  The header
information is written to "stderr".  Actually, there is not much header
information stored in a PBMPLUS file, so this information is minimal.
\item[{{\bf --h}}]
This option allows the header of the PBMPLUS image to be dumped to "stderr"
without converting the file.  It is equivalent to using the --v option except
that no file conversion takes place.
\item[{{\bf --m}}]
This option will cause ppmtorle to use the RGB data to compute an alpha
channel in the resulting RLE file.  For any non-zero RGB data, the alpha
channel will contain a value of 255.  The resulting RLE image file will
contain three color channels and an alpha channel.
\item[{{\bf --o}{\it \ outfile}
}]
If specified, the output will be written to this file.  If 
{\it outfile}
is "--", or if it is not specified, the output will be written to the
standard output stream.
\item[{{\it infile.ppm}}]
The name of the PBMPLUS image data file to be converted.  This file must end
in ".ppm".  However, it is not necessary to supply the ".ppm" extension as it
will be added to the supplied name if not already there.
\end{TPlist}\shead{EXAMPLES}
\begin{TPlist}{ppmtorle --v test.ppm --o test.rle}
\item[{ppmtorle --v test.ppm --o test.rle}]
While running in verbose mode, convert test.ppm to RLE format and store
resulting data in test.rle.
\item[{ppmtorle test $|$ rleflip --v $|$ mcut $>$test.rle}]
Convert test.ppm to RLE format, flip if vertically, and convert to 8 bit
mapped color before storing data in test.rle
\item[{ppmtorle --h test}]
Dump the header information of the PBMPLUS file called test.ppm.
\end{TPlist}\shead{SEE ALSO}
{\it mcut}{\rm (1),}
{\it pgmtorle}{\rm (1),}
{\it rlequant}{\rm (1),}
{\it rletoppm}{\rm (1),}
{\it urt}{\rm (1),}
{\it RLE}{\rm (5).}
\shead{AUTHOR}
\nwl
Wesley C. Barris
\nwl
Army High Performance Computing Research Center (AHPCRC)
\nwl
Minnesota Supercomputer Center, Inc.
\newpage
% -*-LaTeX-*-
% Converted automatically from troff to LaTeX by tr2tex on Tue Aug  7 18:10:35 1990
% tr2tex was written by Kamal Al-Yahya at Stanford University
% (Kamal%Hanauma@SU-SCORE.ARPA)


%[troffman]{article}
%
%
% input file: pyrmask.1
%
% Copyright (c) 1986, University of Utah
\phead{PYRMASK}{1}{Nov\ 8,\ 1987}
 1
\shead{NAME}
pyrmask -- Blend two images together using Gaussian pyramids.
\shead{SYNOPSIS}
{\bf pyrmask} 
[
{\bf --l}
{\it levels}
] [
{\bf --o}
{\it outfile}
]
{\it inmask\ outmask\ maskfile}
\shead{DESCRIPTION}
{\it Pyrmask}
blends two images together by first breaking the images down into
separate bandpass images, combining these separate images, and then adding
the new bandpass images back into a single output image.  This can produce
very seamless blends of digital images.  The two images
are combined on the basis of a third "mask" image.  The resulting image
will contain the 
{\it inmask}
image where the mask contains a maximum value (255) and the 
{\it outmask} 
image where the mask contains zeros.  This is done on a channel by channel
basis, i.e. the maskfile should have data in each channel describing how
to combine each channel of the 
{\it inmask}
and
{\it outmask}
images.  All three images
must have exactly the same dimensions (both image size and number of channels).
For best results, it's often useful to filter the mask image a little with
{\it smush}{\rm (1)}
first.
\shead{OPTIONS}
\begin{TPlist}{{\bf --l}{\it \ levels}
}
\item[{{\bf --l}{\it \ levels}
}]
How many pyramid levels to use (maximum is log(2) of image size).
\item[{{\bf --o}{\it \ outfile}
}]
If specified, output will be written to this file, otherwise it will
go to stdout.
\end{TPlist}\shead{SEE ALSO}
{\it smush}{\rm (1),}
{\it rleswap}{\rm (1),}
{\it urt}{\rm (1),}
{\it RLE}{\rm (5),}
\nwl
Burt and Adelson, "A Multiresolution Spline With Applications to Image
Mosaics", %
\it ACM Transactions on Graphics\rm%
, October 1983, V2 \#4, p. 217.
\nwl
Ogden, Adelson, Bergen and Burt, "Pyramid-based Computer Graphics", %
\it RCA
Engineer\rm%
, Sept/Oct 1985, p. 4.
\shead{AUTHOR}
Rod Bogart
\shead{BUGS}
The current implementation has very
strict requirements for image sizes and dimensions.  The extensive
use of floating point computation makes it very slow for normal sized
images.  It also keeps all of the bandpass images in core at
once, which requires considerable amounts of memory.

Pyrmask is built on top of a library of functions for working with
Gaussian pyramids.  This library has yet to be documented or fully
tested.
\newpage
% -*-LaTeX-*-
% Converted automatically from troff to LaTeX by tr2tex on Tue Aug  7 18:10:35 1990
% tr2tex was written by Kamal Al-Yahya at Stanford University
% (Kamal%Hanauma@SU-SCORE.ARPA)


%[troffman]{article}
%
%
% input file: rastorle.1
%
% Copyright (c) 1988, University of Utah
\phead{RASTORLE}{1}{21\ June\ 1988}
 1
\shead{NAME}
rastorle -- Convert sun raster file image to an RLE format file.
\shead{SYNOPSIS}
{\bf rastorle}
[
{\bf --o}
{\it outfile}
] [ 
{\it infile.ras}
] 
\shead{DESCRIPTION}
{\it Rastorle}
converts a sun raster file to a file in the
Utah Raster Toolkit RLE format.
\shead{OPTIONS}
\begin{TPlist}{{\it infile.ras}}
\item[{{\it infile.ras}}]
The input sun raster will be read from this file, otherwise, input will
be taken from stdin.
\item[{{\bf --o}{\it \ outfile}
}]
If specified, output will be written to this file, otherwise it will
go to stdout.  The output should be run through 
{\it rleflip}{\rm (1),}
{\rm (}{\bf --v}{\rm )}
since RLE and Sun raster files disagree on where the origin is.
\end{TPlist}\shead{SEE ALSO}
{\it rleflip}{\rm (1),}
{\it rletorast}{\rm (1),}
{\it urt}{\rm (1),}
{\it RLE}{\rm (5).}
\shead{AUTHOR}
Berry Kercheval
\shead{BUGS}
It always produces a 24 bit output file, even from a pseudo-colored 8
bit input file.  Thus, the original color set is "lost".
\newpage
% -*-LaTeX-*-
% Converted automatically from troff to LaTeX by tr2tex on Tue Aug  7 18:10:36 1990
% tr2tex was written by Kamal Al-Yahya at Stanford University
% (Kamal%Hanauma@SU-SCORE.ARPA)


%[troffman]{article}
%
%
% input file: rawtorle.1
%
\phead{RAWTORLE}{1}{1990}
 1
\shead{NAME}
rawtorle -- Convert raw image data to RLE.
\shead{SYNOPSIS}
{\bf rawtorle}
[
{\bf --N}
] [
{\bf --s}
] [
{\bf --r}
] [
{\bf --w}
{\it width}
] [
{\bf --h}
{\it height}
] [
{\bf --f}
{\it header-size}
] [
{\bf --t}
{\it trailer-size}
] [
{\bf --n}
{\it nchannels}
] [
{\bf --a}
[
{\it alpha-value}
] ] [
{\bf --p}
{\it scanline-pad}
] [
{\bf --l}
{\it left-scanline-pad}
] [
{\bf --o}
{\it outfile}
] [
{\it infile}
] 
\shead{DESCRIPTION}
This program is used to convert image data in any of a number of "raw" forms to the
{\it RLE}{\rm (5)}
format.  The expected input size is computed from the arguments, so
that several images may be concatenated together and will be processed
in sequence.  In this case, the output file will contain several RLE images.
\shead{OPTIONS}
\begin{TPlist}{{\bf --N}}
\item[{{\bf --N}}]
The input is in non-interleaved order, as might be generated by the command
\nwl
cat pic.r pic.g pic.b
\item[{{\bf --s}}]
The input is in scanline interleaved order.
\item[{{\bf --r}}]
Reverse the channel order.  (E.g., data will be interpreted as ABGR
instead of RGBA.)
\item[{{\bf --w}{\it }{\bf width}
}]
Specify the width of the input image.
\item[{{\bf --h}{\it }{\bf height}
}]
Specify the height of the input image.
\item[{{\bf --f}{\it }{\bf header-size}
}]
This many bytes will be skipped before starting to read image data.
\item[{{\bf --t}{\it }{\bf trailer-size}
}]
This many bytes will be skipped at the end of the image data.
\item[{{\bf --n}{\it }{\bf nchannels}
}]
The input data has this many color channels.
\item[{%
\bf --a \rm%
[%
\it alpha-value\rm%
]}]
Generate a constant alpha channel.  The default value for
{\it alpha-value}
is 255.
\item[{{\bf --p}{\it }{\bf scanline-pad}
}]
This many bytes will be skipped at the end of each scanline.
\item[{{\bf --l}{\it }{\bf left-scanline-pad}
}]
This many bytes will be skipped at the beginning of each scanline.
\item[{{\bf --o}{\it }{\bf outfile}
}]
If specified, output will be written to this file, otherwise it will
go to stdout.
\item[{{\it infile}}]
The input will be read from this file, otherwise, input will
be taken from stdin.
\end{TPlist}\par\noindent
The input data is assumed to have an alpha channel if there are 2 or 4
channels.  The alpha channel is the last input channel unless
{\bf --r}
is specified, in which case it is the first.
\shead{EXAMPLES}
\begin{TPlist}{512x512 grayscale}
\item[{512x512 grayscale}]
rawtorle --w 512 --h 512 --n 1
\item[{640x512 raw RGB}]
rawtorle --w 640 --h 512 --n 3
\item[{picture.[rgb]}]
cat picture.[rgb] $|$ rawtorle --w 640 --h 512 --n 3 --N --r
\nwl
(I.e., separate red, green, blue image files.  This subsumes 
{\it graytorle}{\rm (1).)}
\item[{JPL ODL Voyager pics}]
rawtorle --w 800 --h 800 --f 2508 --t 1672 --n 1 --p 36
\item[{24bit Sun raster file}]
rawtorle --f 32 --w ... --h ... --n 3 --r
\nwl
(But 
{\it rastorle}{\rm (1)}
is easier.)
\item[{pic.\{000-100\}.[rgb]}]
cat pic.* $|$ rawtorle --w ... --h ... --n 3 --s --r
\nwl
(I.e., each color of each scanline is in a separate file.)
\end{TPlist}\shead{SEE ALSO}
{\it graytorle}{\rm (1),}
{\it rastorle}{\rm (1),}
{\it urt}{\rm (1),}
{\it RLE}{\rm (5).}
\shead{AUTHOR}
Martin Friedmann
\newpage
% -*-LaTeX-*-
% Converted automatically from troff to LaTeX by tr2tex on Tue Aug  7 18:10:37 1990
% tr2tex was written by Kamal Al-Yahya at Stanford University
% (Kamal%Hanauma@SU-SCORE.ARPA)


%[troffman]{article}
%
%
% input file: read98721.1
%
% Copyright (c), Program of Computer Graphics, Cornell University.
\phead{READ98721}{1}{Jun\ 11,\ 1987}
 1
\shead{NAME}
read98721 -- read an image from the HP--98721 frame buffer
\shead{SYNOPSIS}
{\bf read98721}
[
{\bf --b}
{\it red\ green\ blue}
] [
{\bf --d}
{\it display}
] [
{\bf --m}
] [
{\bf --o}
{\it outfile}
] [
{\bf --p}
{\it xpos\ ypos}
] [
{\bf --s}
{\it xsize\ ysize}
] [
{\bf --x}
{\it driver}
] [
{\bf --O}
] [
{\it comments}
]
\shead{DESCRIPTION}
This program reads an image from a
{\it HP--98721}
frame buffer and writes it to an
{\it RLE}{\rm (5)}
file. The file will contain three channels of 8 bits each for red, green,
and blue respectively. If an output file name is not specified the image
will be written to the standard output.  The default display device
and device driver are respectively
{\it /dev/crt98721}
and
{\it hp98721.}

\shead{OPTIONS}
\begin{TPlist}{{\bf --b}{\it \ red\ green\ blue}
}
\item[{{\bf --b}{\it \ red\ green\ blue}
}]
Specifies red, green and blue pixel values for the background.
\item[{{\bf --d}{\it \ display}
}]
Gives the name of the display device from which the image is to be read.
\item[{{\bf --m}
}]
Saves the device color maps. By default, no color maps are saved.
\item[{{\bf --o}{\it \ outfile}
}]
Writes the image to
{\it outfile.}
\item[{{\bf --p}{\it \ xpos\ ypos}
}]
Specifies the lower left corner of the portion of the screen to be
saved. The origin is the lower left corner of the display, which is
taken as the default starting position if this option is not specified.
\item[{{\bf --s}{\it \ xsize\ ysize}
}]
Specifies the size of the image to be read.
\item[{{\bf --x}{\it \ driver}
}]
Gives the name of the device driver to be used to communicate with
the display device.
\item[{{\bf --O}
}]
Specifies that the image has no background.
\end{TPlist}\par\noindent
The remaining arguments are taken to be comment strings of the form
{\it name=value}
, and are inserted in the header of the
{\it RLE}{\rm (5)}
output file.
\shead{SEE ALSO}
{\it getren}{\rm (1),}
{\it urt}{\rm (1),}
{\it RLE}{\rm (5).}
\shead{AUTHOR}
Filippo Tampieri, Program of Computer Graphics, Cornell University.


\newpage
% -*-LaTeX-*-
% Converted automatically from troff to LaTeX by tr2tex on Tue Aug  7 18:10:37 1990
% tr2tex was written by Kamal Al-Yahya at Stanford University
% (Kamal%Hanauma@SU-SCORE.ARPA)


%[troffman]{article}
%
%
% input file: repos.1
%
% Copyright (c) 1986, University of Utah
% Template man page.  Taken from wtm's page for getcx3d
\phead{REPOS}{1}{Month\ DD,\ YYYY}
 1
\shead{NAME}
repos -- reposition an RLE image
\shead{SYNOPSIS}
{\bf repos}
[
{\bf --p}
{\it xpos\ ypos}
]
[
{\bf --P}
{\it xinc\ yinc}
] [
{\bf --o}
{\it outfile}
] [ 
{\it infile}
] 
\shead{DESCRIPTION}
{\it repos}
repositions an RLE image.  Repos just changes the coordinates stored in
the RLE header (see
{\it RLE(5)),}
no modification is made to the image itself.
\shead{OPTIONS}
If neither of the following flags are specified,
{\bf --p}
0 0 is assumed.
\begin{TPlist}{{\bf --p}{\it \ xpos\ ypos}
}
\item[{{\bf --p}{\it \ xpos\ ypos}
}]
Reposition the image to the absolute coordinates
{\it xpos} ypos.
\item[{{\bf --P}{\it \ xinc\ yinc}
}]
Move the image by 
{\it xinc} yinc
pixels from where it currently is (relative movement).
\item[{{\it infile}}]
The input will be read from this file, otherwise, input will
be taken from stdin.
\item[{{\bf --o}{\it \ outfile}
}]
If specified, output will be written to this file, otherwise it will
go to stdout.
\end{TPlist}\shead{DIAGNOSTICS}
{\it Repos}
does not allow the image origin to have negative coordinates.
\shead{SEE ALSO}
{\it rlesetbg}{\rm (1),}
{\it urt}{\rm (1),}
{\it RLE}{\rm (5).}
\shead{AUTHORS}
Rod Bogart, John W. Peterson
\newpage
% -*-LaTeX-*-
% Converted automatically from troff to LaTeX by tr2tex on Tue Aug  7 18:10:38 1990
% tr2tex was written by Kamal Al-Yahya at Stanford University
% (Kamal%Hanauma@SU-SCORE.ARPA)


%[troffman]{article}
%
%
% input file: rlatorle.1
%
% Copyright (c) 1990, Minnesota Supercomputer Center, Inc.
\phead{RLATORLE}{1}{May\ 30,\ 1990}
 1
\shead{NAME}
rlatorle -- convert a Wavefront "rlb" image file into an RLE image file.
\shead{SYNOPSIS}
{\bf rlatorle}
[
{\bf --h}
] [
{\bf --v}
] [
{\bf --m}
] [
{\bf --o}{\it \ outfile}
] [
{\it infile.rla}
]
\shead{DESCRIPTION}
This program converts Wavefront image files (rlb format) into Utah
{\it RLE}{\rm (5)}
image files.  Wavefront image files store RGB data as well as a matte channel.
They also define a "bounding box" containing non-background pixels which is in
many cases smaller than the total image area.  Only this non-background area is
run length encoded.  When converting to an RLE file, the matte channel is
stored as an alpha channel and the "bounding box" dimensions are ignored.  It
is for this reason that in general the RLE version of the file will be larger
than its Wavefront counterpart.
\par
The RLE file will contain a "true color" image
(24 bits).  These RLE files may then be viewed using any RLE image viewer.  When
they are displayed on an 8 bit display, the image will be dithered.  In order
to produce a better looking image (on 8 bit displays), it is recommended that
the image be quantizing (to 8 bit mapped color) prior to its display.  This may
be done by piping the output of this program into the Utah
{\it mcut}{\rm (1)}
or
{\it rlequant}{\rm (1)}
utilities.
An example of this is shown later.
\par
\shead{OPTIONS}
\begin{TPlist}{{\bf --v}}
\item[{{\bf --v}}]
This option will cause rlatorle to operate in verbose mode.  The header
information is written to "stderr".
\item[{{\bf --h}}]
This option allows the header of the wavefront image to be dumped to "stderr"
without converting the file.  It is equivalent to using the --v option except
that no file conversion takes place.
\item[{{\bf --m}}]
This option will cause rlatorle to ignore the RGB data and use the matte
channel information to produce a monochrome image.  The resulting RLE image
file will contain only one color channel instead of the usual four
(RGB + alpha).
\item[{{\bf --o}{\it \ outfile}
}]
If specified, the output will be written to this file.  If 
{\it outfile}
is "--", or if it is not specified, the output will be written to the
standard output stream.
\item[{{\it infile.rla}}]
The name of the Wavefront image data file to be converted.  It is not necessary
to supply the ".rla" extension as it will be added to the supplied name if not
already there.
\end{TPlist}\shead{EXAMPLES}
\begin{TPlist}{rlatorle --v test.0001.rla --o test.rle}
\item[{rlatorle --v test.0001.rla --o test.rle}]
While running in verbose mode, convert test.0001.rla to RLE format and store
resulting data in test.rle.
\item[{rlatorle test.0001.rla $|$ mcut $>$test.rle}]
Convert test.0001.rla to RLE format and convert to 8 bit mapped color before
storing data in test.rle
\item[{rlatorle --h test.0001.rla}]
Dump the header information of the Wavefront file called test.0001.rla.
\end{TPlist}\shead{SEE ALSO}
{\it mcut}{\rm (1),}
{\it rlequant}{\rm (1),}
{\it rletorla}{\rm (1),}
{\it urt}{\rm (1),}
{\it RLE}{\rm (5).}
\shead{AUTHOR}
\nwl
Wesley C. Barris
\nwl
Army High Performance Computing Research Center (AHPCRC)
\nwl
Minnesota Supercomputer Center, Inc.
\newpage
% -*-LaTeX-*-
% Converted automatically from troff to LaTeX by tr2tex on Tue Aug  7 18:10:38 1990
% tr2tex was written by Kamal Al-Yahya at Stanford University
% (Kamal%Hanauma@SU-SCORE.ARPA)


%[troffman]{article}
%
%
% input file: rleClock.1
%
% Copyright (c) 1986, University of Utah
\phead{RLECLOCK}{1}{Dec\ 11,\ 1987}
 1
\shead{NAME}
rleClock -- Generate a clock face in RLE format
\shead{SYNOPSIS}
{\bf rleClock}
[
{\it options}
] [
{\bf --o}
{\it outfile}
]
\shead{DESCRIPTION}
This program generates an analog clock face in
{\it RLE}{\rm (5)}
file format and writes it to 
{\it outfile} 
or standard output.
The picture is a standard clock face with optional digital representation
above.
The user has control over the colors of the portions of the clock face, the
text, and the text background.
The user also has control over the clock configuration: number of ticks, scale
of the big and little hands, the values of the big and little hands, and the
format used to generate the digital portion.
\par
By default,
{\bf rleClock}
generates a standard analog clock face displaying the current time and
with no digital portion.
This default face is transparent, that is, the alpha channel is only defined
for the clock outline, tick marks, and the hands.
\par
On those options that expect colors, three numbers must be given after the
option switch.
These are values for red, green, and blue on a scale of zero through 255.
Those color options that are capitalized indicate the colors for the filled
regions (optional for the clock face and text but default for the hands).
Those that are not capitalized are for lines that either outline or constitute
the feature (the clock face is default, but they're optional for the hands).
\shead{OPTIONS}
\begin{TPlist}{{\bf --help}
}
\item[{{\bf --help}
}]
Prints a synopsis of the options.
\end{TPlist}\par
The options that control the value displayed by the clock are
\begin{TPlist}{{\bf --ls}{\it \ FLOAT}
}
\item[{{\bf --ls}{\it \ FLOAT}
}]
This specifies the full scale (360 degrees) of the little hand.
Default is 12.
\item[{{\bf --lv}{\it \ FLOAT}
}]
This specifies the value of the little hand, expressed in units of the little hand
full scale.
Default is the current hour time on a 12-hour scale.
\item[{{\bf --bs}{\it \ FLOAT}
}]
This specifies the full scale (360 degrees) of the big hand.
Default is 60.
\item[{{\bf --bv}{\it \ FLOAT}
}]
This specifies the value of the big hand, expressed in units of the big hand
full scale.
Default is the current minute time.
\end{TPlist}\par\noindent
The following options manage the display configuration of the clock:
\begin{TPlist}{{\bf --x}{\it \ INT}
}
\item[{{\bf --x}{\it \ INT}
}]
The INT specifies the width of the clock in pixels.
Default is 128.
\item[{{\bf --cy}{\it \ INT}
}]
The INT specifies the height of the clock face (minus text portion) in pixels.
The default is 128.
\item[{{\bf --ty}{\it \ INT}
}]
The INT specifies the height in pixels of the text portion of the display.
If it is zero (the default), no text portion is displayed.
\item[{{\bf --t}{\it \ INT}
}]
This specifies the number of tick marks to place around the clock.
The default is 12.
\item[{{\bf --lw}{\it \ INT}
}]
This specifies the line width in pixels of the clock face, the tick marks, the
optional hand borders, and the text.
The default is one, but two or three give better looking clocks.
\item[{{\bf --tf}{\it \ STR}
}]
The string describes how to show the digital portion of the clock.
The rules for forming STR are the same as for
{\it printf}
format strings, that is, a percent sign, optionally followed by field width
values, followed by a key letter.
In this case, the key letter may be
{\bf b,\ l,\ B,}
or
{\bf L.}
Lower case
{\bf b}
means to insert the integer value of the big hand and upper case
{\bf B}
means to insert the floating point value of the big hand.
Lower case
{\bf l}
means to insert the integer value of the little hand and upper case
{\bf L}
means to insert the floating point value of the little hand.
\item[{{\bf --fc}{\it \ R\ G\ B}
}]
This specifies the color in red, green, and blue, of the clock face.
\item[{{\bf --Fc}{\it \ R\ G\ B}
}]
This specifies the color to fill in inside the clock face, under the hands.  If
this option is not supplied, the clock is generated with no inside-face
background (by use of the alpha channel).
\item[{{\bf --Hc}{\it \ R\ G\ B}
}]
This specifies the color to draw in the hands with.
\item[{{\bf --hc}{\it \ R\ G\ B}
}]
This specifies the color to draw the outlines of the hands.
If it is not given, no outlines are drawn on the edges of the hands.
\item[{{\bf --tc}{\it \ R\ G\ B}
}]
This specifies the color of the text above the clock.
It only has effect if a text height (-ty) is supplied.
\item[{{\bf --Tc}{\it \ R\ G\ B}
}]
This specifies the color of a background field to place behind the text.
If omitted, no background (zero alpha channel) is drawn.
\end{TPlist}\shead{EXAMPLES}
\begin{TPlist}{{\bf rleClock}}
\item[{{\bf rleClock}}]
Generates a transparent clock face showing the current time and no digital
representation.
\item[{{\bf rleClock\ --ty\ 32}}]
Generates a current-time clock with digital representation above.
\item[{{\bf rleClock\ --Fc\ 255\ 0\ 0\ --Hc\ 0\ 0\ 255\ --lw\ 3\ --ty\ 96\ --tc\ 0\ 255\ 0\ --Tc\ 128\ 128\ 128}}]
Generates a clock with a red inside, white face, blue hands, wide lines, tall
text field, green test, and grey text background.
\item[{{\bf rleClock\ --ty\ 32\ --bs\ 10\ --bv\ 4.51\ --ls\ 100\ --lv\ 45.1\ --tf\ ``\%2l.\%2.2B''}}]
Generates a clock with the scale of the big hand set to 10 and it's value at
4.51, the scale and value of the little hand as 100 and 45.1, and the format
for the digital portion formatted as
{\bf \%2d.\%2.2f}
to print the integer little hand value (two spaces) and the floating point big
hand value."
\end{TPlist}\shead{SEE ALSO}
{\it urt}{\rm (1),}
{\it RLE}{\rm (5).}
\shead{AUTHOR}
Robert L. Brown, RIACS, NASA Ames Research Center
\shead{BUGS}
Not thoroughly checked when the line width is cranked up.
May dump core.

\newpage
% -*-LaTeX-*-
% Converted automatically from troff to LaTeX by tr2tex on Tue Aug  7 18:10:39 1990
% tr2tex was written by Kamal Al-Yahya at Stanford University
% (Kamal%Hanauma@SU-SCORE.ARPA)


%[troffman]{article}
%
%
% input file: rleaddcom.1
%
% Copyright (c) 1986, University of Utah
\phead{RLEADDCOM}{1}{2/2/87}
 1
\shead{NAME}
rleaddcom -- add picture comments to an RLE file.
\shead{SYNOPSIS}
\bf
rleaddcom 
[
{\bf --d}
] [
{\bf --i}
] [
{\bf --o}
{\it outfile}
] 
{\it infile}
{\it comments}
\shead{DESCRIPTION}
The 
{\it rleaddcom}
program will add one or more comments to an 
{\it RLE}{\rm (5)}
file.  If
{\it infile}
is "--", it will read from the standard input.  The modified 
{\it RLE}{\rm (5)}
file is written to the standard output if the
{\bf --o}
{\it outfile}
option is not given.  All remaining arguments on
the command line are taken as comments.  Comments are nominally of the
form
{\it name=value}
or
{\it name}{\rm .}
Any comment already in the file with the same
{\it name}
will be replaced.
\shead{OPTIONS}
\begin{TPlist}{{\bf --d}}
\item[{{\bf --d}}]
Will cause matching comments to be deleted, no comments will be added
in this case.
\item[{{\bf --i}}]
"In place."  The input file will be rewritten with the added comments.
This argument requires write permission to the directory containing
{\it infile}{\rm ,}
but does not require write permission for
{\it infile}{\rm .}
Of the special file name cases described in 
{\it urt}{\rm (1),}
only compressed files may be updated in place.  (It doesn't make sense
to update the output of a pipe "in place", does it?)

If
{\bf --o}{\it \ outfile}
is specified together with 
{\bf --i}{\rm ,}
then 
{\it outfile}
will not be modified until %
\it rleaddcom \rm%
has finished (this is
similar to the way that
{\it into}{\rm (1)}
works).
\end{TPlist}\shead{SEE ALSO}
{\it into}{\rm (1),}
{\it rlehdr}{\rm (1),}
{\it urt}{\rm (1),}
{\it RLE}{\rm (5).}
\shead{AUTHOR}
Spencer W. Thomas, University of Utah
\newpage
% -*-LaTeX-*-
% Converted automatically from troff to LaTeX by tr2tex on Tue Aug  7 18:10:39 1990
% tr2tex was written by Kamal Al-Yahya at Stanford University
% (Kamal%Hanauma@SU-SCORE.ARPA)


%[troffman]{article}
%
%
% input file: rleaddeof.1
%
% Copyright (c) 1990, University of Michigan
\phead{RLEADDEOF}{1}{June\ 12,\ 1990}
 1
\shead{NAME}
rleaddeof -- Put an end of image marker on an RLE file.
\shead{SYNOPSIS}
{\bf rleaddeof}
% sample options...
[
{\bf --o} 
{\it outfile}
] [ 
{\it infile}
]
\shead{DESCRIPTION}
This program reads a single
{\it RLE}{\rm (5)}
image and writes it to the specified output file with an end of image
code appended.  Some programs that generate RLE files do not properly
indicate the end of the image.  This will cause problems if several
such images are concatenated together in a single file.  It does not
hurt to run a correct %
\it RLE \rm%
file through
{\it rleaddeof}{\rm .}
\shead{OPTIONS}
\begin{TPlist}{{\bf --o} }
\item[{{\bf --o} }]
{\it outfile}
If specified, the output will be written to this file.  If 
{\it outfile}
is "--", or if it is not specified, the output will be written to the
standard output stream.
\item[{{\it infile}}]
The input will be read from this file.  If
{\it infile}
is "--" or is not specified, the input will be read from the standard
input stream.
\end{TPlist}\shead{SEE ALSO}
{\it urt}{\rm (1),}
{\it RLE}{\rm (5).}
\shead{AUTHOR}
Spencer W. Thomas
\newpage
% -*-LaTeX-*-
% Converted automatically from troff to LaTeX by tr2tex on Tue Aug  7 18:10:40 1990
% tr2tex was written by Kamal Al-Yahya at Stanford University
% (Kamal%Hanauma@SU-SCORE.ARPA)


%[troffman]{article}
%
%
% input file: rlebg.1
%
% Copyright (c) 1986, University of Utah
\phead{RLEBG}{1}{November\ 12,\ 1986}
 1
\shead{NAME}
rlebg -- generate simple backgrounds
\shead{SYNOPSIS}
{\bf rlebg}
[
{\bf --l}
] [
{\bf --v}
[
{\it top}
[
{\it bottom}
] ] ] [
{\bf --s}
{\it xsize\ ysize}
] [ 
{\bf --o}
{\it outfile}
]
{\it red\ green\ blue}
[
{\it alpha}
]
\shead{DESCRIPTION}
{\it rlebg}
generates a simple background.  These are typically used for compositing
below other images.  The values
{\it red\ green\ blue}
specify the pixel values (between 0 and 255) the background will have.
If 
{\it alpha}
is not specified, it defaults to 255 (full coverage).
{\it rlebg}
generates both constant backgrounds and backgrounds with continuous 
ramps. 
\shead{OPTIONS}
\begin{TPlist}{{\bf --s}{\it \ xsize\ ysize}
}
\item[{{\bf --s}{\it \ xsize\ ysize}
}]
This is the size of the background image.  The default is 512480.
\item[{{\bf --l}}]
Generate a linear ramp of pixel values.  If no ramp flag is given, 
{\it rlebg}
generates a constant background. 
\item[{{\bf --v}{\it \ top\ bottom}
}]
Generate a variable ramp, using a quadratic function (this looks best
with gamma corrected images).
{\it top}
and
{\it bottom}
are the fractions of the full color values at the top and bottom of the image.
The defaults are 1.0 0.1, respectively.  If both 
{\bf --v}
and
{\bf --l}
are given, then a linear ramp function is used instead of a quadratic ramp.
\item[{{\bf --o}{\it \ outfile}
}]
If specified, the output will be written to this file.  If 
{\it outfile}
is "--", or if it is not specified, the output will be written to the
standard output stream.
\end{TPlist}\par\noindent
\shead{SEE ALSO}
{\it rlesetbg}{\rm (1),}
{\it urt}{\rm (1),}
{\it RLE}{\rm (5).}
\shead{AUTHOR}
Rod Bogart
\newpage
% -*-LaTeX-*-
% Converted automatically from troff to LaTeX by tr2tex on Tue Aug  7 18:10:40 1990
% tr2tex was written by Kamal Al-Yahya at Stanford University
% (Kamal%Hanauma@SU-SCORE.ARPA)


%[troffman]{article}
%
%
% input file: rlebox.1
%
% Copyright (c) 1986, University of Utah
\phead{RLEBOX}{1}{Feb\ 12,\ 1987}
 1
\shead{NAME}
rlebox -- print bounding box for image in an RLE file.
\shead{SYNOPSIS}
{\bf rlebox}
[
{\bf --c}
] [
{\bf --m}
{\it margin}
] [
{\bf --v}
] [ 
{\it infile}
]
\shead{DESCRIPTION}
This program prints the bounding box for the image portion of an
{\it RLE}{\rm (5)}
file.  This is distinct from the bounds in the file header, since it
is computed solely on the basis of the actual image.  All background
pixels are ignored.
\shead{OPTIONS}
\begin{TPlist}{{\bf --c}}
\item[{{\bf --c}}]
Print the numbers in the order that crop wants them on its command
line.  The default order is
{\it xmin\ xmax\ ymin\ ymax}{\rm .}
If this option is specified, the bounds are printed in the order
{\it xmin\ ymin\ xmax\ ymax}{\rm .}
Thus, a file
{\it foo.rle}
could be trimmed to the smallest possible image by the command
\nofill
%.ta 1i
	crop `rlebox --c foo.rle` foo.rle
\fill
\item[{{\bf --m}{\it \ margin}
}]
Pads the output values by the margin given.
\item[{{\bf --v}}]
Verbose mode: label the numbers for human consumption.
\item[{{\it infile}}]
Name of the
{\it RLE}
file (defaults to standard input).
\end{TPlist}\shead{SEE ALSO}
{\it crop}{\rm (1),}
{\it urt}{\rm (1),}
{\it RLE}{\rm (5).}
\shead{AUTHOR}
Spencer W. Thomas, University of Utah
\newpage
% -*-LaTeX-*-
% Converted automatically from troff to LaTeX by tr2tex on Tue Aug  7 18:10:41 1990
% tr2tex was written by Kamal Al-Yahya at Stanford University
% (Kamal%Hanauma@SU-SCORE.ARPA)


%[troffman]{article}
%
%
% input file: rlecomp.1
%
% Copyright (c) 1986, University of Utah
% Template man page.  Taken from wtm's page for getcx3d
\phead{RLECOMP}{1}{December\ 20,\ 1986}
 1
\shead{NAME}
rlecomp -- Digital image compositor
\shead{SYNOPSIS}
{\bf rlecomp}
[
{\bf --o}
{\it outfile}
]
{\it Afile} operator Bfile 
\shead{DESCRIPTION}
{\it rlecomp}
implements an image compositor based on presence of an alpha, or matte channel
the image.  This extra channel usually defines
a mask which represents a sort of a cookie-cutter for the image.  This is the 
case when alpha is 255 (full coverage) for pixels inside the shape, zero
outside, and between zero and 255 on the boundary.  
If %
\it Afile \rm%
or %
\it Bfile \rm%
is just a single --, then 
{\it rlecomp}
reads that file from the standard input.

The operations behave as follows (assuming the operation is
"%
\it A operator B\rm%
"):
\begin{TPlist}{{\bf over}}
\item[{{\bf over}}]
The result will be the union of the two
image shapes, with %
\it A \rm%
obscuring %
\it B \rm%
in the region of overlap.
\item[{{\bf in}}]
The result is simply the image %
\it A \rm%
cut by the shape of %
\it B\rm%
.
None of the image data of %
\it B \rm%
will be in the result.
\item[{{\bf atop}}]
The result is the same shape as image %
\it B\rm%
, with %
\it A \rm%
obscuring
%
\it B \rm%
where the image shapes overlap.  Note this differs from
{\bf over}
because the portion of %
\it A \rm%
outside %
\it B\rm%
's shape does not appear
in the result. 
\item[{{\bf out}}]
The result image is image %
\it A \rm%
with the shape of %
\it B \rm%
cut out.
\item[{{\bf xor}}]
The result is the image data from both images that is
outside the overlap region.  The overlap region will be blank.
\item[{{\bf plus}}]
The result is just the sum of the image data.  Output values are
clipped to 255 (no overflow).  This
operation is actually independent of the alpha channels.
\item[{{\bf minus}}]
The result of %
\it A \rm%
-- %
\it B\rm%
, with underflow clipped to zero.  The
alpha channel is ignored (set to 255, full coverage).
\item[{{\bf diff}}]
The result of %
\it A \rm%
-- %
\it B\rm%
, with underflow wrapping around.
This is useful for comparing two very similar images.
\end{TPlist}\par\noindent
\shead{SEE ALSO}
{\it urt}{\rm (1),}
{\it RLE}{\rm (5),}
\nwl
"Compositing Digital Images", Porter and Duff,
{\it Proceedings\ of\ SIGGRAPH\ '84}
p.255
\shead{AUTHORS}
Rod Bogart and John W. Peterson
\shead{BUGS}
The other operations could be optimized as much as 
{\bf over}
is.

Rlecomp assumes both input files have the same number of channels.
\newpage
% -*-LaTeX-*-
% Converted automatically from troff to LaTeX by tr2tex on Tue Aug  7 18:10:44 1990
% tr2tex was written by Kamal Al-Yahya at Stanford University
% (Kamal%Hanauma@SU-SCORE.ARPA)


%[troffman]{article}
%
%
% input file: rleflip.1
%
% Copyright (c) 1986, University of Utah
% Template man page.  Taken from wtm's page for getcx3d
\phead{RLEFLIP}{1}{Month\ DD,\ YYYY}
 1
\shead{NAME}
rleflip -- Invert, reflect or rotate an image.
\shead{SYNOPSIS}
{\bf rleflip}
% sample options...
{\bf --\{rlhv}\}
[ 
{\bf --o}
{\it outfile}
] [
{\it infile}
]
\shead{DESCRIPTION}
{\it Rleflip}
inverts, reflects an image; or rotates left or right by 90 degrees.  The
picture's origin remains the same.  If no input file is specified, the
image is read from standard input.  For rotations of other than 90
degrees, use
{\it fant}{\rm (1).}
\shead{OPTIONS}
Exactly one of the following flags must be given:
\begin{TPlist}{{\bf --r}}
\item[{{\bf --r}}]
Rotate the image 90 degrees to the right
\item[{{\bf --l}}]
Rotate the image 90 degrees to the left
\item[{{\bf --h}}]
Reflect the image horizontally
\item[{{\bf --v}}]
Flip the image vertically
\item[{{\bf --o}{\it \ outfile}
}]
If specified, the output will be written to this file.  If 
{\it outfile}
is "--", or if it is not specified, the output will be written to the
standard output stream.
\end{TPlist}\shead{SEE ALSO}
{\it fant}{\rm (1),}
{\it urt}{\rm (1),}
{\it RLE}{\rm (5).}
\shead{AUTHOR}
John W. Peterson

\newpage
% -*-LaTeX-*-
% Converted automatically from troff to LaTeX by tr2tex on Tue Aug  7 18:10:45 1990
% tr2tex was written by Kamal Al-Yahya at Stanford University
% (Kamal%Hanauma@SU-SCORE.ARPA)


%[troffman]{article}
%
%
% input file: rlehdr.1
%
% Copyright (c) 1986, University of Utah
\phead{RLEHDR}{1}{Jan\ 22,\ 1987}
 1
\shead{NAME}
rlehdr -- Prints the header of an RLE file
\shead{SYNOPSIS}
{\bf rlehdr}
[
{\bf --d}
] [
{\bf --m}
] [
{\it infile}
]
\shead{DESCRIPTION}
This program prints the header of an
{\it RLE}{\rm (5)}
file in a human readable form.  If the optional
{\it infile}
argument is not supplied, input is read from standard input.
\shead{OPTIONS}
\begin{TPlist}{{\bf --d}}
\item[{{\bf --d}}]
Dump a very verbose version of the image contents as text to the
standard error output stream.
\item[{{\bf --m}}]
Print out the color map information.
\end{TPlist}\shead{SEE ALSO}
{\it urt}{\rm (1),}
{\it RLE}{\rm (5).}
\shead{AUTHOR}
Spencer W. Thomas, University of Utah
\newpage
% -*-LaTeX-*-
% Converted automatically from troff to LaTeX by tr2tex on Tue Aug  7 18:10:46 1990
% tr2tex was written by Kamal Al-Yahya at Stanford University
% (Kamal%Hanauma@SU-SCORE.ARPA)


%[troffman]{article}
%
%
% input file: rlehisto.1
%
% Copyright (c) 1986, University of Utah
% additions by Gregg Townsend, University of Arizona
\phead{RLEHISTO}{1}{June\ 25,\ 1990}
 1
\shead{NAME}
rlehisto -- generate histogram of RLE image.
\shead{SYNOPSIS}
{\bf rlehisto}
[
{\bf --b}
] [
{\bf --c}
] [
{\bf --t}
] [
{\bf --h}
{\it height}
] [
{\bf --o}
{\it outfile}
] [ 
{\it infile}
]
\shead{DESCRIPTION}
{\it Rlehisto}
counts the pixel values in an RLE file,
producing an RLE file graphing frequency of occurrence.
The horizontal axis runs from pixel value 0 on the left
to pixel value 255 on the right.
The height indicates the number of pixels seen for each pixel value.
Histograms are computed independently for each channel,
scaled identically, and then overlaid.
\par\noindent
The following options are available:
\begin{TPlist}{{\bf --b}}
\item[{{\bf --b}}]
Don't count the background pixel values when scaling the histogram.
This is useful if most pixels are colored the background color, so
that the interesting part of the histogram would be too small.  This
option is ignored if
{\bf --c}
is specified.
\item[{{\bf --c}}]
Output cumulative values instead of discrete values.
\item[{{\bf --t}}]
Print the totals instead of generating the histogram as an RLE file.
\item[{{\bf --h} }]
{\it height}
Scale the output image to the specified height.
The default is 256.
\item[{{\bf --o}{\it \ outfile}
}]
Direct the output to
{\it outfile.}
\end{TPlist}\shead{SEE ALSO}
{\it urt}{\rm (1),}
{\it RLE}{\rm (5).}
\shead{AUTHORS}
Gregg Townsend, University of Arizona;
Rod Bogart, University of Utah.
\newpage
% -*-LaTeX-*-
% Converted automatically from troff to LaTeX by tr2tex on Tue Aug  7 18:10:46 1990
% tr2tex was written by Kamal Al-Yahya at Stanford University
% (Kamal%Hanauma@SU-SCORE.ARPA)


%[troffman]{article}
%
%
% input file: rleldmap.1
%
% Copyright (c) 1986, University of Utah
% Template man page.  Taken from wtm's page for getcx3d
\phead{RLELDMAP}{1}{Nov\ 12,\ 1986}
 1
\shead{NAME}
rleldmap -- Load a new color map into an RLE file
\shead{SYNOPSIS}
{\bf rleldmap}
[
{\bf --\{ab}\}
] [
{\bf --n}
{\it nchan} length
] [
{\bf --s}
{\it bits}
] [
{\bf --l}
[
{\it factor}
] ] [
{\bf --g}
{\it gamma}
] [
{\bf --\{tf}\}
{\it file}
] [
{\bf --m}
{\it files} ...
] [
{\bf --r}
{\it rlefile}
] [
{\bf --o}
{\it outfile}
] [
{\it infile}
]
\shead{DESCRIPTION}
The program will load a specified color map into an
{\it RLE}{\rm (5)}
file.  The color map may be computed by 
{\it rleldmap}
or loaded from a file in one of several formats.  The input is read from
{\it infile}
or stdin if no file is given, and the result is written to
{\it outfile}
or stdout.  

The following terms are used in the description of the program and its options:
\begin{TPlist}{input map:}
\item[{input map:}]
A color map already in the input RLE file.
\item[{applied map:}]
The color map specified by the arguments to
{\it rleldmap}{\rm .}
This map will be applied to or will replace the input map to produce
the output map.
\item[{output map:}]
Unless
{\bf --a}
or
{\bf --b}
is specified, this is equal to the applied map.  Otherwise it will be
the composition of the input and applied maps.
\item[{map composition:}]
If the applied map is composed 
{\it after}
the input map, then the output map will be 
{\it applied\ map}{\rm [}{\it input\ map}{\rm ].}
Composing the applied map before the input map produces an output map
equal to
{\it input\ map}{\rm [}{\it applied\ map}{\rm ].}
The maps being composed must either have the same number of channels,
or one of them must have only one channel.  If an entry in the map
being used as a subscript is larger than the length of the map being
subscripted, the output value is equal to the subscript value.
The output map will be the same
length as the subscript map and will have the number of channels that
is the larger of the two.  If the input map is used as a subscript, it
will be downshifted the correct number of bits to serve as a subscript
for the applied map (since the color map in an
{\it RLE}{\rm (5)}
file is always stored left justified in 16 bit words).  This also
applies to the applied map if it is taken from an
{\it RLE}{\rm (5)}
file 
{\rm (}{\bf --r}
option below).  Note that if there is no input map, that the result of
composition will be exactly the applied map.
\item[{nchan:}]
The number of separate lookup tables (channels) making up the color
map.  This defaults to 3.
\item[{length:}]
The number of entries in each channel of the color map.  The default
is 256.
\item[{bits:}]
The size of each color map entry in bits.  The default value is the
log base 2 of the length.
\item[{range:}]
The maximum value of a color map entry, equal to 2**bits -- 1.
\end{TPlist}\shead{OPTIONS}
\begin{TPlist}{{\bf --a}}
\item[{{\bf --a}}]
Compose the applied map 
{\it after}
the input map.
\item[{{\bf --b}}]
Compose the applied map
{\it before}
the input map.  Only one of
{\bf --a}
or
{\bf --b}
may be specified.
\item[{{\bf --n}{\it \ nchan\ length}
}]
Specify the size of the applied map if it is not 3x256.  The
{\it length}
should be a power of two, and will be rounded up if necessary.  If
applying the map
{\it nchan}
must be either 1 or equal to the number of channels in the input map.
It may have any value if the input map has one channel or is not
present.
\item[{{\bf --s}{\it \ bits}
}]
Specify the size in bits of the color map entries.  I.e., only the top
{\it bits}
bits of each color map entry will be set.

Exactly one of the options
{\bf --l}{\rm ,}
{\bf --g}{\rm ,}
{\bf --t}{\rm ,}
{\bf --f}{\rm ,}
{\bf --m}{\rm ,}
or
{\bf --r}{\rm ,}
must be specified.
\item[{{\bf --l}{\it \ factor}
}]
Generate a linear applied map with the 
{\it nth}
entry equal to
\nwl
{\it 		range\ *\ min(1.0,\ factor*(n/(length--1)))}{\rm .}
\nwl
{\it Factor}
defaults to 1.0 if not specified.  Negative values of
{\it factor}
will generate a map with values equal to
\nwl
{\it 		range\ *\ max(0.0,\ 1.0\ --\ factor*(n/(length--1)))}{\rm .}
\item[{{\bf --g}{\it \ gamma}
}]
Generate an applied map to compensate for a display
with the given gamma.  The 
{\it nth}
entry is equal to
\nwl
{\it 		range\ *\ (n/(length--1))**(1/gamma)}{\rm .}
\item[{{\bf --t}{\it \ file}
}]
Read color map entries from a table in a text file.  The values for each
channel of a particular entry follow each other in the file.  Thus,
for an RGB color map, the file would look like:
\nwl
		red0	green0	blue0
\nwl
		red1	green1	blue1
\nwl
		...	...	...
\nwl
Line breaks in the input file are irrelevant.
\item[{{\bf --f}{\it \ file}
}]
Reads the applied map from a text file, with all the entries for each
channel following each other.  Thus, the input file above would appear
as
\nwl
		red0 red1 red2 ... (%
\it length \rm%
values)
\nwl
		green0 green1 green2 ... (%
\it length \rm%
values)
\nwl
		blue0 blue1 blue2 ... (%
\it length \rm%
values)
\nwl
As above, line breaks are irrelevant.
\item[{{\bf --m}{\it \ files\ ...}
}]
Read the color map for each channel from a separate file.  The number
of files specified must equal the number of channels in the applied
map.  (Note: the list of files must be followed by another flag
argument or by the null flag
{\bf --\/--}
to separate it from the
{\it infile}
specification.
\item[{{\bf --o}{\it \ outfile}
}]
The output will be written to the file
{\it outfile}
if this option is specified.  Otherwise the output will go to %
\it stdout\rm%
.
\item[{{\it infile}}]
The input will be taken from this file if specified.  Otherwise, the
input will be read from %
\it stdin\rm%
.
\end{TPlist}\shead{SEE ALSO}
{\it applymap}{\rm (1),}
{\it urt}{\rm (1),}
{\it RLE}{\rm (5).}
\shead{AUTHOR}
Spencer W. Thomas, University of Utah
\newpage
% -*-LaTeX-*-
% Converted automatically from troff to LaTeX by tr2tex on Tue Aug  7 18:10:47 1990
% tr2tex was written by Kamal Al-Yahya at Stanford University
% (Kamal%Hanauma@SU-SCORE.ARPA)


%[troffman]{article}
%
%
% input file: rlemandl.1
%
% Copyright (c) 1986, University of Utah
\phead{RLEMANDL}{1}{Nov\ 8,\ 1987}
 1
\shead{NAME}
rlemandl -- Compute images of the Mandelbrot set.
\shead{SYNOPSIS}
{\bf rlemandl} 
[ 
{\bf --o}
{\it outfile}
] [
{\bf --s}
{\it xsize\ ysize}
] [
{\bf --v}
]
{\it real\ imaginary\ width}
\shead{DESCRIPTION}
{\it Rlemandl}
computes images of the Mandelbrot set as an eight bit gray scale image.
The
{\it real}
and
{\it imaginary}
arguments specify the center of the area in the complex plane to be examined.
{\it Width}
specifies the width area to be examined.
\shead{OPTIONS}
\begin{TPlist}{{\bf --o}{\it \ outfile}
}
\item[{{\bf --o}{\it \ outfile}
}]
If specified, output will be written to this file, otherwise it will
go to stdout.
\item[{{\bf --s} }]
{\it xsize\ ysize}
Specify the resolution of the image (in pixels).
\item[{{\bf --v}}]
Print a message after every 50 lines are generated.
\end{TPlist}\shead{SEE ALSO}
{\it urt}{\rm (1),}
\nwl
"Computer Recreations," %
\it Scientific American\rm%
, August 1985.
\shead{AUTHOR}
John W. Peterson, University of Utah.
\shead{BUGS}
What a frob.  Gratuitous features are left as exercise to the reader.
The command name is spelled incorrectly.
\newpage
% -*-LaTeX-*-
% Converted automatically from troff to LaTeX by tr2tex on Tue Aug  7 18:10:47 1990
% tr2tex was written by Kamal Al-Yahya at Stanford University
% (Kamal%Hanauma@SU-SCORE.ARPA)


%[troffman]{article}
%
%
% input file: rlenoise.1
%
% Copyright (c) 1986, University of Utah
\phead{RLENOISE}{1}{June\ 15,\ 1988}
 1
\shead{NAME}
rlenoise -- Add random noise to an image
\shead{SYNOPSIS}
{\bf rlenoise}
[
{\bf --n}
{\it amount}
] [
{\bf --o}
{\it outfile}
] [ 
{\it infile}
] 
\shead{DESCRIPTION}
{\it Rlenoise}
adds uniform random noise to an image.  The peak-to-peak amplitude of
the noise can be specified with the
{\bf --n}
flag, the default value is 4.  This program may be
useful for trying to deal with quantization in an output device, if
you are able to trade spatial resolution for color resolution, and you
don't have a good characterization of the quantization function.
\shead{OPTIONS}
\begin{TPlist}{{\it infile}}
\item[{{\it infile}}]
The input will be read from this file, otherwise, input will
be taken from stdin.
\item[{{\bf --o}{\it \ outfile}
}]
If specified, output will be written to this file, otherwise it will
go to stdout.
\end{TPlist}\shead{SEE ALSO}
{\it urt}{\rm (1),}
{\it RLE}{\rm (5).}
\shead{AUTHOR}
Spencer W. Thomas, University of Michigan.
\shead{BUGS}
Of limited utility.

\newpage
% -*-LaTeX-*-
% Converted automatically from troff to LaTeX by tr2tex on Tue Aug  7 18:10:48 1990
% tr2tex was written by Kamal Al-Yahya at Stanford University
% (Kamal%Hanauma@SU-SCORE.ARPA)


%[troffman]{article}
%
%
% input file: rlepatch.1
%
% Copyright (c) 1986, University of Utah
\phead{RLEPATCH}{1}{Nov\ 8,\ 1987}
 1
\shead{NAME}
rlepatch -- patch smaller RLE files over a larger image.
\shead{SYNOPSIS}
{\bf rlepatch}
[
{\bf --o} 
{\it outfile}
]
{\it infile} patchfiles...

\shead{DESCRIPTION}
{\it Rlepatch} 
puts smaller RLE files on top of a larger RLE image.  One use
for rlepatch is to place small "fix" images on top of a larger image that
took a long time to compute.  Along with 
{\it repos}{\rm (1),}
{\it rlepatch}
can also be used as a simple way to build image mosaics.  

Unlike 
{\it rlecomp}{\rm (1),}
{\it rlepatch}
does not perform any arithmetic on the pixels.
If the patch images overlap, the patches specified last cover those 
specified first.

If the input files each contain multiple images, they are treated as
streams of images merging to form a stream of output images.  I.e.,
the 
{\it n}{\rm th}
image in each input file becomes part of the
{\it n}{\rm th}
image in the output file.  The process ceases as soon as any input
file reaches its end.
\shead{OPTIONS}
\begin{TPlist}{{\it infile}}
\item[{{\it infile}}]
The background image will be read from this file. 
If input is to be taken from
stdin, "--" must be specified here.
\item[{{\bf --o}{\it \ outfile}
}]
If specified, output will be written to this file, otherwise it will
go to stdout.
\end{TPlist}\shead{SEE ALSO}
{\it rlecomp}{\rm (1),}
{\it crop}{\rm (1),}
{\it repos}{\rm (1),}
{\it urt}{\rm (1),}
{\it RLE}{\rm (5).}
\shead{AUTHOR}
John W. Peterson, University of Utah.
\shead{BUGS}
{\it Rlepatch}
uses the "row" interface to the RLE library.  It would run
much faster using the "raw" interface, particularly for placing small
patches over a large image.   Even fixing it to work like
{\it rlecomp}
(which uses 
"raw" mode only for non-overlapping images) would make a major improvement.
\newpage
% -*-LaTeX-*-
% Converted automatically from troff to LaTeX by tr2tex on Tue Aug  7 18:10:48 1990
% tr2tex was written by Kamal Al-Yahya at Stanford University
% (Kamal%Hanauma@SU-SCORE.ARPA)


%[troffman]{article}
%
%
% input file: rleprint.1
%
% Copyright (c) 1990, University of Michigan
\phead{RLEPRINT}{1}{June\ 12,\ 1990}
 1
\shead{NAME}
rleprint -- Print the values of all the pixels in the file.
\shead{SYNOPSIS}
{\bf rleprint}
[ 
{\it infile}
]
\shead{DESCRIPTION}
This program reads an
{\it RLE}{\rm (5)}
image and prints the values of all the pixels to the standard output.
Each pixel is printed on a single line.  For example, a count of all
the unique pixel values in the file could be obtained by
\nwl
rleprint pic.rle $|$ sort --u $|$ wc
\begin{TPlist}{{\it infile}}
\item[{{\it infile}}]
The input will be read from this file.  If
{\it infile}
is "--" or is not specified, the input will be read from the standard
input stream.
\end{TPlist}\shead{SEE ALSO}
{\it rlehdr}{\rm (1),}
{\it urt}{\rm (1),}
{\it RLE}{\rm (5).}
\shead{AUTHOR}
Spencer W. Thomas
\shead{BUGS}
This program is of limited utility because of the sheer volume of
output it generates.
\newpage
% -*-LaTeX-*-
% Converted automatically from troff to LaTeX by tr2tex on Tue Aug  7 18:10:49 1990
% tr2tex was written by Kamal Al-Yahya at Stanford University
% (Kamal%Hanauma@SU-SCORE.ARPA)


%[troffman]{article}
%
%
% input file: rlequant.1
%
% Copyright (c) 1990, University of Michigan
\phead{RLEQUANT}{1}{June\ 12,\ 1990}
 1
\shead{NAME}
rlequant -- variance based color quantization for RLE images
\shead{SYNOPSIS}
{\bf rlequant}
[
{\bf --b}
{\it bits}
] [
{\bf --c}
{\it colors}
] [
{\bf --d}
] [
{\bf --f}
] [
{\bf --o}
{\it outfile}
] [ 
{\it infile}
] 
\shead{DESCRIPTION}
This program quantizes the colors in an RLE image using a
variance-based method.  See 
{\it colorquant}{\rm (3)}
for more details on the method.
\begin{TPlist}{{\bf --b}{\it \ bits}
}
\item[{{\bf --b}{\it \ bits}
}]
The colors in the input image will be "prequantized" to this many bits
before applying the variance-based method.  Two internal tables of
size 
{\bf 2\^{}(3*}{\rm bits}{\bf )}
are allocated, so values of
{\it bits}
greater than 6 are likely to cause thrashing or may prevent the
program from running at all.  The default value of
{\it bits}
is 5.  It must be less than or equal to 8 and greater than 0.
\item[{{\bf --c}{\it \ colors}
}]
The output image will be quantized to at most
{\it colors}
colors.  It might have fewer if the input image has only a few colors
itself.  The default value of
{\it colors}
is 256.  It must be less than or equal to 256.
\item[{{\bf --d}}]
Floyd Steinberg dithering is performed on the output.  This is very helpful
for images being quantized to a small number of colors.
\item[{{\bf --f}}]
If this option is specified, a faster approximation will be used.  In
most cases, the error so introduced will be barely noticeable.
\item[{{\bf --o}{\it \ outfile}
}]
The output will be written to the file
{\it outfile}{\rm .}
If not specified, or if
{\it outfile}
is "--", the output will be written to the standard output stream.
\item[{{\it infile}}]
This file contains one or more concatenated RLE images.  Each will be
processed in turn.  A separate quantization map will be constructed
for each image.  If not specified, or if
{\it infile}
is "--", the image(s) will be read from the standard input stream.
\end{TPlist}\shead{SEE ALSO}
{\it mcut}{\rm (1),}
{\it rledither}{\rm (1),}
{\it urt}{\rm (1),}
{\it colorquant}{\rm (3),}
{\it RLE}{\rm (5).}
\shead{AUTHOR}
Spencer W. Thomas
\nwl
Craig Kolb (Yale University) wrote the color quantization code.
\nwl
Rod Bogart wrote the dithering code.
\shead{BUGS}
If you specify 
{\bf --d}{\rm ,}
it needs to compute a complete inverse color map.  This takes a long
time (especially if you don't specify
{\bf --f}{\rm ).}
That's life.  
\newpage
% -*-LaTeX-*-
% Converted automatically from troff to LaTeX by tr2tex on Tue Aug  7 18:10:50 1990
% tr2tex was written by Kamal Al-Yahya at Stanford University
% (Kamal%Hanauma@SU-SCORE.ARPA)


%[troffman]{article}
%
%
% input file: rlescale.1
%
% Copyright (c) 1988, University of Utah
\phead{RLESCALE}{1}{Jun\ 15,\ 1988}
 1
\shead{NAME}
rlescale -- produce gray scale images.
\shead{SYNOPSIS}
{\bf rlescale}
[
{\bf --c}
] [
{\bf --n}
{\it nsteps}
] [
{\bf --o}
{\it outfile}
] [
{\it xsize}
] [
{\it ysize}
]
\shead{DESCRIPTION}
{\it Rlescale}
produces an RLE image containing a (more-or-less) standard gray scale
image.  Along the bottom are 8 colored patches (in the standard
primary and secondary colors).  Above these are a sequences of
logarithmically scaled gray patches.  By default, a 16 step scale is
produced.  The size of the output file (default 512 by 480) can be
set with the 
{\it xsize}
and
{\it ysize}
arguments.
\shead{OPTIONS}
\begin{TPlist}{{\bf --c}}
\item[{{\bf --c}}]
Produce red, green, blue, and gray scales.
\item[{{\bf --n}{\it \ nsteps}
}]
Specify the number of steps to be produced.
\end{TPlist}\shead{SEE ALSO}
{\it urt}{\rm (1),}
{\it RLE}{\rm (5).}
\shead{AUTHOR}
Spencer W. Thomas, University of Michigan.
\shead{BUGS}
Can't make an image narrower than 3 * 
{\it nsteps} 
pixels wide.
\newpage
% -*-LaTeX-*-
% Converted automatically from troff to LaTeX by tr2tex on Tue Aug  7 18:10:51 1990
% tr2tex was written by Kamal Al-Yahya at Stanford University
% (Kamal%Hanauma@SU-SCORE.ARPA)


%[troffman]{article}
%
%
% input file: rleselect.1
%
% -*- Text -*-
% Copyright (c) 1990, University of Michigan
\phead{RLESELECT}{1}{July\ 11,\ 1990}
 1
\shead{NAME}
rleselect -- Select images from an RLE file.
\shead{SYNOPSIS}
{\bf rleselect}
[
{\bf --i}
{\it infile} 
] [
{\bf --o} 
{\it outfile}
] [
{\bf --v}
] [ 
{\it image-numbers} ...
]
\shead{DESCRIPTION}
This program selects images from an
{\it RLE}{\rm (5)}
file containing multiple concatenated images.
The selected images are specified by number; the first image in the
file is number 1.  A negative number in the
{\it image-numbers}
list means that all images from the previous number in the list to the
absolute value of this number should be included.  A zero in the list
is taken as '--infinity', so that all images from the previous number
to the last image in the file will be included.  To try to clarify
this, some examples are included below.
\shead{OPTIONS}
\begin{TPlist}{{\bf --i}{\it \ infile}
}
\item[{{\bf --i}{\it \ infile}
}]
The input will be read from this file.  If
{\it infile}
is "--" or is not specified, the input will be read from the standard
input stream.
\item[{{\bf --o}{\it \ outfile}
}]
If specified, the output will be written to this file.  If 
{\it outfile}
is "--", or if it is not specified, the output will be written to the
standard output stream.
\item[{{\bf --v}}]
Verbose output.
\end{TPlist}\shead{EXAMPLES}
\begin{TPlist}{rleselect 1 4 5}
\item[{rleselect 1 4 5}]
Selects image 1, 4, and 5.
\item[{rleselect 4 1 5}]
Also selects image 1, 4, and 5.
\item[{rleselect 1 --4 5}]
Selects images 1 through 4 and 5 (i.e., 1 through 5).
\item[{rleselect 3 0}]
Selects images 3 through the last.
\item[{rleselect --4}]
Selects images 1 through 4.
\end{TPlist}\shead{SEE ALSO}
{\it rlesplit}{\rm (1),}
{\it urt}{\rm (1),}
{\it RLE}{\rm (5).}
\shead{AUTHOR}
Spencer W. Thomas
\newpage
% -*-LaTeX-*-
% Converted automatically from troff to LaTeX by tr2tex on Tue Aug  7 18:10:51 1990
% tr2tex was written by Kamal Al-Yahya at Stanford University
% (Kamal%Hanauma@SU-SCORE.ARPA)


%[troffman]{article}
%
%
% input file: rlesetbg.1
%
% Copyright (c) 1986, University of Utah
\phead{RLESETBG}{1}{December\ 20,\ 1986}
 1
\shead{NAME}
rlesetbg -- Set the background value in the RLE header.
\shead{SYNOPSIS}
{\bf rlesetbg}
% sample options...
[
{\bf --\{DO}\}
] [
{\bf --c}
{\it bgcolor} ...
] [
{\bf --o}
{\it outfile}
] [
{\it infile}
]
\shead{DESCRIPTION}
{\it rlesetbg}
sets the background color field in the image header of an
{\it RLE}{\rm (5)}
image (none of the actual
pixels are changed).  If 
{\it infile}
isn't specified, the image is read from stdin.  

The background color in the header is used to save space in the
run-length encoded file.  Runs of background-colored pixels longer
than 2 pixels are simply not saved.  (Doing this for runs of 1 or 2
background pixels can make the saved image larger than if no encoding
were done.)  Therefore, changing the background color with
{\it rlesetbg}
may still leave some pixels saved in the original background color.
The %
\bf --D \rm%
option will delete the background color altogether from
the header; this can be useful in certain circumstances, but can also
lead to very strange results.
\shead{OPTIONS}
\begin{TPlist}{{\bf --D}}
\item[{{\bf --D}}]
Delete any background specification that might be present.
\item[{{\bf --O}}]
Specifies that the image has no background, it overlays existing images.
\item[{%
\bf --c %
\it bgcolor ...%
\rm }]
Specifies the color values to set the background to.  There should be
at least as many values as there are color channels in the image.  Use
--\/-- or another option to separate the list of colors from 
{\it infile}{\rm .}
\item[{{\bf --o}{\it \ outfile}
}]
If specified, the output will be written to this file.  If 
{\it outfile}
is "--", or if it is not specified, the output will be written to the
standard output stream.
\end{TPlist}\shead{AUTHORS}
John W. Peterson and Rod Bogart
\shead{SEE ALSO}
{\it repos}{\rm (1),}
{\it urt}{\rm (1),}
{\it RLE}{\rm (5).}
\shead{BUGS}
This should really be part of a single program that does all header munging...
\newpage
% -*-LaTeX-*-
% Converted automatically from troff to LaTeX by tr2tex on Tue Aug  7 18:10:52 1990
% tr2tex was written by Kamal Al-Yahya at Stanford University
% (Kamal%Hanauma@SU-SCORE.ARPA)


%[troffman]{article}
%
%
% input file: rleskel.1
%
% Copyright (c) 1990, University of Michigan
\phead{RLESKEL}{1}{June\ 12,\ 1990}
 1
\shead{NAME}
rleskel -- A skeleton tool.
\shead{SYNOPSIS}
{\bf rleskel}
[
{\bf --o}
{\it outfile}
] [ 
{\it infile}
]
\shead{DESCRIPTION}
This program reads an
{\it RLE}{\rm (5)}
image and writes it to the specified output file.  All images in
the input file will be copied.  The program is not normally compiled
and installed, it exists solely to serve as a starting point for
writing simple "filter" tools, just as this man page serves as a starting
point for the documentation of simple tools.
\shead{OPTIONS}
\begin{TPlist}{{\bf --o}{\it \ outfile}
}
\item[{{\bf --o}{\it \ outfile}
}]
If specified, the output will be written to this file.  If 
{\it outfile}
is "--", or if it is not specified, the output will be written to the
standard output stream.
\item[{{\it infile}}]
The input will be read from this file.  If
{\it infile}
is "--" or is not specified, the input will be read from the standard
input stream.
\end{TPlist}\shead{SEE ALSO}
{\it urt}{\rm (1),}
{\it RLE}{\rm (5).}
\shead{AUTHOR}
Spencer W. Thomas
\newpage
% -*-LaTeX-*-
% Converted automatically from troff to LaTeX by tr2tex on Tue Aug  7 18:10:52 1990
% tr2tex was written by Kamal Al-Yahya at Stanford University
% (Kamal%Hanauma@SU-SCORE.ARPA)


%[troffman]{article}
%
%
% input file: rlespiff.1
%
% Copyright (c) 1990, University of Michigan
\phead{RLESPIFF}{1}{June\ 12,\ 1990}
 1
\shead{NAME}
rlespiff -- Use simple contrast enhancement to "spiff up" an image.
\shead{SYNOPSIS}
{\bf rlespiff}
[
{\bf --b}
{\it blacklevel}
] [
{\bf --s}
] [
{\bf --t}
{\it threshold}
] [
{\bf --w} 
{\it whitelevel}
] [
{\bf --o} 
{\it outfile} 
] [
{\it infile}
]
\shead{DESCRIPTION}
{\it Rlespiff}
"spiffs up" an
image by stretching the contrast range so that the darkest pixel maps
to black and the lightest to white.  If the
{\bf --s}
flag is given, the color channels will be treated separately.  This
will likely cause some drastic color shifts.
\shead{OPTIONS}
\begin{TPlist}{{\bf --b}{\it }{\bf blacklevel}
}
\item[{{\bf --b}{\it }{\bf blacklevel}
}]
The darkest input pixel will map to this pixel value in the
output image.  The default is 0.
\item[{{\bf --s}}]
If specified, each color channel will be mapped separately.
\item[{{\bf --t}{\it }{\bf threshold}
}]
This argument controls the number of samples of a pixel value that
should be considered insignificant (and will therefore be ignored).
It is specified in pixels/million.  A threshold of 4 applied to a
512x512 image would mean that any value that existed at only one pixel
would be ignored.  The default value is 10.
\item[{{\bf --w}{\it }{\bf whitelevel}
}]
The lightest input pixel will map to this pixel value in the output
image.  The default is 255.
\item[{{\bf --o}{\it }{\bf outfile}
}]
If specified, the output will be written to this file.  If 
{\it outfile}
is "--", or if it is not specified, the output will be written to the
standard output stream.
\item[{{\it infile}}]
The input will be read from this file.  If
{\it infile}
is "--" or is not specified, the input will be read from the standard
input stream.
\end{TPlist}\shead{SEE ALSO}
{\it urt}{\rm (1),}
{\it RLE}{\rm (5).}
\shead{AUTHOR}
Spencer W. Thomas
\newpage
% -*-LaTeX-*-
% Converted automatically from troff to LaTeX by tr2tex on Tue Aug  7 18:10:53 1990
% tr2tex was written by Kamal Al-Yahya at Stanford University
% (Kamal%Hanauma@SU-SCORE.ARPA)


%[troffman]{article}
%
%
% input file: rlesplice.1
%
% Copyright (c) 1986, University of Utah
\phead{RLESPLICE}{1}{Nov\ 12,\ 1986}
 1
\shead{NAME}
rlesplice -- Splice two RLE files together horizontally or vertically.
\shead{SYNOPSIS}
{\bf rlesplice}
{\bf --\{hv}\}
[
{\bf --c}
] [
{\bf --o}
{\it outfile}
] 
{\it infile1} infile2
\shead{DESCRIPTION}
{\it rlesplice}
splices two RLE images together either vertically or horizontally.  If one
image is smaller, then its background value or black is used to pad that image
to equal the larger dimension in the other image.  The 
{\bf --c}
flag is used to
specify whether the smaller image should be centered when put next to the
larger.  Presently the two images must have the same number of color channels,
the same presence of an alpha channel, and the same colormap size and length.
The colormap from the first image is used for the resultant image.
\shead{SEE ALSO}
{\it rlecomp}{\rm (1),}
{\it rlepatch}{\rm (1),}
{\it unslice}{\rm (1),}
{\it urt}{\rm (1),}
{\it RLE}{\rm (5).}
\shead{AUTHOR}
Martin R. Friedmann
\newpage
% -*-LaTeX-*-
% Converted automatically from troff to LaTeX by tr2tex on Tue Aug  7 18:10:53 1990
% tr2tex was written by Kamal Al-Yahya at Stanford University
% (Kamal%Hanauma@SU-SCORE.ARPA)


%[troffman]{article}
%
%
% input file: rlesplit.1
%
% Copyright (c) 1986, University of Utah
\phead{RLESPLIT}{1}{May\ 12,\ 1987}
 1
\shead{NAME}
rlesplit -- split a file of concatenated RLE images into separate image files
\shead{SYNOPSIS}
{\bf rlesplit}
[
{\bf --n}
{\it number}
[ 
{\it digits}
] ] [
{\bf --o}
{\it prefix}
] [ 
{\it infile}
]
\shead{DESCRIPTION}
This program will split a file containing a concatenated sequence of
{\it RLE}{\rm (5)}
images into separate files, each containing a single image.  The
output file names will be constructed from the input file name or a
specified prefix, and a sequence number.  If an input
{\it infile}
is specified, then the output file names will be in the form
"%
\it rlefileroot\rm%
.%
\it \#\rm%
.rle",
where
{\it rlefileroot}
is 
{\it infile}
with any ".rle" suffix stripped off.  If the option
{\bf --o}{\it \ prefix}
is specified, then the output file names will be of the form
"%
\it prefix\rm%
.%
\it \#\rm%
.rle".
If neither option is given, then the output file names will be in the
form "%
\it \#\rm%
.rle".
Input will be read from
{\it infile}
if specified, from standard input, otherwise.  File names will be
printed on the standard error output as they are generated.

The option
{\bf --n}
allows specification of an initial sequence number, and optionally the
number of digits used for the sequence number.  By default, numbering
starts at 1, and numbers are printed with 3 digits (and leading zeros).
\shead{SEE ALSO}
{\it rleselect}{\rm (1),}
{\it urt}{\rm (1),}
{\it RLE}{\rm (5).}
\shead{AUTHOR}
Spencer W. Thomas

\newpage
% -*-LaTeX-*-
% Converted automatically from troff to LaTeX by tr2tex on Tue Aug  7 18:10:54 1990
% tr2tex was written by Kamal Al-Yahya at Stanford University
% (Kamal%Hanauma@SU-SCORE.ARPA)


%[troffman]{article}
%
%
% input file: rleswap.1
%
% Copyright (c) 1986, University of Utah
\phead{RLESWAP}{1}{Jan\ 22,\ 1987}
 1
\shead{NAME}
rleswap -- swap the channels in an RLE file.
\shead{SYNOPSIS}
{\bf rleswap}
[
{\bf --v}
] [
{\bf --f} 
{\it from-channels,...}
] [
{\bf --t}
{\it to-channels,...}
] [
{\bf --d} 
{\it delete-channels,...}
] [
{\bf --p} 
{\it channel-pairs,...}
] [
{\bf --o}
{\it outfile}
] [
{\it infile}
]
\shead{DESCRIPTION}
This program can be used to select or swap the color channels in a
{\it RLE}{\rm (5)}
file.  The major options provide four different ways of specifying a
mapping between the channels in the input file and the output file.
Only one of the options
{\bf --f}{\rm ,}
{\bf --t}{\rm ,}
{\bf --d}{\rm ,}
or
{\bf --p}
may be specified.  If the optional
{\it infile}
is not given, input will be read from standard input.  A new
{\it RLE}{\rm (5)}
file will be written to the standard output or to
{\it outfile}{\rm ,}
if specified.  The output image will be similar to the input,
except for the specified channel remappings.
\shead{OPTIONS}
\begin{TPlist}{{\bf --v}}
\item[{{\bf --v}}]
Print the channel mappings that will be performed on the standard
error output.
\item[{{\bf --f}}]
Following this option is
a comma separated list of numbers indicating the input channel that
maps to each output channel in sequence.  I.e., the first number
indicates the input channel mapping to output channel 0.  The alpha
channel will be passed through unchanged if present.  Any input
channels not mentioned in the list will not appear in the output.
\item[{{\bf --t}}]
Following this option is
a comma separated list of numbers indicating the output channel to
which each input channel, in sequence, will map.  I.e., the first
number gives the output channel to which the first input channel will
map.  No number may be repeated in this list.  The alpha channel will
be passed through unchanged if present.  Any output channel not
mentioned in the list will not receive image data.  If there are fewer
numbers in the list than there are input channels, the excess input
channels will be ignored.  If there are more numbers than input
channels, it is an error.
\item[{{\bf --d}}]
Following this option is a comma separated list of numbers indicating
channels to be deleted from the input file.  All other channels will
be passed through unchanged.  The alpha channel may be specified as --1.
\item[{{\bf --p}}]
Following this option is
a comma separated list of pairs of channel numbers.
The first channel of each pair indicates a channel
in the input file that will be mapped to the
the channel in the output file indicated by the
second number in the pair.  No output channel
number may appear more than once.  Any input channel
not mentioned will not appear in the output file.
Any output channel not mentioned will not receive
image data.  The alpha channel may be specified as --1.
\end{TPlist}\shead{SEE ALSO}
{\it mergechan}{\rm (1),}
{\it urt}{\rm (1),}
{\it RLE}{\rm (5).}
\shead{AUTHOR}
Spencer W. Thomas, University of Utah

\newpage
% -*-LaTeX-*-
% Converted automatically from troff to LaTeX by tr2tex on Tue Aug  7 18:10:59 1990
% tr2tex was written by Kamal Al-Yahya at Stanford University
% (Kamal%Hanauma@SU-SCORE.ARPA)


%[troffman]{article}
%
%
% input file: rletoabA62.1
%
% Copyright (c) 1988, University of Utah
\phead{RLETOABA62}{1}{6\ February\ 1988}
 1
\shead{NAME}
rletoabA62 -- Convert from RLE Format to Abekas A62 Dump Format
\shead{SYNOPSIS}
{\bf rletoabA62}
[
{\bf --N}
] [
{\bf --f}{\it \ n}
] [
{\bf --n}{\it \ n}
] [
{\it infile}
]
\shead{DESCRIPTION}
{\it RletoabA62}
converts a raster file in the Utah Raster Toolkit RLE format into a format
suitable for writing to an Abekas A62 dump tape and subsequent loading onto the
Abekas disk.
The generated image is 768 pixels wide and 512 pixels high.
If the input is larger, it is truncated.
If it is smaller, it is padded on the top and right with black.
The output is written to
{\it stdout}{\rm ,}
and should be written to a tape in 24K byte blocks with
{\it dd}
as in the following:
\par\noindent
	dd of=/dev/rmt8 obs=24k
\par\noindent
Normally, the output is processed with a simple digital filter; this feature
may be turned off with an option.
{\it RletoabA62}
normally writes two consecutive frames, normally starting at frame 1.
\par\noindent
Input is taken from
{\it stdin}
unless a file name is given on the command line.
Only a single file may be given, and so if multiple invocations of
{\it rletoabA62}
are performed in a script, care must be taken to tell the program to convert
the data for the proper Abekas frame number (1-4).
Otherwise, the colors will appear wrong; they will be rotated on a vector scope
diagram.
\shead{EXAMPLE}
\par\noindent
The following example converts all files ending in
{\it .rle}
in the current directory and writes them to a tape.
Two frames are written per image and the frame number is incremented
accordingly.
\par\noindent
\ind{1.0in}
frame=1
\nwl
number=2
\nwl
for file in *.rle
\nwl
do
\nwl
	rletoabA62 --f \$frame \$file
\nwl
	frame=`expr \bs ( \bs ( \$frame -- 1 \bs ) + \$number \bs ) \% 4 + 1`
\nwl
done $|$
\nwl
dd of=/dev/rmt8 obs=24k
\ind{0.0em}
\shead{OPTIONS}
\par\noindent
Options are parsed by getopt(3).
\begin{TPlist}{{\bf --N}
}
\item[{{\bf --N}
}]
Do not apply digital filtering.
\item[{{\bf --f}{\it }{\bf n}
}]
Create the first frame as Abekas frame number
{\it n}{\rm ,}
having a value from one to four.
Consecutive frames increment this number modulo four.
The default is one.
\item[{{\bf --n}{\it }{\bf n}
}]
Write
{\it n}
frames of output, incrementing the frame number each time.
The default is two.
\end{TPlist}\shead{SEE ALSO}
{\it urt}{\rm (1),}
{\it RLE}{\rm (5).}
\shead{AUTHOR}
Bob Brown, RIACS.
\shead{BUGS}
This program does not preserve the aspect ratio of the input.
\newpage
% -*-LaTeX-*-
% Converted automatically from troff to LaTeX by tr2tex on Tue Aug  7 18:11:00 1990
% tr2tex was written by Kamal Al-Yahya at Stanford University
% (Kamal%Hanauma@SU-SCORE.ARPA)


%[troffman]{article}
%
%
% input file: rletoascii.1
%
% Copyright (c) 1990, University of Michigan
\phead{RLETOASCII}{1}{Jun\ 18,\ 1990}
 1
\shead{NAME}
rletoascii -- Print an RLE image as ASCII chars.
\shead{SYNOPSIS}
{\bf rletoascii}
[
{\bf --S}
{\it asciistr}
] [
{\bf --r}
] [
{\bf --o}
{\it outfile}
] [
{\it infile}
]
\shead{DESCRIPTION}
{\it Rletoascii}
reads a file in
{\it RLE}{\rm (5)}
format, converts it to black and white, then dumps it as ASCII characters. 
The 0 to 255 range of pixel values in
the image is scaled to the length of
{\it asciistr}
and a the character at that position in the string is printed for each pixel.
Input will be read from
{\it infile}
if specified, from standard input, otherwise.  Output dumps to 
standard output, or
{\it outfile,} 
if specified.

Usually, the input will need to be resized by
{\it fant}{\rm (1)}
or
{\it rlezoom}{\rm (1)}
to make it small enough to fit on the screen and to adjust the pixel aspect
ratio to the "character aspect ratio" of the terminal.  To get it
"right side up", use
{\it rleflip}{\rm (1)}
with the
{\bf --v}
option.  Finally, it may be helpful to maximize the dynamic range with
{\it rlespiff}{\rm (1).}
\shead{OPTIONS}
\begin{TPlist}{{\bf --S}{\it }{\bf asciistr}
}
\item[{{\bf --S}{\it }{\bf asciistr}
}]
Specifies the range of ascii characters for conversion.  The default string
(%
\bf @BR*\#\$PX0woIcv:+!\~{}"., \rm%
)
was designed to look good with the X 6x13 font.  
\item[{{\bf --r}}]
Reverse video.  This causes the 0 to 255 range to be mapped to the reverse of
the ascii string.
\end{TPlist}\shead{SEE ALSO}
{\it fant}{\rm (1),}
{\it rleflip}{\rm (1),}
{\it rlespiff}{\rm (1),}
{\it rlezoom}{\rm (1),}
{\it urt}{\rm (1),}
{\it RLE}{\rm (5).}
\shead{AUTHOR}
Rod G. Bogart, University of Michigan.
\shead{DEFICIENCIES}
Could be rewritten to use overprinting for output to a real printer.
\newpage
% -*-LaTeX-*-
% Converted automatically from troff to LaTeX by tr2tex on Tue Aug  7 18:11:00 1990
% tr2tex was written by Kamal Al-Yahya at Stanford University
% (Kamal%Hanauma@SU-SCORE.ARPA)


%[troffman]{article}
%
%
% input file: rletogif.1
%
% Copyright (c) 1990, University of Michigan
\phead{RLETOGIF}{1}{July\ 3,\ 1990}
 1
\shead{NAME}
rletogif -- Convert RLE files to GIF format.
\shead{SYNOPSIS}
{\bf rletogif}
[
{\bf --o} 
{\it outfile.gif}
] [
{\it infile.rle}
]
\shead{DESCRIPTION}
This program converts an
{\it RLE}{\rm (5)}
image file to 
{\it GIF} 
format.  The input file must be a single channel
(8 bit) image.  Three channel (24 bit) images can be converted to
single channel images using the programs
{\it tobw}{\rm (1),}
{\it to8}{\rm (1),}
{\it mcut}{\rm (1),}
or
{\it rlequant}{\rm (1).}
The input image will be flipped vertically, since the 
{\it GIF}
origin is in the upper left, and the
{\it RLE}
origin is in the lower left.  Only a single image will be converted.
\shead{OPTIONS}
\begin{TPlist}{{\bf --o}{\it \ outfile.gif}
}
\item[{{\bf --o}{\it \ outfile.gif}
}]
If specified, the output will be written to this file.  If 
{\it outfile.gif}
is "--", or if it is not specified, the output will be written to the
standard output stream.
\item[{{\it infile.rle}}]
The input will be read from this file.  If
{\it infile.rle}
is "--" or is not specified, the input will be read from the standard
input stream.
\end{TPlist}\shead{SEE ALSO}
{\it to8}{\rm (1),}
{\it mcut}{\rm (1),}
{\it rlequant}{\rm (1),}
{\it giftorle}{\rm (1),}
{\it urt}{\rm (1),}
{\it RLE}{\rm (5).}
\shead{AUTHOR}
Bailey Brown, University of Michigan
\newpage
% -*-LaTeX-*-
% Converted automatically from troff to LaTeX by tr2tex on Tue Aug  7 18:11:01 1990
% tr2tex was written by Kamal Al-Yahya at Stanford University
% (Kamal%Hanauma@SU-SCORE.ARPA)


%[troffman]{article}
%
%
% input file: rletogray.1
%
% Copyright (c) 1988, University of Utah
\phead{RLETOGRAY}{1}{Jun\ 24,\ 1988}
 1
\shead{NAME}
rletogray -- Splits an RLE format file into gray scale images.
\shead{SYNOPSIS}
{\bf rletogray}
[
{\bf --o}
{\it prefix}
] [
{\it infile}
]

\shead{DESCRIPTION}
{\it Rletogray}
reads a file in
{\it RLE}{\rm (5)}
format and splits the file into unencoded binary files, one for each channel
in the RLE file.  The output file names will be constructed from the
input file name or a specified prefix.
\par
If an input 
{\it infile}
is specified, then the output file names will be in the form
"%
\it rlefileroot\rm%
.\{alpha, red, green, blue\}",
where
{\it rlefileroot}
is 
{\it infile}
with any ".rle" suffix stripped off.  If the option
{\bf --o}{\it \ prefix}
is specified, then the output file names will be of the form
"%
\it prefix\rm%
.\{alpha, red, green, blue\}".
If neither option is given, then the output file names will be 
"out.\{alpha, red, green, blue\}".
Input will be read from
{\it infile}
if specified, from standard input, otherwise.  If more channels than
just red, green, blue, and alpha are
present in the input, numeric suffixes will be used for the others.
\shead{OPTIONS}
\begin{TPlist}{{\bf --o}{\it }{\bf prefix}
}
\item[{{\bf --o}{\it }{\bf prefix}
}]
Specifies the output file name prefix to be used.
\item[{{\it infile}}]
This option is used to name the input file.  If not present, input is taken
from
{\it stdin.}
\end{TPlist}\shead{SEE ALSO}
{\it rletoraw}{\rm (1),}
{\it urt}{\rm (1),}
{\it RLE}{\rm (5).}
\shead{AUTHOR}
Michael J. Banks, University of Utah.
\newpage
% -*-LaTeX-*-
% Converted automatically from troff to LaTeX by tr2tex on Tue Aug  7 18:11:01 1990
% tr2tex was written by Kamal Al-Yahya at Stanford University
% (Kamal%Hanauma@SU-SCORE.ARPA)


%[troffman]{article}
%
%
% input file: rletopaint.1
%
% Copyright (c) 1986, University of Utah
% Template man page.  Taken from wtm's page for getcx3d
\phead{RLETOPAINT}{1}{Month\ X,\ YYYY}
 1
\shead{NAME}
rletopaint -- convert an RLE file to MacPaint format using dithering
\shead{SYNOPSIS}
{\bf rletopaint}
% sample options...
[
{\bf --l}
] [
{\bf --r}
] [
{\bf --g}
[
{\it gamma}
]
] [
{\bf --o}
{\it outfile.paint}
] [
{\it infile} 
]
\shead{DESCRIPTION}
{\it Rletopaint}
converts a file from
{\it RLE}{\rm (5)}
format to MacPaint format.  The program uses 
dithering to convert from a full 24 bit color image to a bitmapped image.
If the RLE file is larger than a MacPaint image (576720) it is cropped to 
fit.

Because MacPaint files have their coordinate origin in the upper left instead
of the lower left, the RLE file should be piped through
{\it rleflip}{\rm (1)\ --v}
before 
{\it rletopaint.}

The resulting file can be downloaded to a Macintosh in binary mode,
and should be given a type of 
{\it PNTG}
and a creator of 
{\it MPNT}{\rm ,}
so it will be recognized as a MacPaint file.
\shead{OPTIONS}
\begin{TPlist}{{\bf --l}}
\item[{{\bf --l}}]
Use a linear map in the conversion from 24 bits to bitmapped output.
\item[{%
\bf --g\ %
\rm [%
\it \ gamma\ %
\rm ]}]
Use a gamma map of 
{\it gamma}
(gamma is 2.0 if not specified).
\item[{{\bf --r}}]
Invert the sense of the output pixels (white on black instead of black
on white).  For normal images, you probably want this flag.
\end{TPlist}\shead{SEE ALSO}
{\it painttorle(1),}
{\it urt}{\rm (1),}
{\it RLE}{\rm (5).}
\shead{AUTHOR}
John W. Peterson.  Byte compression routine by Jim Schimpf.
\shead{BUGS}
Should use a color map in the file, if present.
\newpage
% -*-LaTeX-*-
% Converted automatically from troff to LaTeX by tr2tex on Tue Aug  7 18:11:02 1990
% tr2tex was written by Kamal Al-Yahya at Stanford University
% (Kamal%Hanauma@SU-SCORE.ARPA)


%[troffman]{article}
%
%
% input file: rletoppm.1
%
% Copyright (c) 1990, Minnesota Supercomputer Center, Inc.
\phead{RLETOPPM}{1}{July\ 20,\ 1990}
 1
\shead{NAME}
rletoppm -- convert a Utah RLE image file into a PBMPLUS/ppm image file.
\shead{SYNOPSIS}
{\bf rletoppm}
[
{\bf --h}
] [
{\bf --v}
] [
{\it infile}
]
\shead{DESCRIPTION}
This program converts Utah
{\it RLE}{\rm (5)}
image files into PBMPLUS full-color (ppm) image files.  Rletoppm will handle
four types of RLE files: Grayscale (8 bit data, no color map), Pseudocolor
(8 bit data with a color map), Truecolor (24 bit data with color map), and
Directcolor (24 bit data, no color map).  Since the orgins for the RLE and
PBMPLUS image file formats are in different locations, it may be necessary
to "flip" the RLE image before converting to the PBMPLUS format -- see examples.
\shead{OPTIONS}
\begin{TPlist}{{\bf --v}}
\item[{{\bf --v}}]
This option will cause rletoppm to operate in verbose mode.  Header information
is printed to "stderr".
\item[{{\bf --h}}]
This option allows the header of the RLE file to be dumped to "stderr" without
converting the file.  It is equivalent to using the --v option except that no
file conversion takes place.
\item[{{\bf infile}}]
The input will be read from this file.  If
{\it infile}
is "--" or is not specified, the input will be read from the standard
input stream.
The resulting
PBMPLUS/ppm data will be sent to "stdout".
\end{TPlist}\shead{EXAMPLES}
\begin{TPlist}{rletoppm --v lenna.rle $>$lenna.ppm}
\item[{rletoppm --v lenna.rle $>$lenna.ppm}]
While running in verbose mode, convert lenna.rle to PBMPLUS/ppm format and
store resulting data in lenna.ppm.
\item[{rleflip --v lenna.rle $|$ rletoppm $>$lenna.ppm}]
Flip the contents of lenna.rle then convert to PBMPLUS/ppm format and store
resulting data in lenna.ppm.
\item[{rletoppm --h test.rle}]
Dump the header information of the RLE file called test.rle.
\end{TPlist}\shead{SEE ALSO}
{\it ppmtorle}{\rm (1),}
{\it pgmtorle}{\rm (1),}
{\it urt}{\rm (1),}
{\it RLE}{\rm (5)}
\shead{AUTHOR}
\nwl
Wesley C. Barris
\nwl
Army High Performance Computing Research Center (AHPCRC)
\nwl
Minnesota Supercomputer Center, Inc.
\newpage
% -*-LaTeX-*-
% Converted automatically from troff to LaTeX by tr2tex on Tue Aug  7 18:11:02 1990
% tr2tex was written by Kamal Al-Yahya at Stanford University
% (Kamal%Hanauma@SU-SCORE.ARPA)


%[troffman]{article}
%
%
% input file: rletops.1
%
% Copyright (c) 1986, University of Utah
% Template man page.  Taken from wtm's page for getcx3d
\phead{RLETOPS}{1}{December\ 20,\ 1986}
 1
\shead{NAME}
rletops -- Convert RLE images to PostScript
\shead{SYNOPSIS}
{\bf rletops}
[
{\bf --C}
] [
{\bf --a}
{\it aspect}
] [
{\bf --c}
{\it center}
] [
{\bf --h}
{\it height}
] [
{\bf --o}
{\it outfile.ps}
] [
{\bf --s}
] [
{\it infile}
]
\shead{DESCRIPTION}
{\it Rletops}
converts
{\it RLE}{\rm (5)}
images into 
{\it PostScript}{\rm .}
The conversion uses the 
{\it PostScript}
{\bf image}
operator, instructing the device to reproduce the image to the best of its
abilities.  If
{\it infile}
isn't specified, the RLE image is read from stdin.  The PostScript output is
dumped to stdout, or to
{\it outfile.ps,}
if specified.
\shead{OPTIONS}
\begin{TPlist}{{\bf %
\bf --a} %
\it aspect%
\rm }
\item[{{\bf %
\bf --a} %
\it aspect%
\rm }]
Specify aspect ratio of image.  Default is 1.0 (note PostScript uses square
pixels).
\item[{{\bf --C}}]
Causes a color PostScript image to be generated.
This creates larger files and uses the PostScript
{\bf colorimage}
operator, which is not recognized by all devices.
The default is monochrome.
\item[{{\bf %
\bf --c} %
\it center%
\rm }]
Centers the images about a point
{\it center}
inches from the left edge of the page (or left margin if
{\bf --s}
is specified).  Default is 4.25 inches.
\item[{{\bf %
\bf --h} %
\it height%
\rm }]
Specifies the height (in inches) the image is to appear on the page.  The
default is three inches.  The width of the image is calculated from
the image height, aspect ratio, and pixel dimensions.
\item[{{\bf --s}}]
Specifies image is to be generated in "Scribe Mode."  The image is generated
without a PostScript
{\it showpage}
operator at the end, and the default image center is changed to 3.25 inches
from the margin (which usually is 1 inch).
This is to generate PostScript files that can be included in Scribe documents
with the @Picture command.  Images may also be included in LaTex documents
with local conventions like the \bs special\{psfile=image.ps\} command.  
\end{TPlist}\shead{NOTES}
On devices like the Apple LaserWriter, 
{\it rletops}
generates large PostScript files that take a non-trivial amount of time
to download and print.  A 512x512 image takes about ten minutes.
For including images in documents at the default sizes, 256x256 is
usually sufficient resolution.
\shead{SEE ALSO}
{\it avg4}{\rm (1),}
{\it urt}{\rm (1),}
{\it RLE}{\rm (5).}
\shead{AUTHORS}
Rod Bogart, John W. Peterson, Gregg Townsend.

Portions are based on a program by Marc Majka.
\shead{BUGS}
Due to a mis-understanding with the PostScript interpreter,
{\it rletops}
always rounds the image size up to an even number of scanlines.
\newpage
% -*-LaTeX-*-
% Converted automatically from troff to LaTeX by tr2tex on Tue Aug  7 18:11:03 1990
% tr2tex was written by Kamal Al-Yahya at Stanford University
% (Kamal%Hanauma@SU-SCORE.ARPA)


%[troffman]{article}
%
%
% input file: rletorast.1
%
\phead{RLETORAST}{1}{1990}
 1
\shead{NAME}
rletorast -- Convert an RLE file to a Sun rasterfile.
\shead{SYNOPSIS}
{\bf rletorast}
[
{\bf --o} 
{\it outfile.ras}
] [ 
{\it infile}
]
\shead{DESCRIPTION}
This program converts an
{\it RLE}{\rm (5)}
file to a Sun raster file.
\begin{TPlist}{{\bf --o}{\it }{\bf outfile.ras}
}
\item[{{\bf --o}{\it }{\bf outfile.ras}
}]
If specified, the output will be written to this file.  If 
{\it outfile.ras}
is "--", or if it is not specified, the output will be written to the
standard output stream.  The input file should have either 1 or 3
channels, and may have an alpha channel.  Depending on the input,
either a gray scale or color raster file will be generated.  If an
alpha channel is present, a 32 bit raster will always be made.
\item[{{\it infile}}]
The input will be read from this file.  If
{\it infile}
is "--" or is not specified, the input will be read from the standard
input stream.

The programs
{\it mcut}{\rm (1),}
{\it rlequant}{\rm (1),}
{\it to8}{\rm (1),}
and
{\it tobw}{\rm (1)}
will make a 1 channel RLE image from an 3 channel (full color) image.
If the original image also had an alpha channel, 
{\it rleswap} -d -1
can be used to delete it.
\end{TPlist}\shead{SEE ALSO}
\raggedright
{\it mcut}{\rm (1),}
{\it rastorle}{\rm (1),}
{\it rlequant}{\rm (1),}
{\it rleswap}{\rm (1),}
{\it to8}{\rm (1),}
{\it tobw}{\rm (1),}
{\it urt}{\rm (1),}
{\it RLE}{\rm (5).}
%.ad b
\shead{AUTHOR}
Ed Falk, Sun Microsystems.
\newpage
% -*-LaTeX-*-
% Converted automatically from troff to LaTeX by tr2tex on Tue Aug  7 18:11:04 1990
% tr2tex was written by Kamal Al-Yahya at Stanford University
% (Kamal%Hanauma@SU-SCORE.ARPA)


%[troffman]{article}
%
%
% input file: rletoraw.1
%
\phead{RLETORAW}{1}{1990}
 1
\shead{NAME}
rletoraw -- Convert RLE file to raw RGB form.
\shead{SYNOPSIS}
{\bf rletoraw}
[
{\bf --o} 
{\it outfile}
] [
{\it infile}
]
\shead{DESCRIPTION}
This program converts an
{\it RLE}{\rm (5)}
image to a raw RGB form.  The output file is a stream of pixels
(RGBRGB...), in left-to-right, bottom-to-top order.  The width and
height of the input image will be printed on the standard error
stream.
\shead{OPTIONS}
\begin{TPlist}{{\bf --o}{\it }{\bf outfile}
}
\item[{{\bf --o}{\it }{\bf outfile}
}]
If specified, the output will be written to this file.  If 
{\it outfile}
is "--", or if it is not specified, the output will be written to the
standard output stream.
\item[{{\it infile}}]
The input will be read from this file.  If
{\it infile}
is "--" or is not specified, the input will be read from the standard
input stream.
\end{TPlist}\shead{SEE ALSO}
{\it rawtorle}{\rm (1),}
{\it urt}{\rm (1),}
{\it RLE}{\rm (5).}
\shead{AUTHOR}
Martin Friedmann
\shead{BUGS}
Input files must have red, green, and blue channels.  If not, bogus
data will be generated in the missing colors.
\newpage
% -*-LaTeX-*-
% Converted automatically from troff to LaTeX by tr2tex on Tue Aug  7 18:11:04 1990
% tr2tex was written by Kamal Al-Yahya at Stanford University
% (Kamal%Hanauma@SU-SCORE.ARPA)


%[troffman]{article}
%
%
% input file: rletorla.1
%
% Copyright (c) 1990, Minnesota Supercomputer Center, Inc.
\phead{RLETORLA}{1}{May\ 30,\ 1990}
 1
\shead{NAME}
rletorla -- convert a Utah RLE image file into a Wavefront "rlb" image file.
\shead{SYNOPSIS}
{\bf rletorla}
[
{\bf --h}
] [
{\bf --v}
] [
{\it infile}
]
\shead{DESCRIPTION}
This program converts Utah
{\it RLE}{\rm (5)}
image files into Wavefront rlb image files.  Rletorla will handle four types
of RLE files: Grayscale (8 bit data, no color map), Pseudocolor (8 bit data
with a color map), Truecolor (24 bit data with color map), and Directcolor (24
bit data, no color map).  In each case the resulting Wavefront image file will
contain RGB data as well as a matte channel.  If no alpha channel is found in
the RLE file, the Wavefront matte channel will be computed using the RGB or
mapped data.  The entire area of the Wavefront image will be run length encoded.
The size of the Wavefront "bounding box" data structure will be set to that of
the total image area.
\shead{OPTIONS}
\begin{TPlist}{{\bf --v}}
\item[{{\bf --v}}]
This option will cause rletorla to operate in verbose mode.  Header information
is printed to "stderr".
\item[{{\bf --h}}]
This option allows the header of the RLE file to be dumped to "stderr" without
converting the file.  It is equivalent to using the --v option except that no
file conversion takes place.
\item[{{\it infile}}]
The name of the RLE image data file to be converted.  The name of the resulting
Wavefront file will be derived from the name of the input file -- the extension
will be changed from "rle" to "rla".  (Note: if you use the extended
input file names described in
{\it urt}{\rm (1),}
this will result in a very strange filename for the Wavefront file.
\end{TPlist}\shead{EXAMPLES}
\begin{TPlist}{rletorla --v lenna.rle}
\item[{rletorla --v lenna.rle}]
While running in verbose mode, convert lenna.rle to Wavefront rlb format and
store resulting data in lenna.rla.
\item[{rletorla --h test.0001.rle}]
Dump the header information of the RLE file called test.0001.rle.
\end{TPlist}\shead{SEE ALSO}
{\it rlatorle}{\rm (1),}
{\it urt}{\rm (1),}
{\it RLE}{\rm (5).}
\shead{AUTHOR}
\nwl
Wesley C. Barris
\nwl
Army High Performance Computing Research Center (AHPCRC)
\nwl
Minnesota Supercomputer Center, Inc.
\newpage
% -*-LaTeX-*-
% Converted automatically from troff to LaTeX by tr2tex on Tue Aug  7 18:11:05 1990
% tr2tex was written by Kamal Al-Yahya at Stanford University
% (Kamal%Hanauma@SU-SCORE.ARPA)


%[troffman]{article}
%
%
% input file: rletotiff.1
%
% Copyright (c) 1990, University of Michigan
\phead{RLETOTIFF}{1}{July\ 3,\ 1990}
 1
\shead{NAME}
rletotiff -- Convert 24 bit RLE image files to TIFF.
\shead{SYNOPSIS}
{\bf rletotiff}
[
{\bf --\{cC}\}
]
{\bf --o} 
{\it outfile.tif}
[
{\bf --v}
] [ 
{\it infile.rle}
]
\shead{DESCRIPTION}
This program converts a 24 bit image in
{\it RLE}{\rm (5)}
format into 
{\it TIFF}
form.  Only a single image will be converted.
\shead{OPTIONS}
\begin{TPlist}{{\bf --\{cC}\}}
\item[{{\bf --\{cC}\}}]
Sets the type of compression used in the output file.  
{\bf --c}
(the default) will cause the output file to be compressed using the
Lempel-Ziv-Welch (LZW) algorithm.
{\bf --C}
will suppress any compression.
\item[{{\bf --o}{\it \ outfile.tif}
}]
The output will be written to this file.  
{\it outfile.tif}
must be a real file, the special cases described in
{\it urt} (1)
do not apply.  Note also that this "option" is not optional.  The
{\bf --o}
flag is required for consistency with the other tools.
\item[{{\bf --v}}]
Verbose output.
\item[{{\it infile.rle}}]
The input will be read from this file.  If
{\it infile.rle}
is "--" or is not specified, the input will be read from the standard
input stream.
\end{TPlist}\shead{SEE ALSO}
{\it tifftorle}{\rm (1),}
{\it urt}{\rm (1),}
{\it RLE}{\rm (5).}
\shead{AUTHOR}
Bailey Brown, University of Michigan.
\newpage
% -*-LaTeX-*-
% Converted automatically from troff to LaTeX by tr2tex on Tue Aug  7 18:11:05 1990
% tr2tex was written by Kamal Al-Yahya at Stanford University
% (Kamal%Hanauma@SU-SCORE.ARPA)


%[troffman]{article}
%
%
% input file: rlezoom.1
%
% Copyright (c) 1986, University of Utah
\phead{RLEZOOM}{1}{Feb\ 27,\ 1987}
 1
\shead{NAME}
rlezoom -- Magnify an RLE file by pixel replication.
\shead{SYNOPSIS}
{\bf rlezoom}
{\it factor}
[ 
{\it y-factor}
] [
{\bf --f}
] [
{\bf --o}
{\it outfile}
] [
{\it infile}
]
\shead{DESCRIPTION}
This program magnifies (zooms) an
{\it RLE}{\rm (5)}
file by a floating point factor.  Each pixel in the original image becomes
a block of pixels in the output image.  If no
{\it y-factor}
is specified, then the image will be magnified by
{\it factor}
equally in both directions.  If 
{\it y-factor}
is given, then each input pixel becomes a block of
{\it factor}{\rm ''i'u'''i'u'}{\it y-factor}
pixels in the output.  If
{\it factor}
or
{\it y-factor}
is less than 1.0, pixels will be dropped from the image.  There is no
pixel blending performed.  Input is taken from 
{\it infile}{\rm ,}
or from the standard input if not specified.  The magnified image is
written to the standard output, or
{\it outfile,}
if specified.

You should use
{\it rlezoom}
over
{\it fant}{\rm (1)}
if you just want a quick magnification of an image with the pixel
boundaries showing.  It is significantly faster than 
{\it fant}
because it does no arithmetic on the pixel values.
If you need blending between pixels in the magnified
image, then 
{\it fant}
is the correct program to use.  Use
{\it rlezoom\ --f\ factor\ y-factor}
to produce an image the same size as 
{\it fant\ --p\ 0\ 0\ --s\ factor\ y-factor}
for previewing purposes.

Note: due to the way that 
{\it scanargs}{\rm (3)}
parses the arguments from the command line, if the name of
{\it infile}
is a number, and it is in the current directory, you should prefix it
with "./" so that it will not be confused with
{\it factor}
or
{\it y-factor}{\rm .}
\shead{SEE ALSO}
{\it fant}{\rm (1),}
{\it urt}{\rm (1),}
{\it scanargs}{\rm (3),}
{\it RLE}{\rm (5).}
\shead{AUTHOR}
Spencer W. Thomas,
Gerald A. Winters.
\newpage
% -*-LaTeX-*-
% Converted automatically from troff to LaTeX by tr2tex on Tue Aug  7 18:11:06 1990
% tr2tex was written by Kamal Al-Yahya at Stanford University
% (Kamal%Hanauma@SU-SCORE.ARPA)


%[troffman]{article}
%
%
% input file: smush.1
%
% Copyright (c) 1986, University of Utah
% Template man page.  Taken from wtm's page for getcx3d
\phead{SMUSH}{1}{March\ 15,\ 1987}
 1
\shead{NAME}
smush -- defocus an RLE image.
\shead{SYNOPSIS}
{\bf smush}
% sample options...
[ 
{\bf --m}
{\it maskfile}
] [
{\bf --n}
] [
{\bf --o}
{\it outfile} 
] [ 
{\it levels}
] [
{\it infile} 
]
\shead{DESCRIPTION}
{\it Smush}
convolves an image with a 5x5 Gaussian mask, blurring the image.  One 
may also provide a mask in a text file.  The file must contain an integer
to specify the size of the square mask, followed by size*size floats.
The mask will be normalized (forced to sum to 1.0) unless the 
{\bf --n}
flag is given. 

The resulting image is the same size as the input image, no sub-sampling takes
place.  The levels option, which defaults to one, signifies the number of
times which the image will be blurred.  Each successive blurring is done with
a more spread out mask, so a 
{\it smush}
of level 2 is blurrier than piping two level one
{\it smush}
calls.
If no input file is specified, 
{\it smush}
reads from stdin.  If no output file is specified with
{\bf --o}
it writes the result to stdout.
\shead{SEE ALSO}
{\it avg4}{\rm (1),}
{\it urt}{\rm (1),}
{\it RLE}{\rm (5).}
\shead{AUTHOR}
Rod G. Bogart
\shead{BUGS}
{\it Smush}
should probably automatically generate different sized gaussians and other
common filters.



\newpage
% -*-LaTeX-*-
% Converted automatically from troff to LaTeX by tr2tex on Tue Aug  7 18:11:06 1990
% tr2tex was written by Kamal Al-Yahya at Stanford University
% (Kamal%Hanauma@SU-SCORE.ARPA)


%[troffman]{article}
%
%
% input file: targatorle.1
%
% Copyright (c) 1986, University of Utah
% Template man page.  Taken from wtm's page for getcx3d
\phead{TARGATORLE}{1}{September\ 23,\ 1987}
 1
\shead{NAME}
targatorle -- Convert Targa 32 TIPS images to RLE format.
\shead{SYNOPSIS}
{\bf targatorle}
% sample options...
[
{\bf --h}
{\it headerfile}
]
[
{\bf --o}
{\it outfile.rle}
] [
{\it infile.tga}
]
\shead{DESCRIPTION}
{\it Targatorle}
converts a file from Targa 32 TIPS format into RLE format.  Because TIPS and
RLE agree on which end is up,  
{\it rleflip}
is 
{\it not}
necessary to preserve orientation. If no input file is specified, the data is
read from stdin.
\shead{OPTIONS}
\begin{TPlist}{{\bf --h}}
\item[{{\bf --h}}]
Allow the program to write Targa header information to
{\it headerfile}
\item[{{\bf --o}}]
Use 
{\it outfile}
as output instead of 
{\it stdout.}
\end{TPlist}\shead{SEE ALSO}
{\it urt}{\rm (1),}
{\it RLE}{\rm (5).}
\shead{AUTHOR}
Hann-Bin Chuang
\newpage
% -*-LaTeX-*-
% Converted automatically from troff to LaTeX by tr2tex on Tue Aug  7 18:11:07 1990
% tr2tex was written by Kamal Al-Yahya at Stanford University
% (Kamal%Hanauma@SU-SCORE.ARPA)


%[troffman]{article}
%
%
% input file: template.1
%
% -*- Text -*-
% Copyright (c) 1990, University of Michigan
% Template man page.  
\phead{PROGNAME}{1}{Month\ DD,\ YYYY}
 1
\shead{NAME}
progname -- a prog for naming
\shead{SYNOPSIS}
{\bf progname}
% sample options, they should be alphabetized...
% Flags are bold, args are italics
[
{\bf --l}
] [
{\bf --o} 
{\it outfile}
] [
{\bf --p}
{\it x\ y}
] [
{\bf --v}
] [ 
{\it infile} 
]
\shead{DESCRIPTION}
This program accepts an
% .IR refers to another man entry
{\it RLE}{\rm (5)}
file and does something interesting with it.
% An option mentioned in-line
One nice feature is the %
\bf --l %
\rm option.
\shead{OPTIONS}
\begin{TPlist}{{\bf --o}{\it \ outfile}
}
\item[{{\bf --o}{\it \ outfile}
}]
If specified, the output will be written to this file.  If 
{\it outfile}
is "--", or if it is not specified, the output will be written to the
standard output stream.
\item[{{\bf --p}{\it \ x\ y}
}]
Reposition the image. 
\item[{{\bf --v}}]
Verbose output.
\item[{{\it infile}}]
The input will be read from this file.  If
{\it infile}
is "--" or is not specified, the input will be read from the standard
input stream.
\end{TPlist}\shead{FILES}
list files here
\shead{SEE ALSO}
{\it otherprog}{\rm (1),}
{\it urt}{\rm (1),}
{\it RLE}{\rm (5).}
\shead{AUTHOR}
Your name here
\shead{BUGS}
Dirty laundry here.
\newpage
% -*-LaTeX-*-
% Converted automatically from troff to LaTeX by tr2tex on Tue Aug  7 18:11:07 1990
% tr2tex was written by Kamal Al-Yahya at Stanford University
% (Kamal%Hanauma@SU-SCORE.ARPA)


%[troffman]{article}
%
%
% input file: tifftorle.1
%
% Copyright (c) 1990, University of Michigan
\phead{TIFFTORLE}{1}{July\ 3,\ 1990}
 1
\shead{NAME}
tifftorle -- Convert 24 bit TIFF image files to RLE.
\shead{SYNOPSIS}
{\bf tifftorle}
[
{\bf --o} 
{\it outfile.rle}
]
{\it infile.tif}

\shead{DESCRIPTION}
This program converts a 24 bit TIFF image to
{\it RLE}{\rm (5)}
format.
\shead{OPTIONS}
\begin{TPlist}{{\bf --o}{\it \ outfile.rle}
}
\item[{{\bf --o}{\it \ outfile.rle}
}]
If specified, the output will be written to this file.  If 
{\it outfile.rle}
is "--", or if it is not specified, the output will be written to the
standard output stream.
\item[{{\it infile.tif}}]
The input will be read from this file.
{\it infile.tif}
must be a real file, the special cases described in 
{\it urt}{\rm (1)}
do not apply here.
\end{TPlist}\shead{SEE ALSO}
{\it tifftorle}{\rm (1),}
{\it urt}{\rm (1),}
{\it RLE}{\rm (5).}
\shead{AUTHOR}
Bailey Brown, University of Michigan.
\newpage
% -*-LaTeX-*-
% Converted automatically from troff to LaTeX by tr2tex on Tue Aug  7 18:11:08 1990
% tr2tex was written by Kamal Al-Yahya at Stanford University
% (Kamal%Hanauma@SU-SCORE.ARPA)


%[troffman]{article}
%
%
% input file: to8.1
%
% Copyright (c) 1986, University of Utah
% Template man page.  Taken from wtm's page for getcx3d
\phead{TO8}{1}{Month\ DD,\ YYYY}
 1
\shead{NAME}
to8 -- Convert a 24 bit RLE file to eight bits using dithering.
\shead{SYNOPSIS}
{\bf to8}
[
{\bf --g}
{\it display\_gamma}
] [
{\bf --\{iI}\}
{\it image\_gamma}
] [
{\bf --o}
{\it outfile}
] [
{\it infile}
]
\shead{DESCRIPTION}
{\it To8}
Converts an image with 24 bit pixel values (eight bits each of red, green and
blue) to eight bits of color using a dithered color map (the special color
map is automatically written into the output file).
If no input file is specified, 
{\it to8}
reads from stdin.  If no output file is specified with
{\bf --o}
it writes the result to the standard output.

Other options allow control over the gamma, or contrast, of the image.
The dithering process assumes that the incoming image has a gamma of
1.0 (i.e., a 200 in the input represents an intensity twice that of
a 100.)  If this is not the case, the input values must be adjusted
before dithering via the
{\bf --i}
or 
{\bf --I}
option.  The input file may also specify the gamma of the image via a
picture comment (see below).  The output display is assumed to have a gamma of
2.5 (standard for color TV monitors).  This may be modified via the
{\bf --g}
option if a display with a different gamma is used.

{\it To8}
will put a picture comment into the output file indicating the display
gamma assumed in constructing the dithering color map.
\shead{OPTIONS}
\begin{TPlist}{{\bf --i}{\it \ image\_gamma}
}
\item[{{\bf --i}{\it \ image\_gamma}
}]
Specify the gamma (contrast) of the image.  A low contrast image,
suited for direct display without compensation on a high contrast
monitor (as most monitors are) will have a gamma of less than one.
The default image gamma is 1.0.  Image gamma may also be specified by
a picture comment in the
{\it RLE} (5)
file of the form
{\bf image\_gamma=}{\it gamma.}
The command line argument will override the value in the file if specified.
\item[{{\bf --I}{\it \ image\_gamma}
}]
An alternate method of specifying the image gamma, the number
following
{\bf --I}
is the gamma of the display for which the image was originally
computed (and is therefore 1.0 divided by the actual gamma of the
image).  Image display gamma may also be specified by
a picture comment in the
{\it RLE} (5)
file of the form
{\bf display\_gamma=}{\it gamma.}
The command line argument will override the value in the file if specified.
\item[{{\bf --g}{\it \ display\_gamma}
}]
Specify the gamma of the 
{\it X}
display monitor.  The default value is 2.5, suitable for most color TV
monitors (this is the gamma value assumed by the NTSC video standard).
\item[{{\bf --o}{\it \ outfile}
}]
If specified, the output will be written to this file.  If 
{\it outfile}
is "--", or if it is not specified, the output will be written to the
standard output stream.
\end{TPlist}\shead{SEE ALSO}
{\it tobw}{\rm (1),}
{\it getx11}{\rm (1),}
{\it mcut}{\rm (1),}
{\it rlequant}{\rm (1),}
{\it urt}{\rm (1),}
{\it dither}{\rm (3),}
{\it RLE}{\rm (5).}
\shead{AUTHOR}
Spencer Thomas
\newpage
% -*-LaTeX-*-
% Converted automatically from troff to LaTeX by tr2tex on Tue Aug  7 18:11:09 1990
% tr2tex was written by Kamal Al-Yahya at Stanford University
% (Kamal%Hanauma@SU-SCORE.ARPA)


%[troffman]{article}
%
%
% input file: tobw.1
%
% Copyright (c) 1986, University of Utah
\phead{TOBW}{1}{Month\ DD,\ YYYY}
 1
\shead{NAME}
tobw -- Convert a 24 bit RLE file to eight bits of gray scale value.
\shead{SYNOPSIS}
{\bf tobw}
[
{\bf --t}
] [
{\bf --o}
{\it outfile}
] [
{\it infile}
]
\shead{DESCRIPTION}
{\it Tobw}
converts an image with 24 bit pixel values (eight bits each of red,
green and blue) to eight bits of grayscale information.  The 
{\it NTSC} Y
transform is used.  If the
{\bf --t}
flag is given, then the monochrome pixel values are replicated on all three
output channels (otherwise, just one channel of eight bit data is produced).
If no input file is specified, 
{\it tobw}
reads from stdin.  If no output file is specified with
{\bf --o,}
it writes the result to stdout.
\shead{SEE ALSO}
{\it to8}{\rm (1),}
{\it urt}{\rm (1),}
{\it rgb\_to\_bw}{\rm (3),}
{\it RLE}{\rm (5).}
\shead{AUTHOR}
Spencer Thomas


\newpage
% -*-LaTeX-*-
% Converted automatically from troff to LaTeX by tr2tex on Tue Aug  7 18:11:09 1990
% tr2tex was written by Kamal Al-Yahya at Stanford University
% (Kamal%Hanauma@SU-SCORE.ARPA)


%[troffman]{article}
%
%
% input file: unexp.1
%
% Copyright (c) 1986, University of Utah
% Template man page.  Taken from wtm's page for getcx3d
\phead{UNEXP}{1}{November\ 8,\ 1987}
 1
\shead{NAME}
unexp -- Convert "exponential" files into normal files.
\shead{SYNOPSIS}
{\bf unexp}
[
{\bf --m}
{\it maxval}
] [
{\bf --o}
{\it outfile}
] [
{\bf --p}
] [
{\bf --s}
] [
{\bf --v}
] 
{\it infile}
\shead{DESCRIPTION}
{\it Unexp}
Converts a file of "exponential" floating point values into an 
{\it RLE}{\rm (5)}
file containing integer valued bytes.  Exponential files have N-1 channels of
eight bit data, with the Nth channel containing a common exponent for
the other channels.  This allows the values represented by the pixels to have 
a wider dynamic range.

If no maximum value is specified, 
{\it unexp}
first reads the RLE file to find
the dynamic range of the whole file.  It then rewinds the file and scales
the output to fit within that dynamic range.  If a maximum value is specified,
{\it unexp}
runs in one pass, and clamps any values exceeding the maximum.  

Files containing exponential data are expected to have a 
"exponential\_data" comment; 
{\it unexp}
prints a warning if such a comment
doesn't exist.  An exponential file should be %
\it unexp\rm%
'ed before
attempting to use any tools that perform arithmetic on pixels (e.g.,
{\it rlecomp}{\rm (1),}
{\it avg4}{\rm (1),}
{\it fant}{\rm (1),}
or
{\it applymap}{\rm (1))}
or displaying the image.

{\it Unexp}
does not allow piped input.  The
{\it infile}
must be a real file; the special filenames described in 
{\it urt}{\rm (1)}
are not allowed.  ("--" does work, as long as the input is coming from
a real file; this is of minimal utility, therefore, as typing 
{\it unexp\ -\ $<$foo.rle}
is harder than typing
{\it unexp\ foo.rle}{\rm .)}
\shead{OPTIONS}
\begin{TPlist}{{\bf --m}{\it \ maxval}
}
\item[{{\bf --m}{\it \ maxval}
}]
Specify the maximum value (i.e., the data in the file is assumed to be in the 
range 0..maxval).  Only the conversion pass is executed, and values found 
exceeding the maximum are clamped.
\item[{{\bf --o}{\it }{\bf outfile}
}]
If specified, the output will be written to this file.  If 
{\it outfile}
is "--", or if it is not specified, the output will be written to the
standard output stream.
\item[{{\bf --p}}]
Print the maximum value found during the scanning phase
\item[{{\bf --s}}]
Just scan the file to find the maximum, don't generate any output.
\item[{{\bf --v}}]
Verbose mode, print a message to stderr after scanning or converting every 
hundred scanlines.
\end{TPlist}\shead{SEE ALSO}
{\it float\_to\_exp}{\rm (3),}
{\it urt}{\rm (1),}
{\it RLE}{\rm (5).}
\shead{AUTHOR}
John W. Peterson
\shead{BUGS}
{\it Unexp}
is provided because of the lack of floating point or extended
precision RLE files.

The 
{\bf --v}
flag is a historical relict from the slow CPU days.
\newpage
% -*-LaTeX-*-
% Converted automatically from troff to LaTeX by tr2tex on Tue Aug  7 18:11:10 1990
% tr2tex was written by Kamal Al-Yahya at Stanford University
% (Kamal%Hanauma@SU-SCORE.ARPA)


%[troffman]{article}
%
%
% input file: unslice.1
%
% Copyright (c) 1987, University of Utah
\phead{UNSLICE}{1}{May\ 21,\ 1987}
 1
\shead{NAME}
unslice -- Quickly assemble image slices
\shead{SYNOPSIS}
{\bf unslice}
[
{\bf --f}
{\it ctlfile} 
] [
{\bf --y}
{\it ymax}
] [
{\bf --o}
{\it outfile}
] 
{\it infiles} ...
\shead{DESCRIPTION}
{\it Unslice}
quickly assembles a number of horizontal image strips into a single
output image.  A typical use for %
\it unslice \rm%
is to put together
portions of an image ("slices") computed independently into a single
output picture.  Because %
\it unslice \rm%
uses the "raw" RLE library
calls to read and write the images, it runs much faster than doing the
equivalent operations with crop and comp.

%
\it unslice \rm%
has two modes of operation.  If given the
{\bf --f}
flag, %
\it unslice \rm%
reads a control file telling it how to assemble
the images.  This is a text file with two decimal numbers on each
line, one line for each slice to be assembled into the output image.
Each line gives the starting and stopping scanlines (inclusive) for
each slice.  These must be in ascending order.  This is useful if the
slices have excess image area that should be cropped away.

If no control file is given, the
{\bf --y}
flag is used.  This tells %
\it unslice \rm%
what the maximum Y value of
the output image is.
{\it Unslice}
reads the files in order, using the RLE headers to 
determine where to place the slices.  If two slices overlap, the first 
scanlines from the second slice are thrown away.  In both cases, the slices
must be in ascending order, and are expected to be of uniform width.
\shead{SEE ALSO}
{\it crop}{\rm (1),}
{\it rlecomp}{\rm (1),}
{\it rlepatch}{\rm (1),}
{\it repos}{\rm (1),}
{\it urt}{\rm (1),}
{\it RLE}{\rm (5).}
\shead{AUTHOR}
John W. Peterson
\shead{BUGS}
{\it Unslice}
has really been superceded by 
{\it rlepatch}{\rm (1).}
\newpage
% -*-LaTeX-*-
% Converted automatically from troff to LaTeX by tr2tex on Tue Aug  7 18:11:10 1990
% tr2tex was written by Kamal Al-Yahya at Stanford University
% (Kamal%Hanauma@SU-SCORE.ARPA)


%[troffman]{article}
%
%
% input file: urt.1
%
% Copyright (c) 1990, University of Michigan
\phead{URT}{1}{June\ 17,\ 1990}
 1
\shead{NAME}
urt -- overview of the Utah Raster Toolkit
\shead{SYNOPSIS}
%.ta 11n
%
\bf applymap %
\rm Apply color map to image data.
\nwl
%
\bf avg4 %
\rm Simple 2x2 downsizing filter.
\nwl
%
\bf crop %
\rm Crop image.
\nwl
%
\bf cubitorle %
\rm Convert Cubicomp format to RLE.
\nwl
%
\bf dvirle %
\rm Typeset TeX ".dvi" files as RLE images.
\nwl
%
\bf fant %
\rm Image scale/rotate with anti-aliasing.
\nwl
%
\bf get4d %
\rm Display on SGI Iris/4D display.
\nwl
%
\bf get\_orion %
\rm Display on "Orion" display.
\nwl
%
\bf getap %
\rm Display on Apollo.
\nwl
%
\bf getbob %
\rm Display under HP window system.
\nwl
%
\bf getcx3d %
\rm Display RLE on Chromatics CX3D.
\nwl
%
\bf getfb %
\rm Display using BRL generic fb library.
\nwl
%
\bf getgmr %
\rm Display on Grinnell GMR-27 frame buffer.
\nwl
%
\bf getiris %
\rm Display on SGI 2400/3000 w/o window manager.
\nwl
%
\bf getmac %
\rm Display on Mac under MPW.
\nwl
%
\bf getmex %
\rm Display on SGI under the window manager.
\nwl
%
\bf getqcr %
\rm Display on Matrix QCR camera.
\nwl
%
\bf getren %
\rm Display on HP SRX.
\nwl
%
\bf getsun %
\rm Display using SunTools.
\nwl
%
\bf getx10 %
\rm Display on X10 display.
\nwl
%
\bf getx11 %
\rm Display using X11.
\nwl
%
\bf giftorle %
\rm Convert GIF files to RLE.
\nwl
%
\bf graytorle %
\rm Convert separate rrr ggg bbb files to RLE.
\nwl
%
\bf mcut %
\rm Median cut color quantization.
\nwl
%
\bf mergechan %
\rm Merge colors from multiple images.
\nwl
%
\bf painttorle %
\rm Convert MacPaint to RLE.
\nwl
%
\bf pgmtorle %
\rm Convert PBMPLUS pgm format to RLE.
\nwl
%
\bf ppmtorle %
\rm Convert PBMPLUS ppm format to RLE.
\nwl
%
\bf pyrmask %
\rm Generate "pyramid" filter mask.
\nwl
%
\bf rastorle %
\rm Convert Sun Raster to RLE.
\nwl
%
\bf rawtorle %
\rm Convert various raw formats to RLE.
\nwl
%
\bf read98721 %
\rm Read the screen of an HP 98721 "Renaissance" to an RLE file.
\nwl
%
\bf repos %
\rm Reposition an image.
\nwl
%
\bf rlatorle %
\rm Convert Wavefront RLA format to RLE.
\nwl
%
\bf rleClock %
\rm Draws a clock face.
\nwl
%
\bf rleaddcom %
\rm Add comments to an RLE file.
\nwl
%
\bf rleaddeof %
\rm Add an EOF code to an RLE file.
\nwl
%
\bf rlebg %
\rm Generate a "background".
\nwl
%
\bf rlebox %
\rm Find bounding box of an image.
\nwl
%
\bf rlecomp %
\rm Image composition.
\nwl
%
\bf rledither %
\rm Floyd-Steinberg dither an image to a given colormap.
\nwl
%
\bf rleflip %
\rm Flip an image or rotate it 90.
\nwl
%
\bf rlehdr %
\rm Print info about an RLE file.
\nwl
%
\bf rlehisto %
\rm Make a histogram of an image.
\nwl
%
\bf rleldmap %
\rm Load a new colormap into a file.
\nwl
%
\bf rlemandl %
\rm Make a Mandelbrot image.
\nwl
%
\bf rlenoise %
\rm Add noise to an image.
\nwl
%
\bf rlepatch %
\rm Patch smaller images on a big one.
\nwl
%
\bf rleprint %
\rm Print all pixel values in image.
\nwl
%
\bf rlequant %
\rm Variance based color quantization.
\nwl
%
\bf rlescale %
\rm Generate a "gray scale".
\nwl
%
\bf rleselect %
\rm Select images from an RLE file.
\nwl
%
\bf rlesetbg %
\rm Set the background color of an image file.
\nwl
%
\bf rleskel %
\rm Skeleton tool.  Programming example.
\nwl
%
\bf rlespiff %
\rm Simple contrast enhancement.
\nwl
%
\bf rlesplice %
\rm Splice two images horizontally or vertically.
\nwl
%
\bf rlesplit %
\rm Split concatenated images into files.
\nwl
%
\bf rleswap %
\rm Swap or select color channels.
\nwl
%
\bf rletoabA60 %
\rm Convert RLE to Abekas A60 format.
\nwl
%
\bf rletoabA62 %
\rm Convert to Abekas A62 format.
\nwl
%
\bf rletoascii %
\rm Make a line-printer/CRT version of an RLE image.
\nwl
%
\bf rletogif %
\rm Convert RLE images to GIF format.
\nwl
%
\bf rletogray %
\rm Convert RLE to separate rrr ggg bbb files.
\nwl
%
\bf rletopaint %
\rm Convert RLE to MacPaint.
\nwl
%
\bf rletoppm %
\rm Convert RLE to PBMPLUS ppm format.
\nwl
%
\bf rletops %
\rm Convert RLE to (B\&W) PostScript.
\nwl
%
\bf rletorast %
\rm Convert RLE to Sun Raster.
\nwl
%
\bf rletoraw %
\rm Convert RLE to rgbrgb raw format.
\nwl
%
\bf rletorla %
\rm Convert RLE to Wavefront RLA format.
\nwl
%
\bf rletotiff %
\rm Convert RLE to TIFF 24 bit format.
\nwl
%
\bf rlezoom %
\rm Scale image by sub- or super-sampling.
\nwl
%
\bf smush %
\rm Generic filtering.
\nwl
%
\bf targatorle %
\rm Convert TARGA to RLE.
\nwl
%
\bf tifftorle %
\rm Convert TIFF 24 bit images to RLE.
\nwl
%
\bf to8 %
\rm 24 to 8 bit ordered dither color conversion.
\nwl
%
\bf tobw %
\rm ColorB\&W conversion.
\nwl
%
\bf unexp %
\rm Convert "exp" format to normal colors.
\nwl
%
\bf unslice %
\rm Paste together "slices" into a full image.
\nwl
%
\bf wasatchrle %
\rm Convert Wasatch paint system to RLE.

\shead{DESCRIPTION}
The 
{\it Utah} Raster Toolkit
is a collection of programs and C routines for
dealing with raster images commonly encountered in computer graphics.
A device and system independent image format stores images
and information about them.  Called the 
{\it RLE}{\rm (5)}
format, it uses
run length encoding to reduce storage space for most images.

The programs (tools) currently included in the toolkit are listed above,
together with a short description of each one.  Most of the tools read
one or more input RLE files and produce an output RLE file.  Some
generate RLE files from other information, and some read RLE files and
produce output of a different form.

An input file is almost always specified by mentioning its name on the
command line.  Some commands, usually those which take an indefinite
number of non-file arguments (e.g.,
{\it rleaddcom}{\rm )}
require a
{\bf --i}
flag to introduce the input file name.
If the input file name is absent the tool will usually
read from the standard input.  An input file name of "--" also signals
that the input should be taken from the standard input.

On Unix
systems, there are two other specially treated file name forms.  A
file name starting with the character '$|$' will be passed to
{\it sh}{\rm (1)}
to run as a command.  The output from the command will be read by the
tool.  A file whose name ends in ".Z" (and which does not begin with
a '$|$') will be decompressed by the
{\it compress}{\rm (1)}
program.  Both of these options supply input to the tool through a
pipe.  Consequently, certain programs (those that must read their
input twice) cannot take advantage of these features.  This is noted
in the manual pages for the affected commands.

An output file is almost always specified using the option
{\bf --o}
{\it outfile}{\rm .}
If the option is missing, or if 
{\it outfile}
is "--", then the output will be written to the standard output.

On Unix systems, the special file name forms above may also be used
for output files.  File names starting with '$|$' are taken as a command
to which the tool output will be sent.  If the file name ends in ".Z",
then
{\it compress}
will be used to produce a compressed output file.

Several images may be concatenated together into a single file, and
most of the tools will properly process all the images.  Those that
will not are noted in their respective man pages.

{\bf Picture} comments.
Images stored in %
\it RLE \rm%
form may have attached comments.  There are
some comments that are interpreted, created or manipulated by certain
of the tools.  In the list below, a word enclosed in $<$$>$ is a
place-holder for a value.  The $<$$>$ do not appear in the actual comment.
\begin{TPlist}{%
\it image\_gamma=$<$float number$>$\rm%
}
\item[{%
\it image\_gamma=$<$float number$>$\rm%
}]
Images are sometimes computed with a particular ``gamma'' value --
that is, the pixel values in the image are related to the actual
intensity by a power law, %
\it pixel\_value=intensity\^{}image\_gamma\rm%
.
Some of the display programs, and the %
\it buildmap\rm%
(3) function will
look for this comment and automatically build a "compensation table"
to transform the pixel values back to true intensity values.
\item[{%
\it display\_gamma=$<$float number$>$\rm%
}]
The %
\it display\_gamma \rm%
is just %
\it 1/image\_gamma\rm%
.  That is, it is
the ``gamma'' of the display for which the image was computed.  If an
%
\it image\_gamma \rm%
comment is not present, but a %
\it display\_gamma \rm%
is, the displayed image will be gamma corrected as above.  The
%
\it to8 \rm%
program produces a %
\it display\_gamma \rm%
comment.
\item[{%
\it colormap\_length=$<$integer$>$\rm%
}]
The length of the colormap stored in the %
\it RLE \rm%
header must be a
power of two.  However, the number of useful entries in the colormap
may be smaller than this.  This comment can be used to tell some
of the display programs (%
\it getx11\rm%
, in particular) how many of
the colormap entries are used.  The assumption is that entries 0 --
%
\it colormap\_length--1 \rm%
are used.  This comment is produced by
%
\it mcut\rm%
, %
\it rlequant\rm%
, and %
\it rledither\rm%
.
\item[{%
\it image\_title=$<$string$>$\rm%
}]
This comment is used by %
\it getx11 \rm%
to set the window title.  If
present, the comment is used instead of the file name.  (No other
programs currently pay attention to this comment.)  The comments
%
\it IMAGE\_TITLE\rm%
, %
\it title\rm%
, and %
\it TITLE \rm%
are also recognized,
in that order.  No programs produce this comment.
\item[{%
\it HISTORY=$<$string$>$\rm%
}]
All toolkit programs (with the exception of rleaddcom) create or add
to a %
\it HISTORY \rm%
comment.  Each tool appends a line to this comment
that contains its command line arguments and the time it was run.
Thus, the image contains a history of all the things that were done to
it.  No programs interpret this comment.
\item[{%
\it exponential\_data\rm%
}]
This comment should be present in a file stored in ``exponential''
form.  See %
\it unexp\rm%
(1) and %
\it float\_to\_exp\rm%
(3) for more
information.  The %
\it unexp \rm%
program expects to see this comment.
\end{TPlist}\shead{SEE ALSO}
{\it compress}{\rm (1),}
{\it sh}{\rm (1),}
{\it RLE}{\rm (5).}
\shead{AUTHOR}
Many people contributed to the Utah Raster Toolkit.  This manual page
was written by Spencer W. Thomas, University of Michigan.

\newpage
% -*-LaTeX-*-
% Converted automatically from troff to LaTeX by tr2tex on Tue Aug  7 18:11:11 1990
% tr2tex was written by Kamal Al-Yahya at Stanford University
% (Kamal%Hanauma@SU-SCORE.ARPA)


%[troffman]{article}
%
%
% input file: wasatchrle.1
%
% Copyright (c) 1986, University of Utah
\phead{WASATCHRLE}{1}{December\ 20,\ 1987}
 1
\shead{NAME}
wasatchrle -- Convert Wasatch Systems image files to RLE format
\shead{SYNOPSIS}
{\bf wasatchrle} 
[
{\bf --o}
{\it outfile}
]
{\it basename}
\shead{DESCRIPTION}
{\it Wasatchrle}
converts image files generated by the Wasatch Systems Paint program
to RLE format.  It expects to find two files, 
"%
\it basename%
\rm .lut" (the color look-up table) and "%
\it basename%
\rm .rlc" 
(the run-length encoded data).

{\it Wasatchrle}
generates as output a single channel RLE image with a full
color map.  Since the Wasatch Paint program's origin is the top left
of the image, the results should be passed through
{\it rleflip} --v
to correctly orient the image.  If the image is to be used with other
toolkit operations (e.g., compositing), it should first be run through
{\it applymap}{\rm (1)}
to convert the image to a full color (three channel) RLE
file.
\shead{SEE ALSO}
{\it rleflip(1),}
{\it applymap(1),}
{\it urt}{\rm (1),}
{\it RLE}{\rm (5),}
\nwl
Wasatch Systems, "Wasatch Raster Image File Definition for Wasatch
Illustration Software (Version 1.2 and Later)"
\shead{AUTHOR}
John W. Peterson
\end{document}
