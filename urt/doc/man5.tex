% -*-LaTeX-*-
% Converted automatically from troff to LaTeX by tr2tex on Tue Aug  7 18:06:10 1990
% tr2tex was written by Kamal Al-Yahya at Stanford University
% (Kamal%Hanauma@SU-SCORE.ARPA)


\documentstyle[troffman]{article}
\begin{document}
%
% input file: rle.5
%
% Copyright (c) 1986, University of Utah
\phead{RLE}{5}{9/14/82}
 5
% $Header: /usr/users/spencer/src/urt/man/man5/RCS/rle.5,v 3.0 90/08/03 15:33:11 spencer Exp $
\shead{NAME}
rle -- Run length encoded file format produced by the rle library
\shead{DESCRIPTION}
The output file format is (note: all words are 16 bits, and in PDP-11 byte 
order):
\begin{TPlist}{{\bf Word} 0}
\item[{{\bf Word} 0}]
A "magic" number 0xcc52.  (Byte order 0x52, 0xcc.)
\item[{{\bf Words} 1-4}]
The structure (chars saved in PDP-11 order)

\nofill
\{
    short   xpos,                       /* Lower left corner
            ypos,
            xsize,                      /* Size of saved box
            ysize;
\}
\fill
\item[{{\bf Byte} 10}]
{\it (flags)}
The following flags are defined:
\ind{1\parindent} 0.5i
\item[{{\it H\_CLEARFIRST}}]
(0x1) If set, clear the frame buffer to background color before restoring.
\item[{{\it H\_NO\_BACKGROUND}}]
(0x2) If set, no background color is supplied.  If
{\it H\_CLEARFIRST}
is also set, it should be ignored (or alternatively, a clear-to-black
operation could be performed).
\item[{{\it H\_ALPHA}}]
(0x4) If set, an alpha channel is saved as color channel -1.  The alpha
channel does not contribute to the count of colors in
{\it ncolors}{\rm .}
\item[{{\it H\_COMMENT}}]
(0x8) If set, comments will follow the color map in the header.
\ind{0\parindent}
\item[{{\bf Byte} 11}]
{\it (ncolors)}
Number of color channels present.  0 means load only the color map (if
present), 1 means a B\&W image, 3 means a normal color image.
\item[{{\bf Byte} 12}]
{\it (pixelbits)}
Number of bits per pixel, per color channel.  Values greater than 8
currently will not work.
\item[{{\bf Byte} 13}]
{\it (ncmap)}
Number of color map channels present.  Need not be identical to
{\it ncolors}{\rm .}
If this is non-zero, the color map follows immediately after the background
colors.
\item[{{\bf Byte} 14}]
{\it (cmaplen)}
Log base 2 of the number of entries in the color map for each color channel.
I.e., would be 8 for a color map with 256 entries.
\item[{{\bf Bytes} 15--...}]
The background color.  There are 
{\it ncolors}
bytes of background color.  If
{\it ncolors}
is even, an extra padding byte is inserted to end on a 16 bit boundary.
The background color is only present if
{\it H\_NO\_BACKGROUND}
is not set in
{\it flags}{\rm .}
IF
{\it H\_NO} BACKGROUND
is set, there is a single filler byte.  Background color is ignored, but
present, if
{\it H\_CLEARFIRST}
is not set in
{\it flags}{\rm .}

If 
{\it ncmap}
is non-zero, then the color map will follow as
{\it ncmap}{\rm *2\^{}}{\it cmaplen}
16 bit words.  The color map data is left justified in each word.

If the
{\it H\_COMMENT}
flag is set, a set of comments will follow.  The first 16 bit word
gives the length of the comments in bytes.  If this is odd, a filler
byte will be appended to the comments.  The comments are interpreted
as a sequence of null terminated strings which should be, by
convention, of the form
{\it name}{\rm =}{\it value}{\rm ,}
or just
{\it name}{\rm .}

Following the setup information is the Run Length Encoded image.  Each
instruction consists of an opcode, a datum and possibly one or
more following words (all words are 16 bits).  The opcode is encoded in the
first byte of the instruction word.  Instructions come in either a short or
long form.  In the short form, the datum is in the second byte of the
instruction word; in the long form, the datum is a 16 bit value in the word
following the instruction word.  Long form instructions are distinguished by
having the 0x40 bit set in the opcode byte.
The instruction opcodes are:
\item[{{\bf SkipLines} (1)}]
The datum is an unsigned number to be added to the current Y position.
\item[{{\bf SetColor} (2)}]
The datum indicates which color is to be loaded with the data described by the
following ByteData and RunData instructions.  Typically,
0red, 1green, 2blue.  The
operation also resets the X position to the initial X (i.e. a carriage return
operation is performed).
\item[{{\bf SkipPixels} (3)}]
The datum is an unsigned number to be added to the current X
position.
\item[{{\bf ByteData} (5)}]
The datum is one less than the number of bytes of color data following.  If the
number of bytes is odd, a filler byte will be appended to the end of the byte
string to make an integral number of 16-bit words.  The X position is
incremented to follow the last byte of data.
\item[{{\bf RunData} (6)}]
The datum is one less than the run length.  The following word contains (in its
lower 8 bits) the color of the run.  The X position is incremented to follow
the last byte in the run.
\item[{{\bf EOF} (7)}]
This opcode indicates the logical end of image data.  A physical
end-of-file will also serve as well.  The 
{\bf EOF}
opcode may be used to concatenate several images in a single file.
\end{TPlist}\shead{SEE ALSO}
{\it librle(3)}
\shead{AUTHOR}
\par
Spencer W. Thomas, Todd Fuqua
\end{document}
